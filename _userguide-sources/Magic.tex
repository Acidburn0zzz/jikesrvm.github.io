\setNextFileName{Magic.html}
\begin{chapter}{Magic}
\label{cha:magic}

Most Java runtimes rely upon the foreign language APIs of the underlying platform operating system to implement runtime behaviour which involves interaction with the underlying platform. Runtimes also occasionally employ small segments of machine code to provide access to platform hardware state. Note that this is expedient rather than mandatory. With a suitably smart Java bytecode compiler it would be quite possible to implement a full Java-in-Java runtime i.e. one comprising only compiled Java code (the JNode project is an attempt to implement a runtime along these lines; the Xerox, MIT, Lambda and TI Explorer Lisp machine implementations and the Xerox Smalltalk implementation were highly successful attemtps at fully compiled language runtimes).

This section provides information on \textcolor{red}{$\bigstar$} magic \textcolor{red}{$\bigstar$} which is an escape hatch that Jikes\textsuperscript{TM} RVM provides to implement functionality that is not possible using the pure Java\textsuperscript{TM} programming language. For example, the Jikes RVM garbage collectors and runtime system must, on occasion, access memory or perform unsafe casts. The compiler will also translate a call to Magic.threadSwitch() into a sequence of machine code that swaps out old thread registers and swaps in new ones, switching execution to the new thread's stack resumed at its saved PC

There are three mechanisms via which the Jikes RVM \textcolor{red}{$\bigstar$} magic \textcolor{red}{$\bigstar$} is implemented:
\begin{itemize}
  \item Compiler Intrinsics: Most methods are within class librarys but some functions are built in (that is, intrinsic) to the compiler. These are referred to as intrinsic functions or intrinsics.
  \item Compiler Pragmas: Some intrinsics do not provide any behaviour but instead provide information to the compiler that modifies optimizations, calling conventions and activation frame layout. We refer to these mechanisms as compiler pragmas.
  \item Unboxed Types: Besides the primitive types, all Java values are boxed types. Conceptually, they are represented by a pointer to a heap object. However, an unboxed type is represented by the value itself. All methods on an unboxed type must be Compiler Intrinsics.
\end{itemize}

The mechanisms are used to implement the following functionality:
\begin{itemize}
  \item \hyperref[sec:rawmemoryaccess]{RawMemoryAccess}: Unfetted access to memory.
  \item \hyperref[sec:uninterruptiblecode]{Uninterruptible Code}: Declaring code to be uninterruptible.
  \item Alternative Calling Conventions: Declaring different calling conventions and activation frame layout. This is done via annotations, see the \texttt{org.vm\-ma\-gic.prag\-ma} package.
\end{itemize}

\setNextFileName{CompilerIntrinsics.html}
\begin{section}{Compiler Intrinsics}
\label{sec:compilerintrinsics}

A compiler intrinsic will usually generate a specific code sequence. The code sequence will usually be inlined and optimized as part of compilation phase of the optimizing compiler.

\begin{subsubsection}{Magic}

All the methods in \spverb+Magic+ are compiler intrinsics. Because these methods access raw memory or other machine state, perform unsafe casts, or are operating system calls, they cannot be implemented in Java code.

A Jikes\textsuperscript{TM} RVM implementor must be \textit{extremely careful} when writing code that uses \spverb+Magic+ to circumvent the Java type system. The use of \texttt{Ma\-gic.ob\-ject\-As\-Address} to perform various forms of pointer arithmetic is especially hazardous, since it can result in pointers being "lost" during garbage collection. All such uses of magic must either occur in uninterruptible methods or be guarded by calls to \spverb+VM.disableGC+ and \spverb+VM.enableGC+. The optimizing compiler performs aggressive inlining and code motion, so not explicitly marking such dangerous regions in one of these two manners will lead to disaster.

Since magic is inexpressible in the Java programming language , it is unsurprising that the bodies of \spverb+Magic+ methods are undefined. Instead, for each of these methods, the Java instructions to generate the code is stored in \texttt{Ge\-ne\-ra\-te\-Ma\-gic} and \texttt{Ge\-ne\-ra\-te\-Ma\-chi\-ne\-Spe\-ci\-fic\-Ma\-gic} (to generate HIR) and the baseline compilers (to generate assembly code) (Note: The optimizing compiler always uses the set of instructions that generate HIR; the instructions that generate assembly code are only invoked by the baseline compiler.). Whenever the compiler encounters a call to one of these magic methods, it inlines appropriate code for the magic method into the caller method.

\end{subsubsection}

\begin{subsubsection}{sun.misc.Unsafe}

The methods of \spverb+sun.misc.Unsafe+ are not treated specially by the compilers. The Jikes RVM ships a custom \spverb+sun.misc.Unsafe+ implementation that implements the operations with Jikes RVM magics and internal helper routines.

\end{subsubsection}

\end{section}


\setNextFileName{UnboxedTypes.html}
\begin{section}{Unboxed Types}
\label{sec:unboxedtypes}

If a type is boxed then it means that values of that type are represented by a pointer to a heap object. An unboxed type is represented by the value itself such as \spverb+int+, \spverb+double+, \spverb+float+, \spverb+byte+ etc.

In the Jikes RVM terminology, an unboxed type is a custom unboxed type. Normal Java primitives such as \spverb+int+ are never referred to as unboxed types.

The Jikes RVM also defines a number of unboxed types. Due to a limitation of the way the compiler generates code the Jikes RVM must define an unboxed array type for each unboxed type. The unboxed types are:
\begin{itemize}
  \item \spverb+org.vmmagic.unboxed.Address+
  \item \spverb+org.vmmagic.unboxed.Extent+
  \item \spverb+org.vmmagic.unboxed.ObjectReference+
  \item \spverb+org.vmmagic.unboxed.Offset+
  \item \spverb+org.vmmagic.unboxed.Word+
  \item \spverb+org.jikesrvm.ArchitectureSpecific.Code+
\end{itemize}

Values of unboxed types appear only in the virtual machine's stack, registers, or as fields/elements of class/array instances.

Unboxed types may inherit from Object but they are not objects. As such there are some restrictions on the use of unboxed types:
\begin{itemize}
  \item A unboxed type instance must not be passed where an \spverb+Object+ is expected. This will type-check, but it is not what you want. A corollary is to avoid overloading a method where the two overloaded versions of the method can only be distinguished by operating on an Object versus an unboxed type. The optimizing compiler can detect \textit{some} invalid uses of unboxed types.
  \item An unboxed type must not be synchronized on.
  \item They have no virtual methods.
  \item They do not support lock operations, generating hashcodes or any other method inherited from \spverb+Object+.
  \item All methods must be compiler intrinsics.
  \item  Avoid making an array of an unboxed type. Instead represent it by the array version of unboxed type. i.e. \spverb+org.vmmagic.unboxed.Address[]+ should be replaced with \texttt{org.vm\-ma\-gic.un\-box\-ed.Address\-Ar\-ray} but \texttt{org.vm\-ma\-gic.un\-box\-ed.Address\-Ar\-ray[]} is fine.
\end{itemize}


\end{section}


\setNextFileName{RawMemoryAccess.html}
\begin{section}{Raw Memory Access}
\label{sec:rawmemoryaccess}

The type \verb+org.vmmagic.Address+ is used to represent a machine-dependent address type. \verb+org.vmmagic.Address+ is an unboxed type. In the past, the base type \verb+int+ was used to represent addresses but this approach had several shortcomings. First, the lack of abstraction makes porting nightmarish. Equally important is that Java type \verb+int+ is signed whereas addresses are more appropriately considered unsigned. The difference is problematic since an unsigned comparison on \verb+int+ is inexpressible in the Java programming language.

To overcome these problems, instances of \verb+org.vmmagic.Address+ are used to represent addresses. The class supports the expected well-typed methods like adding an integer offset to an address to obtain another address, computing the difference of two addresses, and comparing addresses. Other operations that make sense on \verb+int+ but not on addresses are excluded like multiplication of addresses. Two methods deserve special attention: converting an address into an integer and the inverse. These methods should be avoided where possible.

Without special intervention, using a Java object to represent an address would be at best abysmally inefficient. Instead, when the Jikes RVM compiler encounters creation of an address object, it will return the primitive value that represents an address for that platform. Currently, the address type maps to either a 32-bit or 64-bit unsigned integer. Since an address is an unboxed type it must obey the rules outlined in Unboxed Types.

\end{section}

\setNextFileName{UninterruptibleCode.html}
\begin{section}{Uninterruptible Code}
\label{sec:uninterruptiblecode}

Declaring a method uninterruptible enables a Jikes RVM developer to prevent the Jikes RVM compilers from inserting "hidden" thread switch points in the compiled code for the method. As a result, the code can be written assuming that it cannot involuntarily "lose control" while executing due to a timer-driven thread switch. In particular, neither yield points nor stack overflow checks will be generated for uninterruptible methods.
When writing uninterruptible code, the programmer is restricted to a subset of the Java language. The following are the restrictions on uninterruptible code.

\begin{itemize}
  \item Because a stack overflow check represents a potential yield point (if GC is triggered when the stack is grown), stack overflow checks are omitted from the prologues of uninterruptible code. As a result, all uninterruptible code must be able to execute in the stack space available to them when the first uninterruptible method on the call stack is invoked. This is typically about 8K for uninterruptible regions called from mutator code. The collector threads must preallocate enough stack space, since all collector code is uninterruptible. As a result, using recursive methods in the GC subsystem is a bad idea.
  \item Since no yield points are inserted in uninterruptible code, there will be no timer-driven thread switches while executing it. So, if possible, one should avoid "long running" uninterruptible methods outside of the GC subsystem.
  \item Certain bytecodes are forbidden in uninterruptible code, because Jikes RVM cannot implement them in a manner that ensures uninterruptibility. The forbidden bytecodes are:
    \begin{itemize}
      \item \textit{aastore}
      \item \textit{invokeinterface}
      \item \textit{new}
      \item \textit{newarray}
      \item \textit{anewarray}
      \item \textit{athrow}
      \item \textit{checkcast} and \textit{instanceof} unless the LHS type is a final class
      \item \textit{monitorenter}
      \item \textit{monitorexit}
      \item \textit{multianewarray}
    \end{itemize}
  \item Uninterruptible code cannot cause class loading and thus must not contain unresolved \textit{getstatic}, \textit{putstatic}, \textit{getfield}, \textit{putfield}, \textit{invokevirtual}, or \textit{invokestatic} bytecodes.
  \item Uninterruptible code cannot contain calls to interruptible code. As a consequence, it is illegal to override an uninterruptible virtual method with an interruptible method.
  \item Uninterruptible methods cannot be \spverb+synchronized+. If you need synchronization in an uninterruptible method, you must use one of the internal locks or synchronization primitives.
\end{itemize}

We have augmented the baseline compiler to print a warning message when one of these restrictions is violated. The optimizing compiler currently does not check for uninterruptibility violations. Consequently, it is a good idea to compile a boot image with the baseline compiler (e.g. using prototype-opt) after modifying uninterruptible code.

If uninterruptible code were to raise a runtime exception such as \texttt{Null\-Poin\-ter\-Ex\-ce\-ption}, \texttt{Ar\-ray\-In\-dex\-Out\-Of\-Bounds\-Ex\-ce\-ption}, or \texttt{Class\-Cast\-Ex\-ce\-ption}, then it could be interrupted. We assume that such conditions are a programming error (or VM bug) and do not flag bytecodes that might result in one of these exceptions being raised as a violation of uninterruptibility.

In a few cases it is necessary to modify the conditions of checking for uninterruptibility to avoid spurious warning messages. This should be done with extreme care. The checking conditions for a particular method can be modified by using one of the following annotations:
\begin{itemize}
  \item \texttt{org.vm\-ma\-gic.prag\-ma.Un\-in\-ter\-rup\-ti\-ble\-No\-Warn} - disables checking for uninterruptibility violations but behaves like \texttt{org.vm\-ma\-gic.pra\-gma.Un\-in\-ter\-rup\-ti\-ble} otherwise. Used for methods that need to be uninterruptible but are \textbf{only} executed when writing the boot image.
  \item \texttt{org.vm\-ma\-gic.pra\-gma.Un\-pre\-emp\-tib\-le} - instructs the JVM to avoid inserting operations that could trigger garbage collection or thread switching but does not disallow them. Calls to preemptible code will cause warnings. This is used for code that is involved in thread scheduling, locking or the creation of exception objects.
  \item \texttt{org.vm\-ma\-gic.pra\-gma.Un\-pre\-emp\-tib\-le\-No\-Warn} - used for unpreemptible code that calls interruptible code.
\end{itemize}

Do not use the annotation \texttt{org.vm\-ma\-gic.prag\-ma.Lo\-gi\-cal\-ly\-Un\-in\-ter\-rup\-ti\-b\-le}. Its usage is being \href{https://xtenlang.atlassian.net/browse/RVM-115}{phased out}.

The following rules determine whether or not a method is uninterruptible.
\begin{itemize}
  \item All class initializers are interruptible, since they can only be invoked during class loading.
  \item All object constructors are interruptible, since they an only be invoked as part of the implementation of the new bytecode.
  \item If a method is annotated with \texttt{org.vm\-ma\-gic.prag\-ma.In\-ter\-rup\-ti\-ble} then it is interruptible.
  \item If none of the above rules apply and a method is annotated with \texttt{org.vm\-ma\-gic.prag\-ma.Un\-in\-ter\-rup\-ti\-ble}, then it is uninterruptible.
  \item If none of the above rules apply and the declaring class is annotated with \texttt{org.vm\-ma\-gic.prag\-ma.Un\-in\-ter\-rup\-ti\-ble} then it is uninterruptible.
\end{itemize}

Whether to annotate a class or a method with \texttt{org.vm\-ma\-gic.prag\-ma.Un\-in\-ter\-rup\-ti\-ble} is a matter of taste and mainly depends on the ratio of interruptible to uninterruptible methods in a class. If most methods of the class should be uninterruptible, then annotating the class is preferred.

\end{section}


\end{chapter}
