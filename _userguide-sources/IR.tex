\setNextFileName{IR.html}
\begin{subsection}{IR}
\label{subsec:ir}

The optimizing compiler intermediate representation (IR) is held in an object of type \spverb#IR# and includes a list of instructions. Every instruction is classified into one of the pre-defined instruction formats. Each instruction includes an operator and zero or more operands. Instructions are grouped into basic blocks; basic blocks are constrained to having control-flow instructions at their end. Basic blocks fall-through to other basic blocks or contain branch instructions that have a destination basic block label. The graph of basic blocks is held in the \spverb#cfg# (control-flow graph) field of IR.

This section documents basic information about the intermediate represenation. For a tutorial based introduction to the material it is highly recommended that you read the presentation \href{http://www.jikesrvm.org/Resources/Presentations/}{Jikes RVM Optimizing Compiler Intermediate Code Representation}.

\begin{subsubsection}{IR Operators}

The IR operators are defined by the class \spverb#Operators#, which in turn is automatically generated from a template by a driver. The input to the driver are two files, both called \spverb#OperatorList.dat#. One input file resides in
\spverb#$RVM_ROOT/rvm/src-generated/opt-ir# and defines machine-independent operators. The other resides in
\spverb#$RVM_ROOT/rvm/src-generated/opt-ir/$\{arch\}# and defines machine-dependent operators, where \spverb#$\{arch\}# is the specific instruction architecture of interest.

Each operator in \spverb#OperatorList.dat# is defined by a five-line record, consisting of:

\begin{itemize}
  \item \spverb#SYMBOL#: a static symbol to identify the operator
  \item \spverb#INSTRUCTION_FORMAT#: the instruction format class that accepts this operator.
  \item \spverb#TRAITS#: a set of characteristics of the operator, composed with a bit-wise or (\textbar ) operator. See Operator.java for a list of valid traits.
  \item \spverb#IMPLDEFS#: set of registers implicitly defined by this operator; usually applies only to machine-dependent operators
  \item \spverb#IMPLUSES#: set of registers implicitly used by this operator; usually applies only to machine-dependent operators
\end{itemize}

For example, the entry in \spverb#OperatorList.dat# that defines the integer addition operator is
\begin{lstlisting}
INT_ADD
Binary
none
<blank line>
<blank line>
\end{lstlisting}

The operator for a conditional branch based on values of two references is defined by
\begin{lstlisting}
REF_IFCOMP
IntIfCmp
branch | conditional
<blank line>
<blank line>
\end{lstlisting}
Additionally, the machine-specific \spverb+OperatorList.dat+ file contains another line of information for use by the assembler. See the file for details.

\end{subsubsection}


\begin{subsubsection}{Instruction Format}

Every IR instruction fits one of the pre-defined \textit{Instruction Formats}. The Java package \spverb#org.jikesrvm.compilers.opt.ir# defines roughly 75 architecture\hyp independent instruction formats. For each instruction format, the package includes a class that defines a set of static methods by which optimizing compiler code can access an instruction of that format.

For example, \spverb#INT_MOVE# instructions conform to the \spverb#Move# instruction format. The following code fragment shows code that uses the \spverb#Operators# interface and the \spverb#Move# instruction format:

\begin{lstlisting}[language=Java]
import org.jikesrvm.compilers.opt.ir.*;
class X {
  void foo(Instruction s) {
    if (Move.conforms(s)) {     // if this instruction fits the Move format
      RegisterOperand r1 = Move.getResult(s);
      Operand r2 = Move.getVal(s);
      System.out.println("Found a move instruction: " + r1 + " := " + r2);
    } else {
      System.out.println(s + " is not a MOVE");
    }
  }
}
\end{lstlisting}

This example shows just a subset of the access functions defined for the Move format. Other static access functions can set each operand (in this case, \spverb#Result# and \spverb#Val#), query each operand for nullness, clear operands, create Move instructions, mutate other instructions into Move instructions, and check the index of a particular operand field in the instruction. See the Javadoc\textsuperscript{TM} reference for a complete description of the API.

Each fixed-length instruction format is defined in the text file \spverb#$RVM_ROOT/rvm/src-generated/opt-ir/InstructionFormatList.dat#. Each record in this file has four lines:

\begin{itemize}
\item \spverb#NAME#: the name of the instruction format
\item \spverb#SIZES#: the number of operands defined, defined and used, and used
\item \spverb#SIZES#: the number of operands defined, defined and used, and used
      \begin{itemize}
        \item \spverb#D/DU/U#: Is this operand a def, use, or both?
        \item \spverb#NAME#: the unique name to identify the operand
        \item \spverb#TYPE#: the type of the operand (a subclass of Operand)
        \item \spverb#[opt]#: is this operand optional?
      \end{itemize}
\item \spverb#VARSIG#: a description of repeating operands, used for variable-length instructions.
\end{itemize}

So for example, the record that defines the Move instruction format is

\begin{lstlisting}
Move
1 0 1
"D Result RegisterOperand" "U Val Operand"
<blank line>
\end{lstlisting}

This specifies that the \spverb+Move+ format has two operands, one def and one use. The def is called \spverb+Result+ and must be of type \spverb+RegisterOperand+. The use is called \spverb+Val+ and must be of type \spverb+Operand+.

A few instruction formats have variable number of operands. The format for these records is given at the top of \spverb+InstructionFormatList.dat+. For example, the record for the variable-length \spverb+Call+ instruction format is:

\begin{lstlisting}
Call
1 0 3 1 U 4
"D Result RegisterOperand" \
"U Address Operand" "U Method MethodOperand" "U Guard Operand opt"
"Param Operand"
\end{lstlisting}

This record defines the \spverb+Call+ instruction format. The second line indicates that this format always has at least 4 operands (1 def and 3 uses), plus a variable number of uses of one other type. The trailing 4 on line 2 tells the template generator to generate special constructors for cases of having 1, 2, 3, or 4 of the extra operands. Finally, the record names the \spverb+Call+ instruction operands and constrains the types. The final line specifies the name and types of the variable-numbered operands. In this case, a \spverb+Call+ instruction has a variable number of (use) operands called \spverb+Param+. Client code can access the \spverb+ith+ parameter operand of a Call instruction \spverb+s+ by calling \spverb+Call.getParam(s,i)+.

A number of instruction formats share operands of the same semantic meaning and name. For convenience in accessing like instruction formats, the template generator supports four common operand access types:
\begin{itemize}
  \item \spverb+ResultCarrier+: provides access to an operand of type \spverb+RegisterOperand+ named \spverb+Result+.
  \item \spverb+GuardResultCarrier+: provides access to an operand of type \spverb+RegisterOperand+ named \spverb+GuardResult+.
  \item \spverb+LocationCarrier+: provides access to an operand of type \spverb+LocationOperand+ named \spverb+Location+.
  \item \spverb+GuardCarrier+: provides access to an operand of type \spverb+Operand+ named \spverb+Guard+.
\end{itemize}

For example, for any instruction \spverb+s+ that carries a \spverb+Result+ operand (eg. \spverb+Move+, \spverb+Binary+, and \spverb+Unary+ formats), client code can call \spverb+ResultCarrier.conforms(s)+ and \spverb+ResultCarrier.getResult(s)+ to access the \spverb+Result+ operand.

Finally, a note on rationale. Religious object-oriented philosophers will cringe at the \spverb+InstructionFormats+. Instead, all this functionality could be implemented more cleanly with a hierarchy of instruction types exploiting (multiple) inheritance. We rejected the class hierarchy approach due to efficiency concerns of frequent virtual/interface method dispatch and type checks. Recent improvements in our interface invocation sequence and dynamic type checking algorithms may alleviate some of this concern.

\end{subsubsection}

\end{subsection}
