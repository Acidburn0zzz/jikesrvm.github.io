\setNextFileName{OptTestHarness.html}
\begin{section}{OptTestHarness}
\label{sec:opttestharness}

For optimizing compiler development, it is sometimes useful to exercise careful control over which classes are compiled, and with which optimization level. In many cases, a prototype-opt image will suit this process using the command line option \texttt{-X:aos:initial\_compiler=opt} combined with \texttt{-X:aos:enable\_recompilation=false}. This configuration invokes the optimizing compiler on each method run.The \spverb#OptTestHarness# provides even more control over the optimizing compiler. This driver program allows you to invoke the optimizing compiler as an "application" running on top of the VM.

\begin{table}
\begin{tabular}{p{0.47\linewidth}p{0.47\linewidth}}
-useBootOptions & Use the same OptOptions as the bootimage compiler. \\
-longcommandline \textless filename\textgreater & Read commands (one per line) from a file \\
+baseline & Switch default compiler to baseline \\
-baseline & Switch default compiler to optimizing \\
-load \textless class\textgreater & Load a class \\
-class \textless class\textgreater & Load a class and compile all its methods \\
-method \textless class\textgreater \textless method\textgreater  [- or \textless descrip\textgreater] & Compile method with default compiler \\
-methodOpt \textless class\textgreater \textless method\textgreater  [- or \textless descrip\textgreater] & Compile method with opt compiler \\
-methodBase \textless class\textgreater \textless method\textgreater  [- or \textless descrip\textgreater] & Compile method with base compiler \\
-er \textless class\textgreater \textless method\textgreater  [- or \textless descrip\textgreater] \{args\} & Compile with default compiler and execute a method \\
-performance & Show performance results \\
-oc & pass an option to the optimizing compiler \\
\end{tabular}
\caption{OptTestHarness command line options}
\end{table}

\begin{subsection}{Examples}

To use the OptTestHarness program:

\begin{lstlisting}
rvm org.jikesrvm.tools.oth.OptTestHarness -class Foo
\end{lstlisting}

will invoke the optimizing compiler on all methods of class \spverb#Foo#.

\begin{lstlisting}
rvm org.jikesrvm.tools.oth.OptTestHarness -method Foo bar -
\end{lstlisting}

will invoke the optimizing compiler on the first method bar of class \spverb#Foo# it loads.

\begin{lstlisting}
rvm org.jikesrvm.tools.oth.OptTestHarness -method Foo bar '(I)V;'
\end{lstlisting}

will invoke the optimizing compiler on method \spverb#Foo.bar(I)V;#.
You can specify any number of -method and -class options on the command line. Any arguments passed to OptTestHarness via -oc will be passed on directly to the optimizing compiler. So:

\begin{lstlisting}
rvm org.jikesrvm.tools.oth.OptTestHarness -oc:O1 -oc:print_final_hir=true -method Foo bar -
\end{lstlisting}

will compile \spverb#Foo.bar# at optimization level O1 and print the final HIR.

\end{subsection}

\end{section}
