\ifdefined\HCode
  \newcommand\setNextFileName[1]{%
  \NextFile{#1}%
  }%
  \else
  \newcommand\setNextFileName[1]{}%
\fi

\documentclass[a4paper]{book}
% TODO Separate variants for letter paper and a4 paper to support printing without hassle?
\usepackage[iso,english]{isodate} % Use ISO date formatting to prevent confusion between American date formats (mm-dd-yy) and those used by most of the rest of the world (dd-mm-yy)
% TODO We could use \usepackage[iso]{datetime2} but datetime2 is quite new right now (May 2015) and possibly not available via distributions.
\usepackage[usenames,dvipsnames]{xcolor} % for colored star on the Magic page
\usepackage{amssymb} % for \bigstar on the Magic page
\usepackage{hyperref} % for links
\usepackage{listings} % for code listings and command lines
\usepackage{graphicx} % for images
\usepackage[htt]{hyphenat} % For words that contain underscores (e.g. command line options)
\usepackage{spverbatim} % for spverbatim environment which breaks lines and \spverb macro
\usepackage[utf8]{inputenc}
\title{Jikes RVM User Guide}
\author{Jikes RVM Contributors and Core Team}

\lstset{breaklines=true,breakatwhitespace=true, frame=single}

% Disable paragraph indenting
\setlength{\parindent}{0in}

\begin{document}
\maketitle

% TODO When done with everything, make another pass to make sure everything looks ok in HTML and PDF

% TODO turn "published papers" into link (old one was http://docs.codehaus.org/display/RVM/Publications )
The User Guide provides Jikes™ RVM information that is not typically covered in published papers. For high-level overviews, algorithms, and structures, you will find the published papers to be the best starting place. The User Guide supplements these Jikes RVM papers, focusing on implementation details of how to build, run, and add functionality to the system.

You may find sections of the User Guide missing, incomplete or otherwise confusing. We intend this document to live as a continual work-in-progress, hopefully growing and maturing as community members edit and add to the guide. Please accept this invitation to contribute.

% TODO update for github, link mailing lists
Please send feedback, bug fixes, and text contributions to the mailing list. Constructive criticism will be cheerfully accepted.

% TODO: add links
\begin{itemize}
  \item Care and Feeding: The guide to practical aspects of building, testing, debugging and evaluating Jikes RVM.
  \item Architecture: The guide to the major architectural decisions of Jikes RVM.
  \item MMTk Tutorial: A simple tutorial to building a collector with MMTk.
\end{itemize}

\chapter{Care and Feeding}

% TODO link for Quick start guide
This section describes the practical aspects of getting started using and modifying Jikes RVM. The Quick Start Guide gives a 10 second overview on how to get started while the following sections give more detailed instructions.


\setNextFileName{BuildingJikesRVM.html}
\begin{chapter}{Building Jikes RVM}
\label{cha:buildingjikesrvm}

This guide describes how to build Jikes RVM. The first section is an overview of the Jikes RVM build process and this is followed by your system requirements and a detailed description of the steps required to build Jikes RVM.

Once you have things working, as described below, the \hyperref[sec:usingbuildit]{buildit script} will provide a fast and easy way to build the system.  We recommend you get things working as described below first, so you can be sure you've met the requisite dependencies etc.

\begin{section}{Overview}

To avoid problems with the build, make sure that the path to the Jikes RVM source code doesn't contain any whitespace.

If you run into trouble when building Jikes RVM, don't hesitate to ask for help on the \href{http://www.jikesrvm.org/MailingLists/}{researchers mailing list}.

\begin{subsection}{Compiling the source code}

The majority of Jikes RVM is written in Java and will be compiled into class files just as with other Java applications. There is also a small portion of Jikes RVM that is written in C that must be compiled with a C compiler such as gcc.  Jikes RVM uses \href{https://ant.apache.org}{Ant} version 1.7.0 or later as the build tool that orchestrates the build process and executes the steps required in building Jikes RVM.

Jikes RVM requires a complete install of ant, including the optional tasks. These are present if you download and install ant manually. Some Linux distributions have decided to break ant into multiple packages. So if you are installing on a platform such as Debian you may need to install another package such as 'ant-optional'.
\end{subsection}

\begin{subsection}{Generating source code}

The build process also generates Java and C source code based on build time constants such as the selected instruction architecture, garbage collectors and compilers. The generation of the source code occurs prior to the compilation phase.

\end{subsection}

\begin{subsection}{Bootstrapping Jikes RVM}

Jikes RVM compiles Java class files and produces arrays of code and data. To build itself Jikes RVM will execute on an existing Java Virtual Machine and compiles a copy of it's own class files into a boot image for the code and data using the boot image writer tool. The set of files compiled is called the \hyperref[sec:primordialclasslist]{Primordial Class List}. The boot image runner is a small C program that loads the boot image and transfers control flow into Jikes RVM.

\end{subsection}

\begin{subsection}{Class libraries}

The Java class library is the mechanism by which Java programs communicate with the outside world. Jikes RVM has configurable class library support, the most mature of which is the the \href{http://www.gnu.org/software/classpath/}{GNU Classpath} class library.

For GNU Classpath, the developer can either specify a particular version of GNU Classpath to use. By default the build process will download and build GNU Classpath.

Previous releases of the Jikes RVM had support for the Apache Harmony class library. This is no longer developed or supported because Apache Harmony development \href{https://harmony.apache.org/}{was stopped}. Support for OpenJDK is planned, but not yet implemented.

\end{subsection}

\end{section}

\begin{section}{Target Requirements}

\begin{subsection}{Architectures}
The PowerPC (or ppc) and ia32 instruction set architectures are supported by Jikes RVM.

Intel's Instruction Set Architectures (ISAs) get known by different names:

\begin{itemize}
  \item IA-32 is the name used to describe processors such as 386, 486 and the Pentium processors. It is popularly called x86 or sometimes in our documentation as x86-32.
  \item IA-32e is the name used to describe the extension of the IA-32 architecture to support 8 more registers and a 64-bit address space. It is popularly called x86\textunderscore 64 or AMD64, as AMD chips were the first to support it. It is found in processors such AMD's Opteron and Athlon 64, as well as in Intel's own Pentium 4 processors that have EM64T in their name.
  \item IA-64 is the name of Intel's Itanium processor ISA.
\end{itemize}

Jikes RVM currently supports the IA-32 ISA and work on IA32-e is in progress. As IA-32e is backward compatible with IA-32, Jikes RVM can be built and run upon IA-32e processors. The IA-64 architecture supports IA-32 code through a compatibility mode or through emulation and Jikes RVM should run in this configuration. Native IA-64 is not supported.

On PowerPC, only big endian is supported.

\end{subsection}

\begin{subsection}{Operating Systems}
Jikes RVM is capable of running on any operating system that is supported by the GNU Classpath library, low level library support is implemented and memory layout is defined. The low level library support includes interaction with the threading and signal libraries, memory management facilities and dynamic library loading services. The memory layout must also be known, as Jikes RVM will attempt to locate the boot image code and data at specific memory locations. These memory locations must not conflict with where the native compiler places it's code and data. Operating systems that are known to work include Linux and OS X. At one stage a port to win32 was completed but it was never integrated into the main Jikes RVM codebase. AIX was supported previously but support has been removed due to lack of demand. The same applies for support of Mac OS on PPC.

Note: Current implementation of Jikes RVM implies that system native libraries (like GTK+) have been compiled with frame pointers. Most of Linux distribution have frame pointers enabled in most of the packages, but some explicitly use \spverb+-fomit-frame-pointer+ thus producing the library that can't be used with Jikes RVM.
\end{subsection}

\begin{subsection}{Support Matrix}
The platform support matrix table details the targets that have historically been supported and the current status of the support. The target.name column is the identifier that Jikes RVM uses to identify this 
target. ??? means that we don't have regression machines for this platform so the Jikes RVM team can't guarantee that the target works at a given point in time. We rely on the community to provide a Jikes RVM implementation on these platforms.

\begin{table}
\centering
\begin{tabular}{lcccc}
target.name & OS & ISA & Address size & Status \\
ia32-linux & Linux & IA32 & 32 bits & OK \\
ia32-osx & OS X & IA32 & 32 bits & ??? \\
ia32-solaris & Solaris & IA32 & 32 bits & ??? \\
ia32-cygwin & Windows & IA32 & 32 bits & NYI \\
x86\textunderscore 64-linux & Linux & IA32 & \textbf{32 bits} & OK \\
x86\textunderscore 64-osx & OS X & IA32 & \textbf{32 bits} & ??? \\
x86\textunderscore 64\textunderscore m64-linux & Linux & IA32e & \textbf{64 bits} & \href{https://xtenlang.atlassian.net/browse/RVM-977}{WIP} \\
x86\textunderscore 64\textunderscore m64-osx & OS X & IA32e & \textbf{64 bits} & ??? \\
ppc32-linux & Linux & ppc32 (big e.) & 32 bits & ??? \\
ppc64-linux & Linux & ppc64 (big e.) & 64 bits & OK \\
\end{tabular}
\caption{platform support matrix}
\end{table}

x86\textunderscore 64 is currently only supported using the legacy 32bit addressing mode and instructions. You need to install the 32-bit versions of the required libraries to build and use the x86\textunderscore 64 configurations.

Note that building on Windows is currently not supported. All previous attempts at building on Windows natively (i.e. without cygwin) used the Apache Harmony classlibrary whose development has been discontinued. Support for building with cygwin is not yet implemented.

\end{subsection}

\end{section}

\begin{section}{Tool Requirements}

\begin{paragraph}{Java Virtual Machine}

Jikes RVM requires an existing Java Virtual Machine that conforms to Java 6.0 such as Oracle JDK 1.6, OpenJDK/IcedTea 6 or IBM SDK 6.0. We also aim to support the Java 7.0-conformant ans Java 8.8-conformant versions of these virtual machines.

Some Java Virtual Machines are unable to cope with compiling the Java class library so it is recommended that you install one of the above mentioned JVMs if they are not already installed on your system. The remaining build instructions assume that a suitable Java Virtual Machine is on your path. You can run \spverb+java -version+ to check you are using the correct JVM.

\end{paragraph}

\begin{paragraph}{Ant}

Ant version 1.7.0 or later is the tool required to orchestrate the build process. You can download and install the Ant tool from \href{http://ant.apache.org/}{its Apache homepage} if it is not already installed on your system. The remaining build instructions assume that \spverb+$ANT_HOME/bin+ is on your path and points to a full Ant installation (i.e. including the optional tasks). You can run \spverb+ant -version+ to check you are running the correct version of ant.

\end{paragraph}

\begin{paragraph}{C compilers}

The Jikes RVM build assumes that the GNU Compiler Collection is present on the system. Most modern *nix environments satisfy this requirement. Clang should also work but is untested.

\end{paragraph}

\begin{paragraph}{Bison}

As part of the build process, Jikes RVM uses the bison tool which should be present on most modern *nix environments.

\end{paragraph}

\begin{paragraph}{Perl}

Perl is trivially used as part of the build process but this requirement may be removed in future releases of Jikes RVM. Perl is also used as part of the regression and performance testing framework.

\end{paragraph}

\begin{paragraph}{Awk}

GNU Awk is required as part of the regression and performance testing framework but is not required when building Jikes RVM.

\end{paragraph}

\end{section}

\begin{paragraph}{Extra tools recommended for Solaris}

pkg-get will greatly simplify installing GNU packages on Solaris. Our patches require that GNU patch is picked up in preference to Sun's. You can create a symbolic link to \spverb+/usr/bin/gpatch+ from \spverb+/opt/csw/bin/patch+ and make sure \spverb+/opt/csw/bin+ is in your path before \spverb+/usr/bin+ in order to achieve this.

\end{paragraph}

\begin{section}{Instructions}

\begin{subsection}{Defining Ant properties}

There are a number of ant properties that are used to control the build process of Jikes RVM. These properties may either be specified on the command line by \spverb+-Dproperty=variable+ or they may be specified in a file named \spverb+.ant.properties+ in the base directory of the jikesrvm source tree. The \spverb+.ant.properties+ file is a standard Java property file with each line containing a \spverb+property=variable+ and comments starting with a \spverb+#+ and finishing at the end of the line.

The available properties can be grouped into properties that resolve to values and properties that resolve to directories. For properties that resolve to directories, you must make sure that the value of the property resolves to an absolute path. Relative paths aren't supported by our build system. The path must not contain any whitespace.

\begin{table}
\centering
\begin{tabular}{p{0.15\linewidth}p{0.6\linewidth}p{0.15\linewidth}}
Property & Description & Default \\
host.name & The name of the host environment used for building Jikes RVM. The host environment defines the paths to the tools used during the build, e.g. the path to the C compiler. The name should match one of the files located in the \spverb+build/hosts/+ directory minus the '.properties' extension. & None \\
target.name & The name of the target environment for Jikes RVM. The name should match one of the files located in the \spverb+build/targets/+ directory minus the '.properties' extension. This should only be specified when cross compiling the Jikes RVM. See \hyperref[sec:crossplatformbuilding]{Cross-Platform Building} for a detailed description of cross compilation. & \$\{host.name\} \\
config.name & The name of the configuration used when building Jikes RVM. The name should match one of the files located in the \spverb+build/configs/+ directory minus the '.properties' extension. This setting is further described in the section \hyperref[cha:configuringjikesrvm]{Configuring Jikes RVM}. & None \\
patch.name & An identifier for the current patch applied to the source tree. See \hyperref[sec:buildingpatchedversions]{Building Patched Versions} for a description of how this fits into the standard usage patterns of Jikes RVM. & ``\,'' \\
require.rvm-unit-tests & If set to \spverb+true+, run \hyperref[cha:testingjikesrvm]{unit tests} on the built Jikes RVM image. Use with care as it will significantly increase build times for configurations that are compiled using a non-optimizing compiler (see below). & (Undefined, tests are not run) \\
require.\newline checkstyle & Only useful if you want to \href{http://www.jikesrvm.org/Contributions/}{contribute} changes to the Jikes RVM. If set to true, run checkstyle during the build to check for violations of the Jikes RVM \hyperref[sec:codingstyle]{Coding Style} and \hyperref[sec:codingconventions]{Coding Conventions} for assertions. & (Undefined, no checks run) \\
rvm.debug-symbols & If set to true, build the Jikes RVM with debug symbols for the bootloader code and the code in the bootimage. Note: this is not enabled by default because it causes build failures for configurations that build the bootimage with the optimizing compiler (see \href{https://xtenlang.atlassian.net/browse/RVM-1084}{RVM-1084}). & (Undefined, no symbols built) \\
protect.config-files & Define this property if you do not want the build process to update configuration files when auto downloading components. & (Undefined) \\
\end{tabular}
\caption{Ant value properties for Jikes RVM}
\end{table}

\begin{table}
\centering
\begin{tabular}{p{0.15\linewidth}p{0.6\linewidth}p{0.15\linewidth}}
Property & Description & Default \\
com\-po\-nents.dir & The directory where Ant looks for external components when building Jikes RVM. & \$\{jikesrvm.\newline dir\}/com\-po\-nents \\
dist.dir & The directory where Ant stores the final Jikes RVM runtime. & \$\{jikesrvm.\newline dir\}/dist \\
build.dir & The directory where Ant stores the intermediate artifacts generated when building the Jikes RVM. & \$\{jikesrvm.\newline dir\}/tar\-get \\
com\-po\-nents.\-cache.\-dir & The directory where Ant caches downloaded components.  If you explicitly download a component, place it in this directory. & (Undefined, forcing download) \\
\end{tabular}
\caption{Ant directory properties for Jikes RVM}
\end{table}


At a minimum it is recommended that the user specify the \spverb+host.name+ property in the \spverb+.ant.properties+ file.

The configuration files in \spverb+build/targets/+ and \spverb+build/hosts/+ are designed to work with a typical install but it may be necessary to overide specific properties. The easiest way to achieve this is to specify the properties to override in the \spverb+.ant.properties+ file.

\end{subsection}

\begin{subsection}{Selecting a Configuration}

A configuration in terms of Jikes RVM is the combination of build time parameters and component selection used for a particular Jikes RVM image. The section \hyperref[cha:configuringjikesrvm]{Configuring Jikes RVM} section describes the details of how to define a configuration. Typical configuration names include:
\begin{itemize}
  \item \textbf{production}: This configuration defines a fully optimized version of the Jikes RVM.
  \item \textbf{development}: This configuration is the same as production but with debug options enabled. The debug options perform internal verification of Jikes RVM which means that it builds and executes more slowly.
  \item \textbf{prototype}: This configuration is compiled using an unoptimized compiler and includes minimal components which means it has the fastest build time.
  \item \textbf{prototype-opt}: This configuration is compiled using an unoptimized compiler but it includes the adaptive system and optimizing compiler. This configuration has a reasonably fast build time.
\end{itemize}

If a user is working on a particular configuration most of the time they may specify the config.name ant property in \spverb+.ant.properties+ otherwise it should be passed in on the command line \spverb+-Dconfig.name=...+.

\end{subsection}

\begin{subsection}{Fetching Dependencies}

The Jikes RVM has a build time dependency on the GNU Classpath class library and depending on the configuration may have a dependency on \href{http://www.cs.kent.ac.uk/projects/gc/gcspy/}{GCSpy}. The build system will attempt to download and build these dependencies if they are not present or are the wrong version.

To just download and install the GNU Classpath class library you can run the command "ant -f build/components/classpath.xml". After this command has completed running it should have downloaded and built the GNU Classpath class library for the current host. See the \hyperref[sec:usinggcspy]{Using GCSpy} page for directions on building configurations with GCSpy support.

If you wish to manually download components (for example you need to define a proxy, so ant is not correctly downloading), you can do so and identify the directory containing the downloads using \spverb+-Dcomponents.cache.dir=<download directory>+ when you build with ant.

\end{subsection}

\begin{subsection}{Building Jikes RVM}

The next step in building Jikes RVM is to run the ant command \spverb+ant+ or \spverb+ant -Dconfig.name=...+. This should build a complete RVM runtime in the directory \spverb+${dist.dir}/${config.name}_${target.name}+. A complete list of documented targets can be listed by executing the command \spverb+ant -projecthelp+.

\end{subsection}

\begin{subsection}{Running Jikes RVM}

Jikes RVM can be executed in a similar way to most Java Virtual Machines. The difference is that the command is \spverb+rvm+ and resides in the runtime directory (i.e. \spverb+${dist.dir}/${config.name}_${target.name}+). See \hyperref[cha:runningjikesrvm]{Running Jikes RVM} for a list of command line options.

\end{subsection}

\end{section}

\begin{section}{Building Patched Versions}
\label{sec:buildingpatchedversions}

As part of the research process there will be a need to evaluate a set of changes to the source tree. To make this process easier the property named patch.name can be set to a non-empty string. This will cause the output directory to have the name \spverb+${config.name}_${target.name}_${config.variant}+ rather than \spverb+${config.name}_${target.name}+, thus making it easy to differentiate between the patched and unpatched runtimes.

The following steps will create a runtime without the patch in \texttt{dist/prototype\_ia32-linux} and a runtime with the patch applied in \texttt{dist/prototype\_ia32-linux\_ReadBarriers}:

\begin{lstlisting}
% cd $RVM_ROOT
% ant -Dconfig.name=prototype -Dhost.name=ia32-linux
% patch -p0 < ReadBarriers.diff
% ant -Dconfig.variant=ReadBarriers -Dconfig.name=prototype -Dhost.name=ia32-linux
% patch -R -p0 < ReadBarriers.diff
\end{lstlisting}

The \spverb+config.variant+ property is also supported and reported as part of the test infrastructure.

\end{section}


\setNextFileName{CrossPlatformBuilding.html}
\begin{section}{Cross-Platform Building}
\label{sec:crossplatformbuilding}

The Jikes™ RVM build process consists of two major phases: the building of a \textit{boot image}, and the building of a \textit{bootloader}. The boot image is built using a Java™ program executed within a host JVM and is therefore platform-neutral. By contrast, the boot loader is written in C, and must be compiled on the target platform.

Because building the boot image can be time-consuming, you may prefer to build the boot image on a faster machine than the target platform. You may also be porting Jikes RVM to a target platform that lacks tools such or whose development environment is otherwise unpleasant. To cross-build, simply set your host.name and target.name properties to different values.

For example, to build the prototype configuration for AIX™ on a Linux host:
\begin{lstlisting}
% cd $RVM_ROOT
% ant -Dconfig.name=prototype -Dhost.name=ia32-linux -Dtarget.name=ppc32-aix cross-compile-host
\end{lstlisting}

The build process is then completed by building just the boot loader on an AIX host:
\begin{lstlisting}
% cd $RVM_ROOT
% ant -Dconfig.name=prototype -Dhost.name=ppc32-aix cross-compile-target
\end{lstlisting}

After the script has completed successfully, you should be able to run Jikes RVM.

The building of the boot loader must occur in the same directory as the rest of the build. This can either be done transparently via a network file system, or by copying the build directory from the first host to the target. 

\begin{subsection}{Dependencies}

To compile the boot image on the host system you will also need to have built any dependencies on the target machine and then copied them to the host machine. You will also need to add an appropriate line into your \newline \spverb+${components.dir}components.properties+ file such as the following (if the target system was pppc32-linux):

\begin{lstlisting}[breaklines=true,breakatwhitespace=false]
ppc32-linux.classpath.lib.dir=path/to/components/classpath/95/ppc32-linux/lib
\end{lstlisting}

It may be possible to simply build the dependencies on the host machine. Modify the \spverb+${components.dir}/components.properties+ so that the dependency property for target machine maps to the same value as the dependency property on the host machine. This works at the current time but may fail in the future if classpath changes the API between platforms. i.e.

\begin{lstlisting}[breaklines=true,breakatwhitespace=false]
ppc32-linux.classpath.lib.dir=path/to/components/classpath/95/ia32-linux/lib
\end{lstlisting}


\end{subsection}

\end{section}


\setNextFileName{PrimordialClassList.html}
\begin{section}{Primordial Class List}
\label{sec:primordialclasslist}

The primordial class list indicates which classes should be compiled and baked into the boot image. The bare minimum set of classes needed in the primordial list includes:

\begin{itemize}
  \item All classes that are needed to load a class from the file system. The class may need to be loaded as a single class file or out of a jar. Failing this there will be an infinite regress on the first class load.
  \item All classes that are needed by the baseline compiler to compile any method. Failing this we regress when attempting to compile a method so we can execute it.
  \item Enough of the core VM services and data structures, and class library (java.*) to support the above. This includes threading, memory management, runtime support for compiled code, etc.
\end{itemize}

For increased performance and decreased startup time it is possible to include extra classes that are expected to be needed, i.e. the optimizing compiler or the adaptive system. There are some pieces of these components that would be awkward to load dynamically (there's a core subset of the opt compiler, the classes in the \verb+org.jikesrvm.compilers.opt.runtimesupport+ packages, that must be loaded and fully compiled before any opt-compiled code can be allowed to executed), but it's theoretically possible to do so.

If you took a full closure of the classes referenced by things that have to be in the bootimage you'd actually end up with a lot more in the bootimage than we currently have. The culprit here would I think mainly be java.* classes that we need in the bootimage, but only use in restricted ways, so we don't actually have to drag in everything they depend on to meet the "real" constraints of what has to go in the bootimage. It is unknown how much difference there is between hand-crafted include lists and what an automated tool would discover.

\end{section}


\setNextFileName{UsingBuildit.html}
\begin{section}{Using buildit}
\label{sec:usingbuildit}

The buildit script is a handy way to build and test the system.  It has countless features and options to make building and testing really easy, particularly in a multi-machine environment, where you edit on one machine and build and test on others.  If you really want to get the most of it, take a look at all the options by running:

\begin{lstlisting}
bin/buildit -h
\end{lstlisting}

...or read the script itself.

% It is customary to have at least 2 subsections or none at all. However, examples are generally popular, so we'll make an exception here.
\begin{subsection}{Examples}

Here we just provide a hand full of examples of how it is often used, first for building and secondly for testing (which includes building). Please add to the list if you have other really useful ways of using it.  In the examples below, we'll use three hypothetical hosts: \textbf{habanero} (your desktop), \textbf{jalapeno} (a remote x86 machine) and \textbf{chipotle} (a remote PowerPC machine).

\begin{subsubsection}{Simple Builds}

To build a production image on your desktop, habanero, do the following: 

\begin{lstlisting}
bin/buildit habanero production
\end{lstlisting}

Or equivalently:

\begin{lstlisting}
bin/buildit localhost production
\end{lstlisting}

To build a production image on the remote machine jalapeno, do the following: 

\begin{lstlisting}
bin/buildit jalapeno production
\end{lstlisting}

\end{subsubsection}

\begin{subsubsection}{Cross Platform Building}

To build a production image on the remote PowerPC machine chipotle, do the following: 

\begin{lstlisting}
bin/buildit chipotle production
\end{lstlisting}

Since building on a PowerPC machine can take a long time, you might prefer to build on your x86 machine jalapeno and cross-build to chipotle.  In that case you would just do the following: 

\begin{lstlisting}
bin/buildit jalapeno -c chipotle production
\end{lstlisting}

In each case, buildit figures out the host types by interrogating them and does the right thing (forcing a PPC build on the x86 host jalapeno since you've told it you want a build for chipotle, which it knows is PPC).  Buildit caches the host information, and will prompt you the first time it encounters a new host. 

\end{subsubsection}

\begin{subsubsection}{Full Build Specification}

If you want to specify the build fully, you can do something like this:

\begin{lstlisting}
bin/buildit jalapeno FastAdaptive MarkSweep
\end{lstlisting}

If you want to specify multiple different GCs you could do:

\begin{lstlisting}
bin/buildit jalapeno FastAdaptive MarkSweep SemiSpace GenMS
\end{lstlisting}

which would build all three configurations on jalapeno.
\end{subsubsection}

\begin{subsubsection}{Profiled Builds}

Jikes RVM has the capacity to profile the boot image and then re-build an optimized boot image based on the profiles.  This process takes a little longer, but results in measurably faster builds, and so should be used when doing performance testing.  Buildit lets you do this trivially:

\begin{lstlisting}
bin/buildit jalapeno --profile production
\end{lstlisting}

\end{subsubsection}

\begin{subsubsection}{Testing}

Jikes RVM currently has a notion of a \textbf{"test-run"}, which defines a complete test scenario, including tests and builds.  An example is \textit{pre-commit}, which runs a small suite of pre-commit tests.  It also has the notion of a \textbf{"test"}, which just specifies a particular set of tests, not the full scenario.  An example is \textit{dacapo}, which just runs the DaCapo test suite (see the testing/tests directory for the available tests).

\end{subsubsection}

\begin{subsubsection}{Running a test run}
To run the pre-commit test-run on your host jalapeno just do:

\begin{lstlisting}
bin/buildit jalapeno --test-run pre-commit jalapeno
\end{lstlisting}

\end{subsubsection}

\begin{subsubsection}{Running a test}
To run the dacapo tests against a production on the host jalapeno, do:

\begin{lstlisting}
bin/buildit jalapeno -t dacapo production
\end{lstlisting}

To run the dacapo tests against a FastAdaptive MarkSweep build, on the host jalapeno, do:

\begin{lstlisting}
bin/buildit jalapeno -t dacapo FastAdaptive MarkSweep
\end{lstlisting}

To run the dacapo and SPECjvm98 tests against production on the host jalapeno, do:

\begin{lstlisting}
bin/buildit jalapeno -t dacapo -t SPECjvm98 production
\end{lstlisting}

\end{subsubsection}

\end{subsection}

\end{section}


\end{chapter}


\setNextFileName{PrimordialClassList.html}
\begin{section}{Primordial Class List}
\label{sec:primordialclasslist}

The primordial class list indicates which classes should be compiled and baked into the boot image. The bare minimum set of classes needed in the primordial list includes:

\begin{itemize}
  \item All classes that are needed to load a class from the file system. The class may need to be loaded as a single class file or out of a jar. Failing this there will be an infinite regress on the first class load.
  \item All classes that are needed by the baseline compiler to compile any method. Failing this we regress when attempting to compile a method so we can execute it.
  \item Enough of the core VM services and data structures, and class library (java.*) to support the above. This includes threading, memory management, runtime support for compiled code, etc.
\end{itemize}

For increased performance and decreased startup time it is possible to include extra classes that are expected to be needed, i.e. the optimizing compiler or the adaptive system. There are some pieces of these components that would be awkward to load dynamically (there's a core subset of the opt compiler, the classes in the \verb+org.jikesrvm.compilers.opt.runtimesupport+ packages, that must be loaded and fully compiled before any opt-compiled code can be allowed to executed), but it's theoretically possible to do so.

If you took a full closure of the classes referenced by things that have to be in the bootimage you'd actually end up with a lot more in the bootimage than we currently have. The culprit here would I think mainly be java.* classes that we need in the bootimage, but only use in restricted ways, so we don't actually have to drag in everything they depend on to meet the "real" constraints of what has to go in the bootimage. It is unknown how much difference there is between hand-crafted include lists and what an automated tool would discover.

\end{section}


\chapter{Architecture}

This section describes the architecture of Jikes RVM. The RVM can be divided into the following components:

% TODO links
\begin{itemize}
  \item Core Runtime Services: (thread scheduler, class loader, library support, verifier, etc.) This element is responsible for managing all the underlying data structures required to execute applications and interfacing with libraries.
  \item Magic: The mechanisms used by Jikes RVM to support low-level systems programming in Java.
  \item Compilers: (baseline, optimizing, JNI) This component is responsible for generating executable code from bytecodes.
  \item Memory managers: This component is responsible for the allocation and collection of objects during the execution of an application.
  \item Adaptive Optimization System: This component is responsible for profiling an executing application and judiciously using the optimizing compiler to improve its performance.
\end{itemize}

\NextFile{Magic.html}
\begin{section}{Magic}

Most Java runtimes rely upon the foreign language APIs of the underlying platform operating system to implement runtime behaviour which involves interaction with the underlying platform. Runtimes also occasionally employ small segments of machine code to provide access to platform hardware state. Note that this is expedient rather than mandatory. With a suitably smart Java bytecode compiler it would be quite possible to implement a full Java-in-Java runtime i.e. one comprising only compiled Java code (the JNode project is an attempt to implement a runtime along these lines; the Xerox, MIT, Lambda and TI Explorer Lisp machine implementations and the Xerox Smalltalk implementation were highly successful attemtps at fully compiled language runtimes).

This section provides information on \textcolor{red}{$\bigstar$} magic \textcolor{red}{$\bigstar$} which is an escape hatch that Jikes™ RVM provides to implement functionality that is not possible using the pure Java™ programming language. For example, the Jikes RVM garbage collectors and runtime system must, on occasion, access memory or perform unsafe casts. The compiler will also translate a call to Magic.threadSwitch() into a sequence of machine code that swaps out old thread registers and swaps in new ones, switching execution to the new thread's stack resumed at its saved PC

There are three mechanisms via which the Jikes RVM \textcolor{red}{$\bigstar$} magic \textcolor{red}{$\bigstar$} is implemented:
\begin{itemize}
  \item Compiler Intrinsics: Most methods are within class librarys but some functions are built in (that is, intrinsic) to the compiler. These are referred to as intrinsic functions or intrinsics.
  \item Compiler Pragmas: Some intrinsics are do not provide any behaviour but instead provide information to the compiler that modifies optimizations, calling conventions and activation frame layout. We rever to these mechanisms as compiler pragmas.
  \item Unboxed Types: Besides the primitive types, all Java values are boxed types. Conceptually, they are represented by a pointer to a heap object. However, an unboxed type is represented by the value itself. All methods on an unboxed type must be Compiler Intrinsics.
\end{itemize}

The mechanisms are used to implement the following functionality:
\begin{itemize}
  \item \hyperref[sec:rawmemoryaccess]{RawMemoryAccess}: Unfetted access to memory.
  \item Uninterruptible Code: Declaring code to be uninterruptible.
  \item Alternative Calling Conventions: Declaring different calling conventions and activation frame layout.
\end{itemize}

\end{section}

\setNextFileName{RawMemoryAccess.html}
\begin{section}{Raw Memory Access}
\label{sec:rawmemoryaccess}

The type \verb+org.vmmagic.Address+ is used to represent a machine-dependent address type. \verb+org.vmmagic.Address+ is an unboxed type. In the past, the base type \verb+int+ was used to represent addresses but this approach had several shortcomings. First, the lack of abstraction makes porting nightmarish. Equally important is that Java type \verb+int+ is signed whereas addresses are more appropriately considered unsigned. The difference is problematic since an unsigned comparison on \verb+int+ is inexpressible in the Java programming language.

To overcome these problems, instances of \verb+org.vmmagic.Address+ are used to represent addresses. The class supports the expected well-typed methods like adding an integer offset to an address to obtain another address, computing the difference of two addresses, and comparing addresses. Other operations that make sense on \verb+int+ but not on addresses are excluded like multiplication of addresses. Two methods deserve special attention: converting an address into an integer and the inverse. These methods should be avoided where possible.

Without special intervention, using a Java object to represent an address would be at best abysmally inefficient. Instead, when the Jikes RVM compiler encounters creation of an address object, it will return the primitive value that represents an address for that platform. Currently, the address type maps to either a 32-bit or 64-bit unsigned integer. Since an address is an unboxed type it must obey the rules outlined in Unboxed Types.

\end{section}

\setNextFileName{OptTestHarness.html}
\begin{section}{OptTestHarness}
\label{sec:opttestharness}

For optimizing compiler development, it is sometimes useful to exercise careful control over which classes are compiled, and with which optimization level. In many cases, a prototype-opt image will suit this process using the command line option \texttt{-X:aos:initial\_compiler=opt} combined with \texttt{-X:aos:enable\_recompilation=false}. This configuration invokes the optimizing compiler on each method run.The \spverb#OptTestHarness# provides even more control over the optimizing compiler. This driver program allows you to invoke the optimizing compiler as an "application" running on top of the VM.

\begin{table}
\begin{tabular}{p{0.47\linewidth}p{0.47\linewidth}}
-useBootOptions & Use the same OptOptions as the bootimage compiler. \\
-longcommandline \textless filename\textgreater & Read commands (one per line) from a file \\
+baseline & Switch default compiler to baseline \\
-baseline & Switch default compiler to optimizing \\
-load \textless class\textgreater & Load a class \\
-class \textless class\textgreater & Load a class and compile all its methods \\
-method \textless class\textgreater \textless method\textgreater  [- or \textless descrip\textgreater] & Compile method with default compiler \\
-methodOpt \textless class\textgreater \textless method\textgreater  [- or \textless descrip\textgreater] & Compile method with opt compiler \\
-methodBase \textless class\textgreater \textless method\textgreater  [- or \textless descrip\textgreater] & Compile method with base compiler \\
-er \textless class\textgreater \textless method\textgreater  [- or \textless descrip\textgreater] \{args\} & Compile with default compiler and execute a method \\
-performance & Show performance results \\
-oc & pass an option to the optimizing compiler \\
\end{tabular}
\caption{OptTestHarness command line options}
\end{table}

\begin{subsection}{Examples}

To use the OptTestHarness program:

\begin{lstlisting}
rvm org.jikesrvm.tools.oth.OptTestHarness -class Foo
\end{lstlisting}

will invoke the optimizing compiler on all methods of class \spverb#Foo#.

\begin{lstlisting}
rvm org.jikesrvm.tools.oth.OptTestHarness -method Foo bar -
\end{lstlisting}

will invoke the optimizing compiler on the first method bar of class \spverb#Foo# it loads.

\begin{lstlisting}
rvm org.jikesrvm.tools.oth.OptTestHarness -method Foo bar '(I)V;'
\end{lstlisting}

will invoke the optimizing compiler on method \spverb#Foo.bar(I)V;#.
You can specify any number of -method and -class options on the command line. Any arguments passed to OptTestHarness via -oc will be passed on directly to the optimizing compiler. So:

\begin{lstlisting}
rvm org.jikesrvm.tools.oth.OptTestHarness -oc:O1 -oc:print_final_hir=true -method Foo bar -
\end{lstlisting}

will compile \spverb#Foo.bar# at optimization level O1 and print the final HIR.

\end{subsection}

\end{section}


\NextFile{CostBenefitModel.html}
\begin{section}{Cost Benefit Model}
The Jikes RVM Adaptive Optimization System attempts to evaluate the break-even point for each action using an online competitive algorithm.  It relies on an analytic model to estimate the costs and benefits of each selective recompilation action, and evaluates the best actions according to the model predictions online.

A key advantage of this approach is that it allows a designer to extend the simple "break-even" cost-benefit model to account for more sophisticated adaptive policies, such as selective compilation with multiple optimization levels, on-stack-replacement, and long-running analyses.

In general, each potential action will incur some cost and may confer some benefit. For example, recompiling a method will certainly consume some CPU cycles, but could speed up the program execution by generating better code. In this discussion we focus on costs and benefits defined in terms of time (CPU cycles). However, in general, the controller could consider other measures of cost and benefit, such as memory footprint, garbage allocated, or locality disrupted.

The controller will take some action when it estimates the benefit to exceed the cost. More precisely, when the controller wakes at time $t$, it considers a set of $n$ available actions, the set $A = \{A_1, A_2, ..., A_n\}$. For any subset $S$ in $P(A)$, the controller can estimate the cost $C(S)$ and benefit $B(S)$ of performing all actions $A_i$ in $S$. The controller will attempt to choose the subset $S$ that maximizes $B(S) - C(S)$. Obviously $S = \{\}$ has $B(S) = C(S) = 0$; the controller takes no action if it cannot find a profitable course.

In practice, the precise cost and benefit of each action cannot be known; so, the controller must rely on estimates to make decisions.

The basic model the controller uses to decide which method to recompile, at which optimization level, and at what time is as follows.

Suppose that when the controller wakes at time $t$, and each method $m$ is currently optimized at optimization level $m_i, 0 \leq i \leq k$. Let $M$ be the set of loaded methods in the program. Let $A_{jm}$ be the action "recompile method m at optimization level $j$, or do nothing if $j = i$."

The controller must choose an action for each $m$ in $M$. The set of available actions is $Actions = \{A_{jm} | 0 \leq j \leq k, m \in M\}$.

Each action has an estimated cost and benefit: $C(A_{jm})$, the cost of taking action $A_{jm}$, for $0 \leq j \leq k$ and $T(A_{jm})$, the expected time the program will spend executing method $m$ in the future, if the controller takes action $A_{jm}$.

For $S$ in $Actions$, define $C(S) = \sum_{s \in S} C(s)$. Given $S$, for each $m$ in $M$, define $A_{min_m}$ to be the action $A_{jm}$ in $S$ that minimizes $T(A_{jm})$.  Then define $T(S) = \sum_{m \in M} T(A_{min_m})$.

Using these estimated values, the controller chooses the set $S$ that minimizes $C(S) + T(S)$. Intuitively, for each method $m$, the controller chooses the recompilation level $j$ that minimizes the expected future compilation time and running time of $m$.

It remains to define the functions $C$ and $T$ for each recompilation action. The basic model models the cost $C$ of compiling a method $m$ at level $j$ as a linear function of the size of $m$. The linear function is determined by an offline experiment to fit constants to the model.

The basic model estimates that the speedup for any optimization level $j$ is constant. The implementation determines the constant speedup factor for each optimization level offline, and uses the speedup to compute $T$ for each method and optimization level.

We assume that if the program has run for time $t$, then the program will run for another $t$ units, and then terminate. We further assume program behavior in the future will resemble program behavior in the past. Therefore, for each method we estimate that if no optimization action is performed $T(A_{jm})$ is equal to the time spent executing method $m$ so far.

Let $M=(m_1, ..., m_k)$ be the $k$ compiled methods. When the controller wakes at time $t$, each compiled method $m$ has been sampled $\sum m$ times. Let $\delta$ be the sampling interval, measured in seconds. The controller estimates that method $m$ has executed $\delta \sum m$ seconds so far, and will execute for another $\delta \sum m$ seconds in the future.

When driving recompilation based on sampling, the controller can limit its attention to the set of methods that were sampled in the previous sampling interval. This optimization does not lose precision; if the number of samples associated with a method has not changed, then the controller's estimate of the method's future execution time will not change. This implies that if the controller were to consider a
method that does not appear in the previous sampling interval, the controller would make exactly the same decision it did the last time it considered the method. This optimization, limiting the number of methods the controller must examine in each sampling interval, greatly reduces the amount of work performed by the controller.

Suppose the controller recompiles method m from optimization level $i$ to optimization level $j$ after having seen $\sum m$ samples. Let $S_i$ and $S_j $be the speedup ratios for optimization levels $i$ and $j$, respectively. After optimizing at level $j$, we adjust the sample data to represent the system state as if it had executed method $m$ at optimization level $j$ since program startup. So, we set the new number of samples for $m$ to be $\sum m \cdot (S_i/S_j)$. Thus to compute the time spent in $m$, we need know only one number, the "effective" number of samples.
\end{section}


// TODO convert to latex with proper inclusion of eps image which we didn't have before
Life Cycle of a Compiled Method
===============================
:author: David Grove
:date: 07-07-2008

In early implementations of Jikes RVM's adaptive system, compilation required holding a global lock that serialized compilation and also prevented classloading from occurring concurrently with compilation.  This bottleneck was removed in version 2.1.0 by switching to a finer-grained locking discipline to coordinate compilation, speculative optimization, and class loading. Since no published description of this locking protocol exists outside of the source code, we briefly summarize the life cycle of a compiled method here.

When Jikes RVM compiles a method, it creates a compiled method object to represent this particular compilation of the source method.  A compiled method has a unique id, and stores the compiled code and associated compiler meta-data. After a brief initialization phase, the compiled method transitions from uncompiled to compiling when compilation begins. During compilation, the optimizing compiler may perform speculative optimizations that can be invalidated by future class loading.  Each time the compiler so speculates, it records a relevant entry in an invalidation database.  Upon finishing compilation, the system checks to ensure that the current compilation has not already been  invalidated by concurrent classloading.  If it has not, then the system installs the compiled code, and subsequent  invocations will branch to the newly created code.

Each time a class is loaded, the system checks the invalidation database to identify the set of compiled methods to mark as obsolete,
because this classloading action invalidates speculative optimizations previously applied to that method.  A method may transition from either compiling or installed to obsolete due to a classloading-induced invalidation.  A method can also transition from installed to obsolete when the adaptive system selects a method for optimizing recompilation and a new compiled method is installed to replace it.

image:images/93224965.eps[life cycle of a compiled method]

Once a method is marked obsolete, it will never be invoked again.  However, before the generated code for the compiled method can be garbage collected, all existing invocations of the compiled method must be complete.  A compiled method transitions from obsolete to  dead when no invocations of it exist on any thread stack.  Jikes RVM detects this as part of the stack scanning phase of garbage collection; as stack frames are scanned, their compiled methods are marked as active.  Any obsolete method that is not marked as active when stack scanning completes is marked as dead and the reference to it is removed from the compiled method table.  It will then be freed during the next garbage collection


\setNextFileName{IR.html}
\begin{section}{IR}
\label{sec:ir}

The optimizing compiler intermediate representation (IR) is held in an object of type \spverb#IR# and includes a list of instructions. Every instruction is classified into one of the pre-defined instruction formats. Each instruction includes an operator and zero or more operands. Instructions are grouped into basic blocks; basic blocks are constrained to having control-flow instructions at their end. Basic blocks fall-through to other basic blocks or contain branch instructions that have a destination basic block label. The graph of basic blocks is held in the \spverb#cfg# (control-flow graph) field of IR.

This section documents basic information about the intermediate represenation. For a tutorial based introduction to the material it is highly recommended that you read the presentation \href{http://www.jikesrvm.org/Resources/Presentations/}{Jikes RVM Optimizing Compiler Intermediate Code Representation}.

\begin{subsection}{IR Operators}

The IR operators are defined by the class \spverb#Operators#, which in turn is automatically generated from a template by a driver. The input to the driver are two files, both called \spverb#OperatorList.dat#. One input file resides in
\spverb#$RVM_ROOT/rvm/src-generated/opt-ir# and defines machine-independent operators. The other resides in
\spverb#$RVM_ROOT/rvm/src-generated/opt-ir/$\{arch\}# and defines machine-dependent operators, where \spverb#$\{arch\}# is the specific instruction architecture of interest.

Each operator in \spverb#OperatorList.dat# is defined by a five-line record, consisting of:

\begin{itemize}
  \item \spverb#SYMBOL#: a static symbol to identify the operator
  \item \spverb#INSTRUCTION_FORMAT#: the instruction format class that accepts this operator.
  \item \spverb#TRAITS#: a set of characteristics of the operator, composed with a bit-wise or (\textbar ) operator. See Operator.java for a list of valid traits.
  \item \spverb#IMPLDEFS#: set of registers implicitly defined by this operator; usually applies only to machine-dependent operators
  \item \spverb#IMPLUSES#: set of registers implicitly used by this operator; usually applies only to machine-dependent operators
\end{itemize}

For example, the entry in \spverb#OperatorList.dat# that defines the integer addition operator is
\begin{lstlisting}
INT_ADD
Binary
none
<blank line>
<blank line>
\end{lstlisting}

The operator for a conditional branch based on values of two references is defined by
\begin{lstlisting}
REF_IFCOMP
IntIfCmp
branch | conditional
<blank line>
<blank line>
\end{lstlisting}
Additionally, the machine-specific \spverb+OperatorList.dat+ file contains another line of information for use by the assembler. See the file for details.

\end{subsection}


\begin{subsection}{Instruction Format}

Every IR instruction fits one of the pre-defined \textit{Instruction Formats}. The Java package \spverb#org.jikesrvm.compilers.opt.ir# defines roughly 75 architecture\hyp independent instruction formats. For each instruction format, the package includes a class that defines a set of static methods by which optimizing compiler code can access an instruction of that format.

For example, \spverb#INT_MOVE# instructions conform to the \spverb#Move# instruction format. The following code fragment shows code that uses the \spverb#Operators# interface and the \spverb#Move# instruction format:

\begin{lstlisting}[language=Java]
import org.jikesrvm.compilers.opt.ir.*;
class X {
  void foo(Instruction s) {
    if (Move.conforms(s)) {     // if this instruction fits the Move format
      RegisterOperand r1 = Move.getResult(s);
      Operand r2 = Move.getVal(s);
      System.out.println("Found a move instruction: " + r1 + " := " + r2);
    } else {
      System.out.println(s + " is not a MOVE");
    }
  }
}
\end{lstlisting}

This example shows just a subset of the access functions defined for the Move format. Other static access functions can set each operand (in this case, \spverb#Result# and \spverb#Val#), query each operand for nullness, clear operands, create Move instructions, mutate other instructions into Move instructions, and check the index of a particular operand field in the instruction. See the Javadoc\textsuperscript{TM} reference for a complete description of the API.

Each fixed-length instruction format is defined in the text file \spverb#$RVM_ROOT/rvm/src-generated/opt-ir/InstructionFormatList.dat#. Each record in this file has four lines:

\begin{itemize}
\item \spverb#NAME#: the name of the instruction format
\item \spverb#SIZES#: the number of operands defined, defined and used, and used
\item \spverb#SIZES#: the number of operands defined, defined and used, and used
      \begin{itemize}
        \item \spverb#D/DU/U#: Is this operand a def, use, or both?
        \item \spverb#NAME#: the unique name to identify the operand
        \item \spverb#TYPE#: the type of the operand (a subclass of Operand)
        \item \spverb#[opt]#: is this operand optional?
      \end{itemize}
\item \spverb#VARSIG#: a description of repeating operands, used for variable-length instructions.
\end{itemize}

So for example, the record that defines the Move instruction format is

\begin{lstlisting}
Move
1 0 1
"D Result RegisterOperand" "U Val Operand"
<blank line>
\end{lstlisting}

This specifies that the \spverb+Move+ format has two operands, one def and one use. The def is called \spverb+Result+ and must be of type \spverb+RegisterOperand+. The use is called \spverb+Val+ and must be of type \spverb+Operand+.

A few instruction formats have variable number of operands. The format for these records is given at the top of \spverb+InstructionFormatList.dat+. For example, the record for the variable-length \spverb+Call+ instruction format is:

\begin{lstlisting}
Call
1 0 3 1 U 4
"D Result RegisterOperand" \
"U Address Operand" "U Method MethodOperand" "U Guard Operand opt"
"Param Operand"
\end{lstlisting}

This record defines the \spverb+Call+ instruction format. The second line indicates that this format always has at least 4 operands (1 def and 3 uses), plus a variable number of uses of one other type. The trailing 4 on line 2 tells the template generator to generate special constructors for cases of having 1, 2, 3, or 4 of the extra operands. Finally, the record names the \spverb+Call+ instruction operands and constrains the types. The final line specifies the name and types of the variable-numbered operands. In this case, a \spverb+Call+ instruction has a variable number of (use) operands called \spverb+Param+. Client code can access the \spverb+ith+ parameter operand of a Call instruction \spverb+s+ by calling \spverb+Call.getParam(s,i)+.

A number of instruction formats share operands of the same semantic meaning and name. For convenience in accessing like instruction formats, the template generator supports four common operand access types:
\begin{itemize}
  \item \spverb+ResultCarrier+: provides access to an operand of type \spverb+RegisterOperand+ named \spverb+Result+.
  \item \spverb+GuardResultCarrier+: provides access to an operand of type \spverb+RegisterOperand+ named \spverb+GuardResult+.
  \item \spverb+LocationCarrier+: provides access to an operand of type \spverb+LocationOperand+ named \spverb+Location+.
  \item \spverb+GuardCarrier+: provides access to an operand of type \spverb+Operand+ named \spverb+Guard+.
\end{itemize}

For example, for any instruction \spverb+s+ that carries a \spverb+Result+ operand (eg. \spverb+Move+, \spverb+Binary+, and \spverb+Unary+ formats), client code can call \spverb+ResultCarrier.conforms(s)+ and \spverb+ResultCarrier.getResult(s)+ to access the \spverb+Result+ operand.

Finally, a note on rationale. Religious object-oriented philosophers will cringe at the \spverb+InstructionFormats+. Instead, all this functionality could be implemented more cleanly with a hierarchy of instruction types exploiting (multiple) inheritance. We rejected the class hierarchy approach due to efficiency concerns of frequent virtual/interface method dispatch and type checks. Recent improvements in our interface invocation sequence and dynamic type checking algorithms may alleviate some of this concern.

\end{subsection}

\end{section}

\chapter{MMTk Tutorial}

% TODO formatting

This tutorial will build up a sophisticated garbage collector from scratch, starting with the empty shell that is the NoGC "collector" in MMTk (collector is a misnomer in this case since NoGC does not collect), and gradually adding functionality.

This tutorial will tell you the mechanics of building a collector in MMTk. It will tell you how but it does not tell you anything about why. The tutorial thus serves two purposes: 1) to give you some insight into the mechanics of MMTk (but not the underlying reasons or design rationale), and 2) show you that the mechanics of building a non-trivial GC in MMTk is not hard, hopefully giving you confidence to start exploring MMTk more deeply.
Icon

% TODO use head
The current version of the tutorial was written with respect to the Jikes RVM just prior to 3.0.2. So please use either the head or 3.0.2 (if it is available).

\end{document}