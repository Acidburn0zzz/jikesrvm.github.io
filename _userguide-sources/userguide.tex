\ifdefined\HCode
  \newcommand\setNextFileName[1]{%
  \NextFile{#1}%
  }%
  \else
  \newcommand\setNextFileName[1]{}%
\fi

\documentclass[a4paper]{book}
% TODO Separate variants for letter paper and a4 paper to support printing without hassle?
\usepackage[iso,english]{isodate} % Use ISO date formatting to prevent confusion between American date formats (mm-dd-yy) and those used by most of the rest of the world (dd-mm-yy)
% TODO We could use \usepackage[iso]{datetime2} but datetime2 is quite new right now (May 2015) and possibly not available via distributions.
\usepackage[usenames,dvipsnames]{xcolor} % for colored star on the Magic page
\usepackage{amssymb} % for \bigstar on the Magic page
% TODO possibly the symbol with an image because we probably need one for HTML
\usepackage{hyperref} % for links
\usepackage{listings} % for code listings and command lines
\usepackage{graphicx} % for images
\usepackage[utf8]{inputenc}
\title{Jikes RVM User Guide}
\author{Jikes RVM Contributors and Core Team}

\lstset{breaklines=true,breakatwhitespace=true, frame=single}

% Use something other than verbatim for terms to prevent problems with linebreaks. Then define a macro for it. For example, we could use \textt{...content...} and use \- in the content to ensure proper line breaks.

% Disable paragraph indenting
\setlength{\parindent}{0in}

\begin{document}
\maketitle

% TODO turn "published papers" into link (old one was http://docs.codehaus.org/display/RVM/Publications )
The User Guide provides Jikes™ RVM information that is not typically covered in published papers. For high-level overviews, algorithms, and structures, you will find the published papers to be the best starting place. The User Guide supplements these Jikes RVM papers, focusing on implementation details of how to build, run, and add functionality to the system.

You may find sections of the User Guide missing, incomplete or otherwise confusing. We intend this document to live as a continual work-in-progress, hopefully growing and maturing as community members edit and add to the guide. Please accept this invitation to contribute.

% TODO update for github, link mailing lists
Please send feedback, bug fixes, and text contributions to the mailing list. Constructive criticism will be cheerfully accepted.

% TODO: add links
\begin{itemize}
  \item Care and Feeding: The guide to practical aspects of building, testing, debugging and evaluating Jikes RVM.
  \item Architecture: The guide to the major architectural decisions of Jikes RVM.
  \item MMTk Tutorial: A simple tutorial to building a collector with MMTk.
\end{itemize}

\chapter{Care and Feeding}

% TODO link for Quick start guide
This section describes the practical aspects of getting started using and modifying Jikes RVM. The Quick Start Guide gives a 10 second overview on how to get started while the following sections give more detailed instructions.


\chapter{Architecture}

This section describes the architecture of Jikes RVM. The RVM can be divided into the following components:

% TODO links
\begin{itemize}
  \item Core Runtime Services: (thread scheduler, class loader, library support, verifier, etc.) This element is responsible for managing all the underlying data structures required to execute applications and interfacing with libraries.
  \item Magic: The mechanisms used by Jikes RVM to support low-level systems programming in Java.
  \item Compilers: (baseline, optimizing, JNI) This component is responsible for generating executable code from bytecodes.
  \item Memory managers: This component is responsible for the allocation and collection of objects during the execution of an application.
  \item Adaptive Optimization System: This component is responsible for profiling an executing application and judiciously using the optimizing compiler to improve its performance.
\end{itemize}

\NextFile{Magic.html}
\begin{section}{Magic}

Most Java runtimes rely upon the foreign language APIs of the underlying platform operating system to implement runtime behaviour which involves interaction with the underlying platform. Runtimes also occasionally employ small segments of machine code to provide access to platform hardware state. Note that this is expedient rather than mandatory. With a suitably smart Java bytecode compiler it would be quite possible to implement a full Java-in-Java runtime i.e. one comprising only compiled Java code (the JNode project is an attempt to implement a runtime along these lines; the Xerox, MIT, Lambda and TI Explorer Lisp machine implementations and the Xerox Smalltalk implementation were highly successful attemtps at fully compiled language runtimes).

This section provides information on \textcolor{red}{$\bigstar$} magic \textcolor{red}{$\bigstar$} which is an escape hatch that Jikes™ RVM provides to implement functionality that is not possible using the pure Java™ programming language. For example, the Jikes RVM garbage collectors and runtime system must, on occasion, access memory or perform unsafe casts. The compiler will also translate a call to Magic.threadSwitch() into a sequence of machine code that swaps out old thread registers and swaps in new ones, switching execution to the new thread's stack resumed at its saved PC

There are three mechanisms via which the Jikes RVM \textcolor{red}{$\bigstar$} magic \textcolor{red}{$\bigstar$} is implemented:
\begin{itemize}
  \item Compiler Intrinsics: Most methods are within class librarys but some functions are built in (that is, intrinsic) to the compiler. These are referred to as intrinsic functions or intrinsics.
  \item Compiler Pragmas: Some intrinsics are do not provide any behaviour but instead provide information to the compiler that modifies optimizations, calling conventions and activation frame layout. We rever to these mechanisms as compiler pragmas.
  \item Unboxed Types: Besides the primitive types, all Java values are boxed types. Conceptually, they are represented by a pointer to a heap object. However, an unboxed type is represented by the value itself. All methods on an unboxed type must be Compiler Intrinsics.
\end{itemize}

The mechanisms are used to implement the following functionality:
\begin{itemize}
  \item \hyperref[sec:rawmemoryaccess]{RawMemoryAccess}: Unfetted access to memory.
  \item Uninterruptible Code: Declaring code to be uninterruptible.
  \item Alternative Calling Conventions: Declaring different calling conventions and activation frame layout.
\end{itemize}

\end{section}

\setNextFileName{RawMemoryAccess.html}
\begin{section}{Raw Memory Access}
\label{sec:rawmemoryaccess}

The type \verb+org.vmmagic.Address+ is used to represent a machine-dependent address type. \verb+org.vmmagic.Address+ is an unboxed type. In the past, the base type \verb+int+ was used to represent addresses but this approach had several shortcomings. First, the lack of abstraction makes porting nightmarish. Equally important is that Java type \verb+int+ is signed whereas addresses are more appropriately considered unsigned. The difference is problematic since an unsigned comparison on \verb+int+ is inexpressible in the Java programming language.

To overcome these problems, instances of \verb+org.vmmagic.Address+ are used to represent addresses. The class supports the expected well-typed methods like adding an integer offset to an address to obtain another address, computing the difference of two addresses, and comparing addresses. Other operations that make sense on \verb+int+ but not on addresses are excluded like multiplication of addresses. Two methods deserve special attention: converting an address into an integer and the inverse. These methods should be avoided where possible.

Without special intervention, using a Java object to represent an address would be at best abysmally inefficient. Instead, when the Jikes RVM compiler encounters creation of an address object, it will return the primitive value that represents an address for that platform. Currently, the address type maps to either a 32-bit or 64-bit unsigned integer. Since an address is an unboxed type it must obey the rules outlined in Unboxed Types.

\end{section}

\setNextFileName{OptTestHarness.html}
\begin{section}{OptTestHarness}
\label{sec:opttestharness}

For optimizing compiler development, it is sometimes useful to exercise careful control over which classes are compiled, and with which optimization level. In many cases, a prototype-opt image will suit this process using the command line option \texttt{-X:aos:initial\_compiler=opt} combined with \texttt{-X:aos:enable\_recompilation=false}. This configuration invokes the optimizing compiler on each method run.The \spverb#OptTestHarness# provides even more control over the optimizing compiler. This driver program allows you to invoke the optimizing compiler as an "application" running on top of the VM.

\begin{table}
\begin{tabular}{p{0.47\linewidth}p{0.47\linewidth}}
-useBootOptions & Use the same OptOptions as the bootimage compiler. \\
-longcommandline \textless filename\textgreater & Read commands (one per line) from a file \\
+baseline & Switch default compiler to baseline \\
-baseline & Switch default compiler to optimizing \\
-load \textless class\textgreater & Load a class \\
-class \textless class\textgreater & Load a class and compile all its methods \\
-method \textless class\textgreater \textless method\textgreater  [- or \textless descrip\textgreater] & Compile method with default compiler \\
-methodOpt \textless class\textgreater \textless method\textgreater  [- or \textless descrip\textgreater] & Compile method with opt compiler \\
-methodBase \textless class\textgreater \textless method\textgreater  [- or \textless descrip\textgreater] & Compile method with base compiler \\
-er \textless class\textgreater \textless method\textgreater  [- or \textless descrip\textgreater] \{args\} & Compile with default compiler and execute a method \\
-performance & Show performance results \\
-oc & pass an option to the optimizing compiler \\
\end{tabular}
\caption{OptTestHarness command line options}
\end{table}

\begin{subsection}{Examples}

To use the OptTestHarness program:

\begin{lstlisting}
rvm org.jikesrvm.tools.oth.OptTestHarness -class Foo
\end{lstlisting}

will invoke the optimizing compiler on all methods of class \spverb#Foo#.

\begin{lstlisting}
rvm org.jikesrvm.tools.oth.OptTestHarness -method Foo bar -
\end{lstlisting}

will invoke the optimizing compiler on the first method bar of class \spverb#Foo# it loads.

\begin{lstlisting}
rvm org.jikesrvm.tools.oth.OptTestHarness -method Foo bar '(I)V;'
\end{lstlisting}

will invoke the optimizing compiler on method \spverb#Foo.bar(I)V;#.
You can specify any number of -method and -class options on the command line. Any arguments passed to OptTestHarness via -oc will be passed on directly to the optimizing compiler. So:

\begin{lstlisting}
rvm org.jikesrvm.tools.oth.OptTestHarness -oc:O1 -oc:print_final_hir=true -method Foo bar -
\end{lstlisting}

will compile \spverb#Foo.bar# at optimization level O1 and print the final HIR.

\end{subsection}

\end{section}


\NextFile{CostBenefitModel.html}
\begin{section}{Cost Benefit Model}
The Jikes RVM Adaptive Optimization System attempts to evaluate the break-even point for each action using an online competitive algorithm.  It relies on an analytic model to estimate the costs and benefits of each selective recompilation action, and evaluates the best actions according to the model predictions online.

A key advantage of this approach is that it allows a designer to extend the simple "break-even" cost-benefit model to account for more sophisticated adaptive policies, such as selective compilation with multiple optimization levels, on-stack-replacement, and long-running analyses.

In general, each potential action will incur some cost and may confer some benefit. For example, recompiling a method will certainly consume some CPU cycles, but could speed up the program execution by generating better code. In this discussion we focus on costs and benefits defined in terms of time (CPU cycles). However, in general, the controller could consider other measures of cost and benefit, such as memory footprint, garbage allocated, or locality disrupted.

The controller will take some action when it estimates the benefit to exceed the cost. More precisely, when the controller wakes at time $t$, it considers a set of $n$ available actions, the set $A = \{A_1, A_2, ..., A_n\}$. For any subset $S$ in $P(A)$, the controller can estimate the cost $C(S)$ and benefit $B(S)$ of performing all actions $A_i$ in $S$. The controller will attempt to choose the subset $S$ that maximizes $B(S) - C(S)$. Obviously $S = \{\}$ has $B(S) = C(S) = 0$; the controller takes no action if it cannot find a profitable course.

In practice, the precise cost and benefit of each action cannot be known; so, the controller must rely on estimates to make decisions.

The basic model the controller uses to decide which method to recompile, at which optimization level, and at what time is as follows.

Suppose that when the controller wakes at time $t$, and each method $m$ is currently optimized at optimization level $m_i, 0 \leq i \leq k$. Let $M$ be the set of loaded methods in the program. Let $A_{jm}$ be the action "recompile method m at optimization level $j$, or do nothing if $j = i$."

The controller must choose an action for each $m$ in $M$. The set of available actions is $Actions = \{A_{jm} | 0 \leq j \leq k, m \in M\}$.

Each action has an estimated cost and benefit: $C(A_{jm})$, the cost of taking action $A_{jm}$, for $0 \leq j \leq k$ and $T(A_{jm})$, the expected time the program will spend executing method $m$ in the future, if the controller takes action $A_{jm}$.

For $S$ in $Actions$, define $C(S) = \sum_{s \in S} C(s)$. Given $S$, for each $m$ in $M$, define $A_{min_m}$ to be the action $A_{jm}$ in $S$ that minimizes $T(A_{jm})$.  Then define $T(S) = \sum_{m \in M} T(A_{min_m})$.

Using these estimated values, the controller chooses the set $S$ that minimizes $C(S) + T(S)$. Intuitively, for each method $m$, the controller chooses the recompilation level $j$ that minimizes the expected future compilation time and running time of $m$.

It remains to define the functions $C$ and $T$ for each recompilation action. The basic model models the cost $C$ of compiling a method $m$ at level $j$ as a linear function of the size of $m$. The linear function is determined by an offline experiment to fit constants to the model.

The basic model estimates that the speedup for any optimization level $j$ is constant. The implementation determines the constant speedup factor for each optimization level offline, and uses the speedup to compute $T$ for each method and optimization level.

We assume that if the program has run for time $t$, then the program will run for another $t$ units, and then terminate. We further assume program behavior in the future will resemble program behavior in the past. Therefore, for each method we estimate that if no optimization action is performed $T(A_{jm})$ is equal to the time spent executing method $m$ so far.

Let $M=(m_1, ..., m_k)$ be the $k$ compiled methods. When the controller wakes at time $t$, each compiled method $m$ has been sampled $\sum m$ times. Let $\delta$ be the sampling interval, measured in seconds. The controller estimates that method $m$ has executed $\delta \sum m$ seconds so far, and will execute for another $\delta \sum m$ seconds in the future.

When driving recompilation based on sampling, the controller can limit its attention to the set of methods that were sampled in the previous sampling interval. This optimization does not lose precision; if the number of samples associated with a method has not changed, then the controller's estimate of the method's future execution time will not change. This implies that if the controller were to consider a
method that does not appear in the previous sampling interval, the controller would make exactly the same decision it did the last time it considered the method. This optimization, limiting the number of methods the controller must examine in each sampling interval, greatly reduces the amount of work performed by the controller.

Suppose the controller recompiles method m from optimization level $i$ to optimization level $j$ after having seen $\sum m$ samples. Let $S_i$ and $S_j $be the speedup ratios for optimization levels $i$ and $j$, respectively. After optimizing at level $j$, we adjust the sample data to represent the system state as if it had executed method $m$ at optimization level $j$ since program startup. So, we set the new number of samples for $m$ to be $\sum m \cdot (S_i/S_j)$. Thus to compute the time spent in $m$, we need know only one number, the "effective" number of samples.
\end{section}


// TODO convert to latex with proper inclusion of eps image which we didn't have before
Life Cycle of a Compiled Method
===============================
:author: David Grove
:date: 07-07-2008

In early implementations of Jikes RVM's adaptive system, compilation required holding a global lock that serialized compilation and also prevented classloading from occurring concurrently with compilation.  This bottleneck was removed in version 2.1.0 by switching to a finer-grained locking discipline to coordinate compilation, speculative optimization, and class loading. Since no published description of this locking protocol exists outside of the source code, we briefly summarize the life cycle of a compiled method here.

When Jikes RVM compiles a method, it creates a compiled method object to represent this particular compilation of the source method.  A compiled method has a unique id, and stores the compiled code and associated compiler meta-data. After a brief initialization phase, the compiled method transitions from uncompiled to compiling when compilation begins. During compilation, the optimizing compiler may perform speculative optimizations that can be invalidated by future class loading.  Each time the compiler so speculates, it records a relevant entry in an invalidation database.  Upon finishing compilation, the system checks to ensure that the current compilation has not already been  invalidated by concurrent classloading.  If it has not, then the system installs the compiled code, and subsequent  invocations will branch to the newly created code.

Each time a class is loaded, the system checks the invalidation database to identify the set of compiled methods to mark as obsolete,
because this classloading action invalidates speculative optimizations previously applied to that method.  A method may transition from either compiling or installed to obsolete due to a classloading-induced invalidation.  A method can also transition from installed to obsolete when the adaptive system selects a method for optimizing recompilation and a new compiled method is installed to replace it.

image:images/93224965.eps[life cycle of a compiled method]

Once a method is marked obsolete, it will never be invoked again.  However, before the generated code for the compiled method can be garbage collected, all existing invocations of the compiled method must be complete.  A compiled method transitions from obsolete to  dead when no invocations of it exist on any thread stack.  Jikes RVM detects this as part of the stack scanning phase of garbage collection; as stack frames are scanned, their compiled methods are marked as active.  Any obsolete method that is not marked as active when stack scanning completes is marked as dead and the reference to it is removed from the compiled method table.  It will then be freed during the next garbage collection


\setNextFileName{IR.html}
\begin{section}{IR}
\label{sec:ir}

The optimizing compiler intermediate representation (IR) is held in an object of type \spverb#IR# and includes a list of instructions. Every instruction is classified into one of the pre-defined instruction formats. Each instruction includes an operator and zero or more operands. Instructions are grouped into basic blocks; basic blocks are constrained to having control-flow instructions at their end. Basic blocks fall-through to other basic blocks or contain branch instructions that have a destination basic block label. The graph of basic blocks is held in the \spverb#cfg# (control-flow graph) field of IR.

This section documents basic information about the intermediate represenation. For a tutorial based introduction to the material it is highly recommended that you read the presentation \href{http://www.jikesrvm.org/Resources/Presentations/}{Jikes RVM Optimizing Compiler Intermediate Code Representation}.

\begin{subsection}{IR Operators}

The IR operators are defined by the class \spverb#Operators#, which in turn is automatically generated from a template by a driver. The input to the driver are two files, both called \spverb#OperatorList.dat#. One input file resides in
\spverb#$RVM_ROOT/rvm/src-generated/opt-ir# and defines machine-independent operators. The other resides in
\spverb#$RVM_ROOT/rvm/src-generated/opt-ir/$\{arch\}# and defines machine-dependent operators, where \spverb#$\{arch\}# is the specific instruction architecture of interest.

Each operator in \spverb#OperatorList.dat# is defined by a five-line record, consisting of:

\begin{itemize}
  \item \spverb#SYMBOL#: a static symbol to identify the operator
  \item \spverb#INSTRUCTION_FORMAT#: the instruction format class that accepts this operator.
  \item \spverb#TRAITS#: a set of characteristics of the operator, composed with a bit-wise or (\textbar ) operator. See Operator.java for a list of valid traits.
  \item \spverb#IMPLDEFS#: set of registers implicitly defined by this operator; usually applies only to machine-dependent operators
  \item \spverb#IMPLUSES#: set of registers implicitly used by this operator; usually applies only to machine-dependent operators
\end{itemize}

For example, the entry in \spverb#OperatorList.dat# that defines the integer addition operator is
\begin{lstlisting}
INT_ADD
Binary
none
<blank line>
<blank line>
\end{lstlisting}

The operator for a conditional branch based on values of two references is defined by
\begin{lstlisting}
REF_IFCOMP
IntIfCmp
branch | conditional
<blank line>
<blank line>
\end{lstlisting}
Additionally, the machine-specific \spverb+OperatorList.dat+ file contains another line of information for use by the assembler. See the file for details.

\end{subsection}


\begin{subsection}{Instruction Format}

Every IR instruction fits one of the pre-defined \textit{Instruction Formats}. The Java package \spverb#org.jikesrvm.compilers.opt.ir# defines roughly 75 architecture\hyp independent instruction formats. For each instruction format, the package includes a class that defines a set of static methods by which optimizing compiler code can access an instruction of that format.

For example, \spverb#INT_MOVE# instructions conform to the \spverb#Move# instruction format. The following code fragment shows code that uses the \spverb#Operators# interface and the \spverb#Move# instruction format:

\begin{lstlisting}[language=Java]
import org.jikesrvm.compilers.opt.ir.*;
class X {
  void foo(Instruction s) {
    if (Move.conforms(s)) {     // if this instruction fits the Move format
      RegisterOperand r1 = Move.getResult(s);
      Operand r2 = Move.getVal(s);
      System.out.println("Found a move instruction: " + r1 + " := " + r2);
    } else {
      System.out.println(s + " is not a MOVE");
    }
  }
}
\end{lstlisting}

This example shows just a subset of the access functions defined for the Move format. Other static access functions can set each operand (in this case, \spverb#Result# and \spverb#Val#), query each operand for nullness, clear operands, create Move instructions, mutate other instructions into Move instructions, and check the index of a particular operand field in the instruction. See the Javadoc\textsuperscript{TM} reference for a complete description of the API.

Each fixed-length instruction format is defined in the text file \spverb#$RVM_ROOT/rvm/src-generated/opt-ir/InstructionFormatList.dat#. Each record in this file has four lines:

\begin{itemize}
\item \spverb#NAME#: the name of the instruction format
\item \spverb#SIZES#: the number of operands defined, defined and used, and used
\item \spverb#SIZES#: the number of operands defined, defined and used, and used
      \begin{itemize}
        \item \spverb#D/DU/U#: Is this operand a def, use, or both?
        \item \spverb#NAME#: the unique name to identify the operand
        \item \spverb#TYPE#: the type of the operand (a subclass of Operand)
        \item \spverb#[opt]#: is this operand optional?
      \end{itemize}
\item \spverb#VARSIG#: a description of repeating operands, used for variable-length instructions.
\end{itemize}

So for example, the record that defines the Move instruction format is

\begin{lstlisting}
Move
1 0 1
"D Result RegisterOperand" "U Val Operand"
<blank line>
\end{lstlisting}

This specifies that the \spverb+Move+ format has two operands, one def and one use. The def is called \spverb+Result+ and must be of type \spverb+RegisterOperand+. The use is called \spverb+Val+ and must be of type \spverb+Operand+.

A few instruction formats have variable number of operands. The format for these records is given at the top of \spverb+InstructionFormatList.dat+. For example, the record for the variable-length \spverb+Call+ instruction format is:

\begin{lstlisting}
Call
1 0 3 1 U 4
"D Result RegisterOperand" \
"U Address Operand" "U Method MethodOperand" "U Guard Operand opt"
"Param Operand"
\end{lstlisting}

This record defines the \spverb+Call+ instruction format. The second line indicates that this format always has at least 4 operands (1 def and 3 uses), plus a variable number of uses of one other type. The trailing 4 on line 2 tells the template generator to generate special constructors for cases of having 1, 2, 3, or 4 of the extra operands. Finally, the record names the \spverb+Call+ instruction operands and constrains the types. The final line specifies the name and types of the variable-numbered operands. In this case, a \spverb+Call+ instruction has a variable number of (use) operands called \spverb+Param+. Client code can access the \spverb+ith+ parameter operand of a Call instruction \spverb+s+ by calling \spverb+Call.getParam(s,i)+.

A number of instruction formats share operands of the same semantic meaning and name. For convenience in accessing like instruction formats, the template generator supports four common operand access types:
\begin{itemize}
  \item \spverb+ResultCarrier+: provides access to an operand of type \spverb+RegisterOperand+ named \spverb+Result+.
  \item \spverb+GuardResultCarrier+: provides access to an operand of type \spverb+RegisterOperand+ named \spverb+GuardResult+.
  \item \spverb+LocationCarrier+: provides access to an operand of type \spverb+LocationOperand+ named \spverb+Location+.
  \item \spverb+GuardCarrier+: provides access to an operand of type \spverb+Operand+ named \spverb+Guard+.
\end{itemize}

For example, for any instruction \spverb+s+ that carries a \spverb+Result+ operand (eg. \spverb+Move+, \spverb+Binary+, and \spverb+Unary+ formats), client code can call \spverb+ResultCarrier.conforms(s)+ and \spverb+ResultCarrier.getResult(s)+ to access the \spverb+Result+ operand.

Finally, a note on rationale. Religious object-oriented philosophers will cringe at the \spverb+InstructionFormats+. Instead, all this functionality could be implemented more cleanly with a hierarchy of instruction types exploiting (multiple) inheritance. We rejected the class hierarchy approach due to efficiency concerns of frequent virtual/interface method dispatch and type checks. Recent improvements in our interface invocation sequence and dynamic type checking algorithms may alleviate some of this concern.

\end{subsection}

\end{section}

\chapter{MMTk Tutorial}

% TODO formatting

This tutorial will build up a sophisticated garbage collector from scratch, starting with the empty shell that is the NoGC "collector" in MMTk (collector is a misnomer in this case since NoGC does not collect), and gradually adding functionality.

This tutorial will tell you the mechanics of building a collector in MMTk. It will tell you how but it does not tell you anything about why. The tutorial thus serves two purposes: 1) to give you some insight into the mechanics of MMTk (but not the underlying reasons or design rationale), and 2) show you that the mechanics of building a non-trivial GC in MMTk is not hard, hopefully giving you confidence to start exploring MMTk more deeply.
Icon

% TODO use head
The current version of the tutorial was written with respect to the Jikes RVM just prior to 3.0.2. So please use either the head or 3.0.2 (if it is available).

\end{document}