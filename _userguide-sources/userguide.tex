\ifdefined\HCode
  \newcommand\setNextFileName[1]{%
  \NextFile{#1}%
  }%
  \else
  \newcommand\setNextFileName[1]{}%
\fi

\documentclass[a4paper]{book}
% TODO Separate variants for letter paper and a4 paper to support printing without hassle?
\usepackage[iso,english]{isodate} % Use ISO date formatting to prevent confusion between American date formats (mm-dd-yy) and those used by most of the rest of the world (dd-mm-yy)
% TODO We could use \usepackage[iso]{datetime2} but datetime2 is quite new right now (May 2015) and possibly not available via distributions.
\usepackage[usenames,dvipsnames]{xcolor} % for colored star on the Magic page
\usepackage{amssymb} % for \bigstar on the Magic page
\usepackage{hyperref} % for links
\usepackage{listings} % for code listings and command lines
\usepackage{graphicx} % for images
\usepackage[htt]{hyphenat} % For words that contain underscores (e.g. command line options)
\usepackage{spverbatim} % for spverbatim environment which breaks lines and \spverb macro
\usepackage{float} % for strong placement options for floats like H
\usepackage{wasysym} % for smiley on adding new gc page
\usepackage{textcomp} % for \textrightarrow
\usepackage[utf8]{inputenc}
\title{Jikes RVM User Guide}
\author{Jikes RVM Contributors and Core Team}

\lstset{breaklines=true,breakatwhitespace=true, frame=single}

% Disable paragraph indenting
\setlength{\parindent}{0in}

\begin{document}
\maketitle

% TODO When done with everything, make another pass to make sure everything looks ok in HTML and PDF

% TODO turn "published papers" into link (old one was http://docs.codehaus.org/display/RVM/Publications )
The User Guide provides Jikes™ RVM information that is not typically covered in published papers. For high-level overviews, algorithms, and structures, you will find the published papers to be the best starting place. The User Guide supplements these Jikes RVM papers, focusing on implementation details of how to build, run, and add functionality to the system.

You may find sections of the User Guide missing, incomplete or otherwise confusing. We intend this document to live as a continual work-in-progress, hopefully growing and maturing as community members edit and add to the guide. Please accept this invitation to contribute.

% TODO update for github, link mailing lists
Please send feedback, bug fixes, and text contributions to the mailing list. Constructive criticism will be cheerfully accepted.

% TODO: add links
\begin{itemize}
  \item Care and Feeding: The guide to practical aspects of building, testing, debugging and evaluating Jikes RVM.
  \item Architecture: The guide to the major architectural decisions of Jikes RVM.
  \item MMTk Tutorial: A simple tutorial to building a collector with MMTk.
\end{itemize}

\chapter{Care and Feeding}
\label{cha:careandfeeding}

% TODO link for Quick start guide
This section describes the practical aspects of getting started using and modifying Jikes RVM. The Quick Start Guide gives a 10 second overview on how to get started while the following sections give more detailed instructions.

\setNextFileName{BuildingJikesRVM.html}
\begin{chapter}{Building Jikes RVM}
\label{cha:buildingjikesrvm}

This guide describes how to build Jikes RVM. The first section is an overview of the Jikes RVM build process and this is followed by your system requirements and a detailed description of the steps required to build Jikes RVM.

Once you have things working, as described below, the \hyperref[sec:usingbuildit]{buildit script} will provide a fast and easy way to build the system.  We recommend you get things working as described below first, so you can be sure you've met the requisite dependencies etc.

\begin{section}{Overview}

To avoid problems with the build, make sure that the path to the Jikes RVM source code doesn't contain any whitespace.

If you run into trouble when building Jikes RVM, don't hesitate to ask for help on the \href{http://www.jikesrvm.org/MailingLists/}{researchers mailing list}.

\begin{subsection}{Compiling the source code}

The majority of Jikes RVM is written in Java and will be compiled into class files just as with other Java applications. There is also a small portion of Jikes RVM that is written in C that must be compiled with a C compiler such as gcc.  Jikes RVM uses \href{https://ant.apache.org}{Ant} version 1.7.0 or later as the build tool that orchestrates the build process and executes the steps required in building Jikes RVM.

Jikes RVM requires a complete install of ant, including the optional tasks. These are present if you download and install ant manually. Some Linux distributions have decided to break ant into multiple packages. So if you are installing on a platform such as Debian you may need to install another package such as 'ant-optional'.
\end{subsection}

\begin{subsection}{Generating source code}

The build process also generates Java and C source code based on build time constants such as the selected instruction architecture, garbage collectors and compilers. The generation of the source code occurs prior to the compilation phase.

\end{subsection}

\begin{subsection}{Bootstrapping Jikes RVM}

Jikes RVM compiles Java class files and produces arrays of code and data. To build itself Jikes RVM will execute on an existing Java Virtual Machine and compiles a copy of it's own class files into a boot image for the code and data using the boot image writer tool. The set of files compiled is called the \hyperref[sec:primordialclasslist]{Primordial Class List}. The boot image runner is a small C program that loads the boot image and transfers control flow into Jikes RVM.

\end{subsection}

\begin{subsection}{Class libraries}

The Java class library is the mechanism by which Java programs communicate with the outside world. Jikes RVM has configurable class library support, the most mature of which is the the \href{http://www.gnu.org/software/classpath/}{GNU Classpath} class library.

For GNU Classpath, the developer can either specify a particular version of GNU Classpath to use. By default the build process will download and build GNU Classpath.

Previous releases of the Jikes RVM had support for the Apache Harmony class library. This is no longer developed or supported because Apache Harmony development \href{https://harmony.apache.org/}{was stopped}. Support for OpenJDK is planned, but not yet implemented.

\end{subsection}

\end{section}

\begin{section}{Target Requirements}

\begin{subsection}{Architectures}
The PowerPC (or ppc) and ia32 instruction set architectures are supported by Jikes RVM.

Intel's Instruction Set Architectures (ISAs) get known by different names:

\begin{itemize}
  \item IA-32 is the name used to describe processors such as 386, 486 and the Pentium processors. It is popularly called x86 or sometimes in our documentation as x86-32.
  \item IA-32e is the name used to describe the extension of the IA-32 architecture to support 8 more registers and a 64-bit address space. It is popularly called x86\textunderscore 64 or AMD64, as AMD chips were the first to support it. It is found in processors such AMD's Opteron and Athlon 64, as well as in Intel's own Pentium 4 processors that have EM64T in their name.
  \item IA-64 is the name of Intel's Itanium processor ISA.
\end{itemize}

Jikes RVM currently supports the IA-32 ISA and work on IA32-e is in progress. As IA-32e is backward compatible with IA-32, Jikes RVM can be built and run upon IA-32e processors. The IA-64 architecture supports IA-32 code through a compatibility mode or through emulation and Jikes RVM should run in this configuration. Native IA-64 is not supported.

On PowerPC, only big endian is supported.

\end{subsection}

\begin{subsection}{Operating Systems}
Jikes RVM is capable of running on any operating system that is supported by the GNU Classpath library, low level library support is implemented and memory layout is defined. The low level library support includes interaction with the threading and signal libraries, memory management facilities and dynamic library loading services. The memory layout must also be known, as Jikes RVM will attempt to locate the boot image code and data at specific memory locations. These memory locations must not conflict with where the native compiler places it's code and data. Operating systems that are known to work include Linux and OS X. At one stage a port to win32 was completed but it was never integrated into the main Jikes RVM codebase. AIX was supported previously but support has been removed due to lack of demand. The same applies for support of Mac OS on PPC.

Note: Current implementation of Jikes RVM implies that system native libraries (like GTK+) have been compiled with frame pointers. Most of Linux distribution have frame pointers enabled in most of the packages, but some explicitly use \spverb+-fomit-frame-pointer+ thus producing the library that can't be used with Jikes RVM.
\end{subsection}

\begin{subsection}{Support Matrix}
The platform support matrix table details the targets that have historically been supported and the current status of the support. The target.name column is the identifier that Jikes RVM uses to identify this 
target. ??? means that we don't have regression machines for this platform so the Jikes RVM team can't guarantee that the target works at a given point in time. We rely on the community to provide a Jikes RVM implementation on these platforms.

\begin{table}
\centering
\begin{tabular}{lcccc}
target.name & OS & ISA & Address size & Status \\
ia32-linux & Linux & IA32 & 32 bits & OK \\
ia32-osx & OS X & IA32 & 32 bits & ??? \\
ia32-solaris & Solaris & IA32 & 32 bits & ??? \\
ia32-cygwin & Windows & IA32 & 32 bits & NYI \\
x86\textunderscore 64-linux & Linux & IA32 & \textbf{32 bits} & OK \\
x86\textunderscore 64-osx & OS X & IA32 & \textbf{32 bits} & ??? \\
x86\textunderscore 64\textunderscore m64-linux & Linux & IA32e & \textbf{64 bits} & \href{https://xtenlang.atlassian.net/browse/RVM-977}{WIP} \\
x86\textunderscore 64\textunderscore m64-osx & OS X & IA32e & \textbf{64 bits} & ??? \\
ppc32-linux & Linux & ppc32 (big e.) & 32 bits & ??? \\
ppc64-linux & Linux & ppc64 (big e.) & 64 bits & OK \\
\end{tabular}
\caption{platform support matrix}
\end{table}

x86\textunderscore 64 is currently only supported using the legacy 32bit addressing mode and instructions. You need to install the 32-bit versions of the required libraries to build and use the x86\textunderscore 64 configurations.

Note that building on Windows is currently not supported. All previous attempts at building on Windows natively (i.e. without cygwin) used the Apache Harmony classlibrary whose development has been discontinued. Support for building with cygwin is not yet implemented.

\end{subsection}

\end{section}

\begin{section}{Tool Requirements}

\begin{paragraph}{Java Virtual Machine}

Jikes RVM requires an existing Java Virtual Machine that conforms to Java 6.0 such as Oracle JDK 1.6, OpenJDK/IcedTea 6 or IBM SDK 6.0. We also aim to support the Java 7.0-conformant ans Java 8.8-conformant versions of these virtual machines.

Some Java Virtual Machines are unable to cope with compiling the Java class library so it is recommended that you install one of the above mentioned JVMs if they are not already installed on your system. The remaining build instructions assume that a suitable Java Virtual Machine is on your path. You can run \spverb+java -version+ to check you are using the correct JVM.

\end{paragraph}

\begin{paragraph}{Ant}

Ant version 1.7.0 or later is the tool required to orchestrate the build process. You can download and install the Ant tool from \href{http://ant.apache.org/}{its Apache homepage} if it is not already installed on your system. The remaining build instructions assume that \spverb+$ANT_HOME/bin+ is on your path and points to a full Ant installation (i.e. including the optional tasks). You can run \spverb+ant -version+ to check you are running the correct version of ant.

\end{paragraph}

\begin{paragraph}{C compilers}

The Jikes RVM build assumes that the GNU Compiler Collection is present on the system. Most modern *nix environments satisfy this requirement. Clang should also work but is untested.

\end{paragraph}

\begin{paragraph}{Bison}

As part of the build process, Jikes RVM uses the bison tool which should be present on most modern *nix environments.

\end{paragraph}

\begin{paragraph}{Perl}

Perl is trivially used as part of the build process but this requirement may be removed in future releases of Jikes RVM. Perl is also used as part of the regression and performance testing framework.

\end{paragraph}

\begin{paragraph}{Awk}

GNU Awk is required as part of the regression and performance testing framework but is not required when building Jikes RVM.

\end{paragraph}

\end{section}

\begin{paragraph}{Extra tools recommended for Solaris}

pkg-get will greatly simplify installing GNU packages on Solaris. Our patches require that GNU patch is picked up in preference to Sun's. You can create a symbolic link to \spverb+/usr/bin/gpatch+ from \spverb+/opt/csw/bin/patch+ and make sure \spverb+/opt/csw/bin+ is in your path before \spverb+/usr/bin+ in order to achieve this.

\end{paragraph}

\begin{section}{Instructions}

\begin{subsection}{Defining Ant properties}

There are a number of ant properties that are used to control the build process of Jikes RVM. These properties may either be specified on the command line by \spverb+-Dproperty=variable+ or they may be specified in a file named \spverb+.ant.properties+ in the base directory of the jikesrvm source tree. The \spverb+.ant.properties+ file is a standard Java property file with each line containing a \spverb+property=variable+ and comments starting with a \spverb+#+ and finishing at the end of the line.

The available properties can be grouped into properties that resolve to values and properties that resolve to directories. For properties that resolve to directories, you must make sure that the value of the property resolves to an absolute path. Relative paths aren't supported by our build system. The path must not contain any whitespace.

\begin{table}
\centering
\begin{tabular}{p{0.15\linewidth}p{0.6\linewidth}p{0.15\linewidth}}
Property & Description & Default \\
host.name & The name of the host environment used for building Jikes RVM. The host environment defines the paths to the tools used during the build, e.g. the path to the C compiler. The name should match one of the files located in the \spverb+build/hosts/+ directory minus the '.properties' extension. & None \\
target.name & The name of the target environment for Jikes RVM. The name should match one of the files located in the \spverb+build/targets/+ directory minus the '.properties' extension. This should only be specified when cross compiling the Jikes RVM. See \hyperref[sec:crossplatformbuilding]{Cross-Platform Building} for a detailed description of cross compilation. & \$\{host.name\} \\
config.name & The name of the configuration used when building Jikes RVM. The name should match one of the files located in the \spverb+build/configs/+ directory minus the '.properties' extension. This setting is further described in the section \hyperref[cha:configuringjikesrvm]{Configuring Jikes RVM}. & None \\
patch.name & An identifier for the current patch applied to the source tree. See \hyperref[sec:buildingpatchedversions]{Building Patched Versions} for a description of how this fits into the standard usage patterns of Jikes RVM. & ``\,'' \\
require.rvm-unit-tests & If set to \spverb+true+, run \hyperref[cha:testingjikesrvm]{unit tests} on the built Jikes RVM image. Use with care as it will significantly increase build times for configurations that are compiled using a non-optimizing compiler (see below). & (Undefined, tests are not run) \\
require.\newline checkstyle & Only useful if you want to \href{http://www.jikesrvm.org/Contributions/}{contribute} changes to the Jikes RVM. If set to true, run checkstyle during the build to check for violations of the Jikes RVM \hyperref[sec:codingstyle]{Coding Style} and \hyperref[sec:codingconventions]{Coding Conventions} for assertions. & (Undefined, no checks run) \\
rvm.debug-symbols & If set to true, build the Jikes RVM with debug symbols for the bootloader code and the code in the bootimage. Note: this is not enabled by default because it causes build failures for configurations that build the bootimage with the optimizing compiler (see \href{https://xtenlang.atlassian.net/browse/RVM-1084}{RVM-1084}). & (Undefined, no symbols built) \\
protect.config-files & Define this property if you do not want the build process to update configuration files when auto downloading components. & (Undefined) \\
\end{tabular}
\caption{Ant value properties for Jikes RVM}
\end{table}

\begin{table}
\centering
\begin{tabular}{p{0.15\linewidth}p{0.6\linewidth}p{0.15\linewidth}}
Property & Description & Default \\
com\-po\-nents.dir & The directory where Ant looks for external components when building Jikes RVM. & \$\{jikesrvm.\newline dir\}/com\-po\-nents \\
dist.dir & The directory where Ant stores the final Jikes RVM runtime. & \$\{jikesrvm.\newline dir\}/dist \\
build.dir & The directory where Ant stores the intermediate artifacts generated when building the Jikes RVM. & \$\{jikesrvm.\newline dir\}/tar\-get \\
com\-po\-nents.\-cache.\-dir & The directory where Ant caches downloaded components.  If you explicitly download a component, place it in this directory. & (Undefined, forcing download) \\
\end{tabular}
\caption{Ant directory properties for Jikes RVM}
\end{table}


At a minimum it is recommended that the user specify the \spverb+host.name+ property in the \spverb+.ant.properties+ file.

The configuration files in \spverb+build/targets/+ and \spverb+build/hosts/+ are designed to work with a typical install but it may be necessary to overide specific properties. The easiest way to achieve this is to specify the properties to override in the \spverb+.ant.properties+ file.

\end{subsection}

\begin{subsection}{Selecting a Configuration}

A configuration in terms of Jikes RVM is the combination of build time parameters and component selection used for a particular Jikes RVM image. The section \hyperref[cha:configuringjikesrvm]{Configuring Jikes RVM} section describes the details of how to define a configuration. Typical configuration names include:
\begin{itemize}
  \item \textbf{production}: This configuration defines a fully optimized version of the Jikes RVM.
  \item \textbf{development}: This configuration is the same as production but with debug options enabled. The debug options perform internal verification of Jikes RVM which means that it builds and executes more slowly.
  \item \textbf{prototype}: This configuration is compiled using an unoptimized compiler and includes minimal components which means it has the fastest build time.
  \item \textbf{prototype-opt}: This configuration is compiled using an unoptimized compiler but it includes the adaptive system and optimizing compiler. This configuration has a reasonably fast build time.
\end{itemize}

If a user is working on a particular configuration most of the time they may specify the config.name ant property in \spverb+.ant.properties+ otherwise it should be passed in on the command line \spverb+-Dconfig.name=...+.

\end{subsection}

\begin{subsection}{Fetching Dependencies}

The Jikes RVM has a build time dependency on the GNU Classpath class library and depending on the configuration may have a dependency on \href{http://www.cs.kent.ac.uk/projects/gc/gcspy/}{GCSpy}. The build system will attempt to download and build these dependencies if they are not present or are the wrong version.

To just download and install the GNU Classpath class library you can run the command "ant -f build/components/classpath.xml". After this command has completed running it should have downloaded and built the GNU Classpath class library for the current host. See the \hyperref[sec:usinggcspy]{Using GCSpy} page for directions on building configurations with GCSpy support.

If you wish to manually download components (for example you need to define a proxy, so ant is not correctly downloading), you can do so and identify the directory containing the downloads using \spverb+-Dcomponents.cache.dir=<download directory>+ when you build with ant.

\end{subsection}

\begin{subsection}{Building Jikes RVM}

The next step in building Jikes RVM is to run the ant command \spverb+ant+ or \spverb+ant -Dconfig.name=...+. This should build a complete RVM runtime in the directory \spverb+${dist.dir}/${config.name}_${target.name}+. A complete list of documented targets can be listed by executing the command \spverb+ant -projecthelp+.

\end{subsection}

\begin{subsection}{Running Jikes RVM}

Jikes RVM can be executed in a similar way to most Java Virtual Machines. The difference is that the command is \spverb+rvm+ and resides in the runtime directory (i.e. \spverb+${dist.dir}/${config.name}_${target.name}+). See \hyperref[cha:runningjikesrvm]{Running Jikes RVM} for a list of command line options.

\end{subsection}

\end{section}

\begin{section}{Building Patched Versions}
\label{sec:buildingpatchedversions}

As part of the research process there will be a need to evaluate a set of changes to the source tree. To make this process easier the property named patch.name can be set to a non-empty string. This will cause the output directory to have the name \spverb+${config.name}_${target.name}_${config.variant}+ rather than \spverb+${config.name}_${target.name}+, thus making it easy to differentiate between the patched and unpatched runtimes.

The following steps will create a runtime without the patch in \texttt{dist/prototype\_ia32-linux} and a runtime with the patch applied in \texttt{dist/prototype\_ia32-linux\_ReadBarriers}:

\begin{lstlisting}
% cd $RVM_ROOT
% ant -Dconfig.name=prototype -Dhost.name=ia32-linux
% patch -p0 < ReadBarriers.diff
% ant -Dconfig.variant=ReadBarriers -Dconfig.name=prototype -Dhost.name=ia32-linux
% patch -R -p0 < ReadBarriers.diff
\end{lstlisting}

The \spverb+config.variant+ property is also supported and reported as part of the test infrastructure.

\end{section}


\setNextFileName{CrossPlatformBuilding.html}
\begin{section}{Cross-Platform Building}
\label{sec:crossplatformbuilding}

The Jikes™ RVM build process consists of two major phases: the building of a \textit{boot image}, and the building of a \textit{bootloader}. The boot image is built using a Java™ program executed within a host JVM and is therefore platform-neutral. By contrast, the boot loader is written in C, and must be compiled on the target platform.

Because building the boot image can be time-consuming, you may prefer to build the boot image on a faster machine than the target platform. You may also be porting Jikes RVM to a target platform that lacks tools such or whose development environment is otherwise unpleasant. To cross-build, simply set your host.name and target.name properties to different values.

For example, to build the prototype configuration for AIX™ on a Linux host:
\begin{lstlisting}
% cd $RVM_ROOT
% ant -Dconfig.name=prototype -Dhost.name=ia32-linux -Dtarget.name=ppc32-aix cross-compile-host
\end{lstlisting}

The build process is then completed by building just the boot loader on an AIX host:
\begin{lstlisting}
% cd $RVM_ROOT
% ant -Dconfig.name=prototype -Dhost.name=ppc32-aix cross-compile-target
\end{lstlisting}

After the script has completed successfully, you should be able to run Jikes RVM.

The building of the boot loader must occur in the same directory as the rest of the build. This can either be done transparently via a network file system, or by copying the build directory from the first host to the target. 

\begin{subsection}{Dependencies}

To compile the boot image on the host system you will also need to have built any dependencies on the target machine and then copied them to the host machine. You will also need to add an appropriate line into your \newline \spverb+${components.dir}components.properties+ file such as the following (if the target system was pppc32-linux):

\begin{lstlisting}[breaklines=true,breakatwhitespace=false]
ppc32-linux.classpath.lib.dir=path/to/components/classpath/95/ppc32-linux/lib
\end{lstlisting}

It may be possible to simply build the dependencies on the host machine. Modify the \spverb+${components.dir}/components.properties+ so that the dependency property for target machine maps to the same value as the dependency property on the host machine. This works at the current time but may fail in the future if classpath changes the API between platforms. i.e.

\begin{lstlisting}[breaklines=true,breakatwhitespace=false]
ppc32-linux.classpath.lib.dir=path/to/components/classpath/95/ia32-linux/lib
\end{lstlisting}


\end{subsection}

\end{section}


\setNextFileName{PrimordialClassList.html}
\begin{section}{Primordial Class List}
\label{sec:primordialclasslist}

The primordial class list indicates which classes should be compiled and baked into the boot image. The bare minimum set of classes needed in the primordial list includes:

\begin{itemize}
  \item All classes that are needed to load a class from the file system. The class may need to be loaded as a single class file or out of a jar. Failing this there will be an infinite regress on the first class load.
  \item All classes that are needed by the baseline compiler to compile any method. Failing this we regress when attempting to compile a method so we can execute it.
  \item Enough of the core VM services and data structures, and class library (java.*) to support the above. This includes threading, memory management, runtime support for compiled code, etc.
\end{itemize}

For increased performance and decreased startup time it is possible to include extra classes that are expected to be needed, i.e. the optimizing compiler or the adaptive system. There are some pieces of these components that would be awkward to load dynamically (there's a core subset of the opt compiler, the classes in the \verb+org.jikesrvm.compilers.opt.runtimesupport+ packages, that must be loaded and fully compiled before any opt-compiled code can be allowed to executed), but it's theoretically possible to do so.

If you took a full closure of the classes referenced by things that have to be in the bootimage you'd actually end up with a lot more in the bootimage than we currently have. The culprit here would I think mainly be java.* classes that we need in the bootimage, but only use in restricted ways, so we don't actually have to drag in everything they depend on to meet the "real" constraints of what has to go in the bootimage. It is unknown how much difference there is between hand-crafted include lists and what an automated tool would discover.

\end{section}


\setNextFileName{UsingBuildit.html}
\begin{section}{Using buildit}
\label{sec:usingbuildit}

The buildit script is a handy way to build and test the system.  It has countless features and options to make building and testing really easy, particularly in a multi-machine environment, where you edit on one machine and build and test on others.  If you really want to get the most of it, take a look at all the options by running:

\begin{lstlisting}
bin/buildit -h
\end{lstlisting}

...or read the script itself.

% It is customary to have at least 2 subsections or none at all. However, examples are generally popular, so we'll make an exception here.
\begin{subsection}{Examples}

Here we just provide a hand full of examples of how it is often used, first for building and secondly for testing (which includes building). Please add to the list if you have other really useful ways of using it.  In the examples below, we'll use three hypothetical hosts: \textbf{habanero} (your desktop), \textbf{jalapeno} (a remote x86 machine) and \textbf{chipotle} (a remote PowerPC machine).

\begin{subsubsection}{Simple Builds}

To build a production image on your desktop, habanero, do the following: 

\begin{lstlisting}
bin/buildit habanero production
\end{lstlisting}

Or equivalently:

\begin{lstlisting}
bin/buildit localhost production
\end{lstlisting}

To build a production image on the remote machine jalapeno, do the following: 

\begin{lstlisting}
bin/buildit jalapeno production
\end{lstlisting}

\end{subsubsection}

\begin{subsubsection}{Cross Platform Building}

To build a production image on the remote PowerPC machine chipotle, do the following: 

\begin{lstlisting}
bin/buildit chipotle production
\end{lstlisting}

Since building on a PowerPC machine can take a long time, you might prefer to build on your x86 machine jalapeno and cross-build to chipotle.  In that case you would just do the following: 

\begin{lstlisting}
bin/buildit jalapeno -c chipotle production
\end{lstlisting}

In each case, buildit figures out the host types by interrogating them and does the right thing (forcing a PPC build on the x86 host jalapeno since you've told it you want a build for chipotle, which it knows is PPC).  Buildit caches the host information, and will prompt you the first time it encounters a new host. 

\end{subsubsection}

\begin{subsubsection}{Full Build Specification}

If you want to specify the build fully, you can do something like this:

\begin{lstlisting}
bin/buildit jalapeno FastAdaptive MarkSweep
\end{lstlisting}

If you want to specify multiple different GCs you could do:

\begin{lstlisting}
bin/buildit jalapeno FastAdaptive MarkSweep SemiSpace GenMS
\end{lstlisting}

which would build all three configurations on jalapeno.
\end{subsubsection}

\begin{subsubsection}{Profiled Builds}

Jikes RVM has the capacity to profile the boot image and then re-build an optimized boot image based on the profiles.  This process takes a little longer, but results in measurably faster builds, and so should be used when doing performance testing.  Buildit lets you do this trivially:

\begin{lstlisting}
bin/buildit jalapeno --profile production
\end{lstlisting}

\end{subsubsection}

\begin{subsubsection}{Testing}

Jikes RVM currently has a notion of a \textbf{"test-run"}, which defines a complete test scenario, including tests and builds.  An example is \textit{pre-commit}, which runs a small suite of pre-commit tests.  It also has the notion of a \textbf{"test"}, which just specifies a particular set of tests, not the full scenario.  An example is \textit{dacapo}, which just runs the DaCapo test suite (see the testing/tests directory for the available tests).

\end{subsubsection}

\begin{subsubsection}{Running a test run}
To run the pre-commit test-run on your host jalapeno just do:

\begin{lstlisting}
bin/buildit jalapeno --test-run pre-commit jalapeno
\end{lstlisting}

\end{subsubsection}

\begin{subsubsection}{Running a test}
To run the dacapo tests against a production on the host jalapeno, do:

\begin{lstlisting}
bin/buildit jalapeno -t dacapo production
\end{lstlisting}

To run the dacapo tests against a FastAdaptive MarkSweep build, on the host jalapeno, do:

\begin{lstlisting}
bin/buildit jalapeno -t dacapo FastAdaptive MarkSweep
\end{lstlisting}

To run the dacapo and SPECjvm98 tests against production on the host jalapeno, do:

\begin{lstlisting}
bin/buildit jalapeno -t dacapo -t SPECjvm98 production
\end{lstlisting}

\end{subsubsection}

\end{subsection}

\end{section}


\end{chapter}


\begin{section}{Building Patched Versions}
\label{sec:buildingpatchedversions}

As part of the research process there will be a need to evaluate a set of changes to the source tree. To make this process easier the property named patch.name can be set to a non-empty string. This will cause the output directory to have the name \spverb+${config.name}_${target.name}_${config.variant}+ rather than \spverb+${config.name}_${target.name}+, thus making it easy to differentiate between the patched and unpatched runtimes.

The following steps will create a runtime without the patch in \texttt{dist/prototype\_ia32-linux} and a runtime with the patch applied in \texttt{dist/prototype\_ia32-linux\_ReadBarriers}:

\begin{lstlisting}
% cd $RVM_ROOT
% ant -Dconfig.name=prototype -Dhost.name=ia32-linux
% patch -p0 < ReadBarriers.diff
% ant -Dconfig.variant=ReadBarriers -Dconfig.name=prototype -Dhost.name=ia32-linux
% patch -R -p0 < ReadBarriers.diff
\end{lstlisting}

The \spverb+config.variant+ property is also supported and reported as part of the test infrastructure.

\end{section}


\setNextFileName{CrossPlatformBuilding.html}
\begin{section}{Cross-Platform Building}
\label{sec:crossplatformbuilding}

The Jikes™ RVM build process consists of two major phases: the building of a \textit{boot image}, and the building of a \textit{bootloader}. The boot image is built using a Java™ program executed within a host JVM and is therefore platform-neutral. By contrast, the boot loader is written in C, and must be compiled on the target platform.

Because building the boot image can be time-consuming, you may prefer to build the boot image on a faster machine than the target platform. You may also be porting Jikes RVM to a target platform that lacks tools such or whose development environment is otherwise unpleasant. To cross-build, simply set your host.name and target.name properties to different values.

For example, to build the prototype configuration for AIX™ on a Linux host:
\begin{lstlisting}
% cd $RVM_ROOT
% ant -Dconfig.name=prototype -Dhost.name=ia32-linux -Dtarget.name=ppc32-aix cross-compile-host
\end{lstlisting}

The build process is then completed by building just the boot loader on an AIX host:
\begin{lstlisting}
% cd $RVM_ROOT
% ant -Dconfig.name=prototype -Dhost.name=ppc32-aix cross-compile-target
\end{lstlisting}

After the script has completed successfully, you should be able to run Jikes RVM.

The building of the boot loader must occur in the same directory as the rest of the build. This can either be done transparently via a network file system, or by copying the build directory from the first host to the target. 

\begin{subsection}{Dependencies}

To compile the boot image on the host system you will also need to have built any dependencies on the target machine and then copied them to the host machine. You will also need to add an appropriate line into your \newline \spverb+${components.dir}components.properties+ file such as the following (if the target system was pppc32-linux):

\begin{lstlisting}[breaklines=true,breakatwhitespace=false]
ppc32-linux.classpath.lib.dir=path/to/components/classpath/95/ppc32-linux/lib
\end{lstlisting}

It may be possible to simply build the dependencies on the host machine. Modify the \spverb+${components.dir}/components.properties+ so that the dependency property for target machine maps to the same value as the dependency property on the host machine. This works at the current time but may fail in the future if classpath changes the API between platforms. i.e.

\begin{lstlisting}[breaklines=true,breakatwhitespace=false]
ppc32-linux.classpath.lib.dir=path/to/components/classpath/95/ia32-linux/lib
\end{lstlisting}


\end{subsection}

\end{section}


\setNextFileName{PrimordialClassList.html}
\begin{section}{Primordial Class List}
\label{sec:primordialclasslist}

The primordial class list indicates which classes should be compiled and baked into the boot image. The bare minimum set of classes needed in the primordial list includes:

\begin{itemize}
  \item All classes that are needed to load a class from the file system. The class may need to be loaded as a single class file or out of a jar. Failing this there will be an infinite regress on the first class load.
  \item All classes that are needed by the baseline compiler to compile any method. Failing this we regress when attempting to compile a method so we can execute it.
  \item Enough of the core VM services and data structures, and class library (java.*) to support the above. This includes threading, memory management, runtime support for compiled code, etc.
\end{itemize}

For increased performance and decreased startup time it is possible to include extra classes that are expected to be needed, i.e. the optimizing compiler or the adaptive system. There are some pieces of these components that would be awkward to load dynamically (there's a core subset of the opt compiler, the classes in the \verb+org.jikesrvm.compilers.opt.runtimesupport+ packages, that must be loaded and fully compiled before any opt-compiled code can be allowed to executed), but it's theoretically possible to do so.

If you took a full closure of the classes referenced by things that have to be in the bootimage you'd actually end up with a lot more in the bootimage than we currently have. The culprit here would I think mainly be java.* classes that we need in the bootimage, but only use in restricted ways, so we don't actually have to drag in everything they depend on to meet the "real" constraints of what has to go in the bootimage. It is unknown how much difference there is between hand-crafted include lists and what an automated tool would discover.

\end{section}


\setNextFileName{UsingBuildit.html}
\begin{section}{Using buildit}
\label{sec:usingbuildit}

The buildit script is a handy way to build and test the system.  It has countless features and options to make building and testing really easy, particularly in a multi-machine environment, where you edit on one machine and build and test on others.  If you really want to get the most of it, take a look at all the options by running:

\begin{lstlisting}
bin/buildit -h
\end{lstlisting}

...or read the script itself.

% It is customary to have at least 2 subsections or none at all. However, examples are generally popular, so we'll make an exception here.
\begin{subsection}{Examples}

Here we just provide a hand full of examples of how it is often used, first for building and secondly for testing (which includes building). Please add to the list if you have other really useful ways of using it.  In the examples below, we'll use three hypothetical hosts: \textbf{habanero} (your desktop), \textbf{jalapeno} (a remote x86 machine) and \textbf{chipotle} (a remote PowerPC machine).

\begin{subsubsection}{Simple Builds}

To build a production image on your desktop, habanero, do the following: 

\begin{lstlisting}
bin/buildit habanero production
\end{lstlisting}

Or equivalently:

\begin{lstlisting}
bin/buildit localhost production
\end{lstlisting}

To build a production image on the remote machine jalapeno, do the following: 

\begin{lstlisting}
bin/buildit jalapeno production
\end{lstlisting}

\end{subsubsection}

\begin{subsubsection}{Cross Platform Building}

To build a production image on the remote PowerPC machine chipotle, do the following: 

\begin{lstlisting}
bin/buildit chipotle production
\end{lstlisting}

Since building on a PowerPC machine can take a long time, you might prefer to build on your x86 machine jalapeno and cross-build to chipotle.  In that case you would just do the following: 

\begin{lstlisting}
bin/buildit jalapeno -c chipotle production
\end{lstlisting}

In each case, buildit figures out the host types by interrogating them and does the right thing (forcing a PPC build on the x86 host jalapeno since you've told it you want a build for chipotle, which it knows is PPC).  Buildit caches the host information, and will prompt you the first time it encounters a new host. 

\end{subsubsection}

\begin{subsubsection}{Full Build Specification}

If you want to specify the build fully, you can do something like this:

\begin{lstlisting}
bin/buildit jalapeno FastAdaptive MarkSweep
\end{lstlisting}

If you want to specify multiple different GCs you could do:

\begin{lstlisting}
bin/buildit jalapeno FastAdaptive MarkSweep SemiSpace GenMS
\end{lstlisting}

which would build all three configurations on jalapeno.
\end{subsubsection}

\begin{subsubsection}{Profiled Builds}

Jikes RVM has the capacity to profile the boot image and then re-build an optimized boot image based on the profiles.  This process takes a little longer, but results in measurably faster builds, and so should be used when doing performance testing.  Buildit lets you do this trivially:

\begin{lstlisting}
bin/buildit jalapeno --profile production
\end{lstlisting}

\end{subsubsection}

\begin{subsubsection}{Testing}

Jikes RVM currently has a notion of a \textbf{"test-run"}, which defines a complete test scenario, including tests and builds.  An example is \textit{pre-commit}, which runs a small suite of pre-commit tests.  It also has the notion of a \textbf{"test"}, which just specifies a particular set of tests, not the full scenario.  An example is \textit{dacapo}, which just runs the DaCapo test suite (see the testing/tests directory for the available tests).

\end{subsubsection}

\begin{subsubsection}{Running a test run}
To run the pre-commit test-run on your host jalapeno just do:

\begin{lstlisting}
bin/buildit jalapeno --test-run pre-commit jalapeno
\end{lstlisting}

\end{subsubsection}

\begin{subsubsection}{Running a test}
To run the dacapo tests against a production on the host jalapeno, do:

\begin{lstlisting}
bin/buildit jalapeno -t dacapo production
\end{lstlisting}

To run the dacapo tests against a FastAdaptive MarkSweep build, on the host jalapeno, do:

\begin{lstlisting}
bin/buildit jalapeno -t dacapo FastAdaptive MarkSweep
\end{lstlisting}

To run the dacapo and SPECjvm98 tests against production on the host jalapeno, do:

\begin{lstlisting}
bin/buildit jalapeno -t dacapo -t SPECjvm98 production
\end{lstlisting}

\end{subsubsection}

\end{subsection}

\end{section}


\setNextFileName{ConfiguringJikesRVM.html}
\begin{chapter}{Configuring Jikes RVM}
\label{cha:configuringjikesrvm}

The build process requires a number of build time parameters that specify the features and components for a Jikes RVM build. Typically the build parameters are defined within a property file located in the build/configs directory. The following table defines the parameters for the build configuration.

\begin{table}
\centering
\begin{tabular}{p{0.25\linewidth}p{0.6\linewidth}p{0.15\linewidth}}
Property & Description & Default \\
config.name & A unique name that identifies the set of build parameters. & None \\
config.bootimage.\newline compiler & Parameter selects the compiler used when creating the bootimage. Must be either opt or base. & base \\
config.bootimage.\newline compiler.args & Parameter specifies any extra args that are passed to the bootimage compiler. & "" \\
config.runtime.\newline compiler & Parameter selects the compiler used at runtime. Must be either opt or base. & base \\
config.include.\newline aos & Include the adaptive system if set to true. Parameter will be ignored if config.runtime.compiler is not opt. & false \\
config.mmtk.plan & The name of the GC plan to use for the build. See MMTk for more details. & None \\
config.default-heapsize.initial & Parameter specifying the default initial heap size in MB. & 20 \\
config.default-heapsize.maximum & Parameter specifying the default maximum heap size in MB. & 100 \\
config.assertions & Parameter specifies the level of assertions in the code base. Must be one of extreme, normal or none. & normal \\
config.stress-gc-interval & The build will stress test the gc subsytem if set to a positive value. The value indicates the number of allocations between collections & 0 \\
config.include.\newline perfevent & Set to true to build Jikes RVM with support for performance counters. & false \\
config.include.gcspy & Set to true to build Jikes RVM with GCSpy support. See Using GCSpy for more details. & false \\
config.include.gcspy-client & Set to true to bundle the GCSpy client with the Jikes RVM build. Parameter will be ignored if config.include.gcspy is not true. & false \\
config.include.gcspy-stub & Set to true to use the GCSpy stub rather than the real GCSpy component. Parameter will be ignored if config.include.gcspy is not true. & false \\
config.include.all-classes & Include all the Jikes RVM classes in the bootimage if set to true. & false \\
\end{tabular}
\caption{Parameters for build configurations}
\end{table}

\begin{section}{Jikes RVM Configurations}

A typical user will use one of the existing build configurations and thus the build system only requires that the user specify the config.name property. The name should match one of the files located in the \spverb+build/configs/+ directory minus the '.properties' extension.

\begin{subsection}{Logical Configurations}

There are many possible Jikes RVM configurations. Therefore, we define four "logical" configurations that are most suitable for casual or novice users of the system. The four configurations are:

\begin{itemize}
  \item \textbf{prototype:} A simple, fast to build, but low performance configuration of Jikes RVM. This configuration does not include the optimizing compiler or adaptive system. Most useful for rapid prototyping of the core virtual machine.
  \item \textbf{prototype-opt}: A simple, fast to build, but low performance configuration of Jikes RVM. Unlike prototype, this configuration does include the optimizing compiler and adaptive system. Most useful for rapid prototyping of the core virtual machine, adaptive system, and optimizing compiler.
  \item \textbf{development:} A fully functional configuration of Jikes RVM with reasonable performance that includes the adaptive system and optimizing compiler. This configuration takes longer to build than the two prototype configurations.
  \item \textbf{production:} The same as the development configuration, except all assertions are disabled. This is the highest performance configuration of Jikes RVM and is the one to use for benchmarking and performance analysis. Build times are similar to the development configuration.
\end{itemize}

The mapping of logical to actual configurations may vary from release to release. In particular, it is expected that the choice of garbage collector for these logical configurations may be different as MMTk evolves.

Logical configurations that are not mentioned here are not recommended for novice users of the system.

\end{subsection}

\begin{subsection}{Configurations in Depth}

Most standard Jikes RVM configuration files follow the following naming scheme:

\textit{[ExtremeAssertions]} \textbf{(Base \textbar\ Full \textbar\ Fast)} (Base \textbar\ Adaptive) \textit{\textless garbage collector\textgreater }
where
\begin{itemize}
  \item \textit{ExtremeAssertions} is optional. Its presence indicates that the \texttt{con\-fig.as\-ser\-tions} configuration parameter is set to \spverb+extreme+. This turns on a number of expensive assertions.
  \item \textbf{Base \textbar\ Full \textbar\ Fast} determines the performance of the Jikes RVM boot image. \textbf{Base} denotes baseline compiler and enabled assertions, \textbf{Full} denotes optimizing compiler and enabled assertions, \textbf{Fast} denotes optimizing compiler and disabled assertions. Note that \textbf{Fast} is exclusive with \textit{ExtremeAssertions} and that \textbf{Full} and \textbf{Fast} imply that adaptive system and optimizing compiler are included in the image.
  \item Base \textbar\ Adaptive denotes whether or not the adaptive system and optimizing compiler are included in the build.
  \item the \textit{\textless garbage collector\textgreater} is the garbage collection scheme used.
\end{itemize}

Each version of Jikes RVM provides several garbage collector choices. For a definitive list of garbage collector choices, please refer to the configurations that are shipped with your version of Jikes RVM. If you need a configuration that is not available by default, you can just define your own based on the existing ones (it's easy!).

Some garbage collector suffixes that may be available are:
\begin{itemize}
  \item "NoGC" no garbage collection is performed.
  \item "SemiSpace" a copying semi-space collector
  \item "MarkSweep" a mark-and-sweep (non copying) collector
  \item "GenCopy" a classic copying generational collector with a copying higher generation
  \item "GenMS" a copying generational collector with a non-copying mark-and-sweep mature space
  \item "CopyMS" a hybrid non-generational collector with a copying space (into which all allocation goes), and a non-copying space into which survivors go
  \item "RefCount" a reference counting collector with synchronous (non-concurrent) cycle collection
\end{itemize}

For example, to specify a Jikes RVM configuration:
\begin{enumerate}
  \item with a baseline-compiled boot image,
  \item that will compile classes loaded at runtime using the baseline compiler and
  \item that uses a non-generational semi-space copying garbage collector,
\end{enumerate}

use the name \textbf{"BaseBaseSemiSpace"}.

In configurations that include the adaptive system (denoted by \textbf{"Adaptive"} in their name), methods are initially compiled by one compiler (by default the baseline compiler) and then online profiling is used to automatically select hot methods for recompilation by the optimizing compiler at an appropriate optimization level.

For example, to a build for an adaptive configuration with assertions, where the optimizing compiler is used to compile the boot image and the semi-space garbage collector is used, use the following command:

\begin{lstlisting}
% ant -Dconfig.name=FullAdaptiveSemiSpace
\end{lstlisting}

\begin{table}
\centering
\begin{tabular}{p{0.3\linewidth}p{0.35\linewidth}p{0.35\linewidth}}
Configuration & Description & Potential uses \\
BaseBaseSomeGC & baseline compiled bootimage with assertions, baseline compiler at runtime & prototyping; debugging of garbage collector SomeGC without having to worry about complexities introduced by compiler optimizations; checking for problems related to uninterruptible code \\
BaseAdaptiveSomeGC & baseline compiled bootimage with assertions, baseline compiler, adaptive system and optimizing compiler at runtime & prototyping that includes optimizing compiler and adaptive system; debugging of optimizing compiler problems with compiler advice; sanity checks with comparatively short benchmarks; checking for problems related to uninterruptible code \\
FullAdaptiveSomeGC & bootimage compiled with optimizing compiler and assertions; everything available at runtime & extensive testing including long-running benchmarks; checking for incorrect usage of unboxed types \\
ExtremeAssertions* & enables all generally useful assertions, including very expensive ones & debugging and testing in special cases \\
FastAdaptiveSomeGC & bootimage compiled with optimizing compiler; assertions disabled; everything available at runtime & benchmarking \\
FullBase* & INVALID - Full implies Adaptive & \\
FastBase* & INVALID - Fast implies Adaptive & \\
ExtremeAssertionsFast* & INVALID - ExtremeAssertions is incompatible with Fast & \\	 
\end{tabular}
\caption{Example configurations and their uses}
\end{table}

\begin{table}
\centering
\begin{tabular}{p{0.3\linewidth}p{0.3\linewidth}}
LogicalConfiguration & Actual configuration \\
prototype & BaseBaseGenImmix \\
prototype-opt & BaseAdaptiveGenImmix \\
development & FullAdaptiveGenImmix \\
production & FastAdaptiveGenImmix \\
\end{tabular}
\caption{Mapping of logical configurations to actual configurations in Jikes RVM 3.1.3}
\end{table}

\end{subsection}

\end{section}

\end{chapter}


\setNextFileName{DebuggingJikesRVM.html}
\begin{chapter}{Debugging Jikes RVM}
\label{cha:debuggingjikesrvm}

This page contains some debugging hints for Jikes RVM. It is assumed that you are familiar with debugging techniques. If you aren't, it is advisable to read a book about the subject.

\begin{section}{General debugging tips}

\begin{subsection}{Assertions}

All debugging should be done with assertion-enabled builds if possible. You can also try using ExtremeAssertion builds.

\end{subsection}

\begin{subsection}{Options}

The Jikes RVM and MMTk provide several options to print out debugging information.

If you're debugging a problem in the optimizing compiler, you can also print out the IR.

You can also use the options to change the behaviour in various ways (e.g. switch off certain optimizations) if you have a suspicion about the causes of the problem.

\end{subsection}

\begin{subsection}{Debugger Thread}

Jikes has an interactive debugger that you can invoke by sending SIGQUIT to Jikes while it's running:

\begin{lstlisting}
pkill -SIGQUIT JikesRVM
\end{lstlisting}

In previous versions of Jikes, that stopped all threads and provided an interactive prompt, but currently it just dumps the state of the VM and continues immediately (that's a known issue: \href{https://xtenlang.atlassian.net/browse/RVM-570}{RVM-570}).
Debug fields in classes

Several classes in the code base provide static boolean fields like DEBUG or VERBOSE which can be set to get more debugging information.

\end{subsection}

\begin{subsection}{Shutdown hooks}

You can write custom shutdown hooks to dump gathered information when the VM terminates. Note that shutdown hooks won't be run if the VM is terminated via a signal (see \href{https://xtenlang.atlassian.net/browse/RVM-555}{RVM-555})

Do not use the ExitMonitor from the Callbacks class because it's less reliable.

\end{subsection}

\begin{subsection}{Tests}

The test coverage is poor at the moment. Nevertheless, if you're very lucky, one of the smaller test cases will fail. See \hyperref[cha:testingjikesrvm]{Testing Jikes RVM} for details on how to run the tests and define your own.

\end{subsection}

\end{section}

\begin{section}{Tools}

There are different tools for debugging Jikes RVM:

\begin{subsection}{GDB}

There is a limited amount of C code used to start Jikes RVM. The rvm script will start Jikes RVM using GDB (the GNU debugger) if the first argument is -gdb. Break points can be set in the C code, variables, registers can be expected in the C code.

\begin{lstlisting}
rvm -gdb <RVM args> <name of Java application> <application args>
\end{lstlisting}

The dynamically created Java code doesn't provide GDB with the necessary symbol information for debugging. As some of the Java code is created in the boot image, it is possible to find the location of some Java methods and to break upon them. To build with debug symbols, you'll need to set the appropriate property as described in \hyperref[cha:buildingjikesrvm]{Building Jikes RVM}.

Details of how to manually walk the stack in GDB can be found \hyperref[sec:gdbstackwalking]{here}.
\end{subsection}

\begin{subsection}{rdb}

\href{http://sape.inf.usi.ch/rdb}{rdb} is a debugger that was developed specifically for Jikes RVM. It allows you to inspect the bootimage. If you're running Mac OS, you can also use it to debug a running Jikes RVM.
\end{subsection}

\begin{subsection}{Other Tools}

Other tools, such as valgrind, are occasionally useful in debugging or understanding the behaviour of JikesRVM.  The rvm script facilitates using these tools with the '-wrap' argument.

\begin{lstlisting}
rvm -wrap "<wrapper-script-and-args>" <rest of command line>
\end{lstlisting}

For example, cachegrind can be invoked by

\begin{lstlisting}
rvm -wrap "/path/to/valgrind --tool=cachegrind" <java-command-line>
\end{lstlisting}

The command and arguments immediately after the -wrap argument will be inserted into the script on the command line that invokes the boot image runner.  One useful variant is

\begin{lstlisting}
rvm -wrap echo <rest of command line>
\end{lstlisting}

\end{subsection}

\end{section}

\begin{section}{Debugging Optimizing Compiler Problems}

To debug problems in the optimizing compiler, use a configuration whose bootimage is compiled with the baseline compiler and which contains the AOS (prototype-opt, BaseAdaptive*). Faster configurations (such as development) have the drawback of a longer bootimage compilation time which won't be amortized unless the problem occurs late.

It is advisable to use \spverb+-X:vm:errorsFatal=true+ when debugging optimizing compiler problems. This will prevent the optimizing compiler from reverting to the baseline compiler for certain kinds of errors.

It is strongly recommended to run with advice file generation (see \hyperref[cha:experimentalguidelines]{Experimental Guidelines}). The produced advice files can then be used to try to reproduce the bug. If this step is successful, the advice files should be minimized to determine the set of methods that cause the failures. This can be done automatically (e.g. via delta debugging) or by hand.

You can also switch on paranoid IR verification in IR.java. Note that this is not well tested at the moment because we don't run any regression tests with it. Use a BaseAdaptive* configuration if you switch this on (bootimage builds with the optimizing compiler and paranoid IR fail at the time of this writing).

\begin{subsection}{Deadlocks}

To debug a deadlock, run the VM under a time limit and send SIGQUIT (to force a thread dump) a few seconds before killing the VM. On Linux, you can use the timelimit program (should be available in the repositories for Debian-based distributions).
\end{subsection}

\begin{subsection}{Excluding Garbage Collection problems}

The garbage collectors that are included with the Jikes RVM are generally stable. Therefore, if you see a problem that does not occur during the collection itself, it is likely not a garbage collection problem. You can exclude problems related to garbage collection by building with other collectors. For example, you can choose a collector that doesn't move objects (e.g. MarkSweep) or a collector that doesn't require write barriers (e.g. Immix instead of GenImmix).
\end{subsection}

\end{section}

\setNextFileName{GDBStackWalking.html}
\begin{section}{GDB Stack Walking}
\label{sec:gdbstackwalking}



Sometimes it is desirable to examine the state of the Java stack whilst using GDB to step instructions, break on addresses or watch particular variables. These instructions are based on an email sent by Martin Hirzel to the rvm-devel list around 15th September 2003. The instructions have been updated by Laurence Hellyer to deal with native threading and renamed RVM classes.

1) To learn about the stack frame layout on IA32, look at rvm/src/org/jikes\-rvm/ia32/Stack\-frame\-Layout\-Constants.java

Currently (2009/10/23) the layout is: 
\begin{lstlisting}
+4: return address
fp -> 0: caller's fp
-4: method id
(remember stack grows from high to low)
\end{lstlisting}

2) To learn how to get the current frame pointer and other context information, look at the genPrologue() method in rvm/src/org/jikesrvm/compilers/baseline/ia32/BaselineCompilerImpl.java. It first retrieves the thread register (esi on IA32), which points to an instance of RVMThread, and then retrieve fields from that instance.

3) To find the offset of field RVMThread.framePointer, add the following lines to the end of BootImageWriter.main(String[]):

\begin{lstlisting}[language=Java]
    // added to get framePointer offset from RVMThread to manually walk stacks in GDB
    say("offset of RVMThread.framePointer== " + ArchEntrypoints.framePointerField.getOffset());
\end{lstlisting}

Do a build to print this info. On my config I got +148, but this can change between versions

4) To get started, let's examine an example stack that contains methods whose code is in the boot image. We pick one that is likely to be invoked even in a simple hello-world program. In my RVM.map, 0x351eae9c is the address of org.jikesrvm.mm.mmtk.ReferenceProcessor.growReferenceTable();

5) Setting a break point on this address

\begin{lstlisting}
(gdb) break *0x351eae9c
Breakpoint 2 at 0x351eae9c
\end{lstlisting}

And run the program to the break point

\begin{lstlisting}
Breakpoint 2, 0x351eae9c in ?? ()
\end{lstlisting}

Step some instructions into the method and then dump the registers

\begin{lstlisting}
(gdb) stepi 30
0x351eaf03 in ?? ()
(gdb) info registers
eax            0x200	512
ecx            0x0	0
edx            0x0	0
ebx            0x7431	29745
esp            0x420e1934	0x420e1934
ebp            0xb0206ed0	0xb0206ed0
esi            0x4100758c	1090549132
edi            0x19c54	105556
eip            0x351eaf03	0x351eaf03
eflags         0x202	514
cs             0x17	23
ss             0x1f	31
ds             0x1f	31
es             0x1f	31
fs             0x1f	31
gs             0x37	55
\end{lstlisting}

The current FP is stored in RVMThread.framePointer which we found out in 3) is at offset +148. ESI points to the current RVMThread object so we can access the FP value like so:

\begin{lstlisting}
(gdb) x ($esi+148)
0x41007620:	0x420e1954
\end{lstlisting}

Note that the FP is at a higher address than ESP which is what we would expect

The return address is at FP+4 so to get the return address we can do:

\begin{lstlisting}
(gdb) x (*($esi+148))+4
0x420e1958:	0x351eadde
\end{lstlisting}

We can look in RVM.map for the largest method address smaller than 0x351eadde which is org.jikes\-rvm.mm.mmtk.Reference\-Processor.add\-Can\-di\-da\-te(java.\-lang.\-ref.\-Re\-fe\-rence, org.vmmagic.unboxed.Object\-Reference). Examining ReferenceProcessor.java we find that this is the only method that calls growReferenceTable so this is correct

Get the return address from the next frame

\begin{lstlisting}
(gdb) x *(*($esi+148))+4
0x420e1980:	0x352ebd1e
\end{lstlisting}

Which corresponds to org.jikes\-rvm.mm.mmtk.Reference\-Processor.add\-Soft\-Can\-di\-da\-te(java.\-lang\-.ref.\-Soft\-Reference, org.vmmagic.unboxed.Object\-Reference) which is a caller of addCandidate.

We can follow the stack back up to the top where we will read a FP of 0 (look in rvm/src/org/jikesrvm/ia32/StackframeLayoutConstants.java for details)

\end{section}


\end{chapter}


\setNextFileName{GDBStackWalking.html}
\begin{section}{GDB Stack Walking}
\label{sec:gdbstackwalking}



Sometimes it is desirable to examine the state of the Java stack whilst using GDB to step instructions, break on addresses or watch particular variables. These instructions are based on an email sent by Martin Hirzel to the rvm-devel list around 15th September 2003. The instructions have been updated by Laurence Hellyer to deal with native threading and renamed RVM classes.

1) To learn about the stack frame layout on IA32, look at rvm/src/org/jikes\-rvm/ia32/Stack\-frame\-Layout\-Constants.java

Currently (2009/10/23) the layout is: 
\begin{lstlisting}
+4: return address
fp -> 0: caller's fp
-4: method id
(remember stack grows from high to low)
\end{lstlisting}

2) To learn how to get the current frame pointer and other context information, look at the genPrologue() method in rvm/src/org/jikesrvm/compilers/baseline/ia32/BaselineCompilerImpl.java. It first retrieves the thread register (esi on IA32), which points to an instance of RVMThread, and then retrieve fields from that instance.

3) To find the offset of field RVMThread.framePointer, add the following lines to the end of BootImageWriter.main(String[]):

\begin{lstlisting}[language=Java]
    // added to get framePointer offset from RVMThread to manually walk stacks in GDB
    say("offset of RVMThread.framePointer== " + ArchEntrypoints.framePointerField.getOffset());
\end{lstlisting}

Do a build to print this info. On my config I got +148, but this can change between versions

4) To get started, let's examine an example stack that contains methods whose code is in the boot image. We pick one that is likely to be invoked even in a simple hello-world program. In my RVM.map, 0x351eae9c is the address of org.jikesrvm.mm.mmtk.ReferenceProcessor.growReferenceTable();

5) Setting a break point on this address

\begin{lstlisting}
(gdb) break *0x351eae9c
Breakpoint 2 at 0x351eae9c
\end{lstlisting}

And run the program to the break point

\begin{lstlisting}
Breakpoint 2, 0x351eae9c in ?? ()
\end{lstlisting}

Step some instructions into the method and then dump the registers

\begin{lstlisting}
(gdb) stepi 30
0x351eaf03 in ?? ()
(gdb) info registers
eax            0x200	512
ecx            0x0	0
edx            0x0	0
ebx            0x7431	29745
esp            0x420e1934	0x420e1934
ebp            0xb0206ed0	0xb0206ed0
esi            0x4100758c	1090549132
edi            0x19c54	105556
eip            0x351eaf03	0x351eaf03
eflags         0x202	514
cs             0x17	23
ss             0x1f	31
ds             0x1f	31
es             0x1f	31
fs             0x1f	31
gs             0x37	55
\end{lstlisting}

The current FP is stored in RVMThread.framePointer which we found out in 3) is at offset +148. ESI points to the current RVMThread object so we can access the FP value like so:

\begin{lstlisting}
(gdb) x ($esi+148)
0x41007620:	0x420e1954
\end{lstlisting}

Note that the FP is at a higher address than ESP which is what we would expect

The return address is at FP+4 so to get the return address we can do:

\begin{lstlisting}
(gdb) x (*($esi+148))+4
0x420e1958:	0x351eadde
\end{lstlisting}

We can look in RVM.map for the largest method address smaller than 0x351eadde which is org.jikes\-rvm.mm.mmtk.Reference\-Processor.add\-Can\-di\-da\-te(java.\-lang.\-ref.\-Re\-fe\-rence, org.vmmagic.unboxed.Object\-Reference). Examining ReferenceProcessor.java we find that this is the only method that calls growReferenceTable so this is correct

Get the return address from the next frame

\begin{lstlisting}
(gdb) x *(*($esi+148))+4
0x420e1980:	0x352ebd1e
\end{lstlisting}

Which corresponds to org.jikes\-rvm.mm.mmtk.Reference\-Processor.add\-Soft\-Can\-di\-da\-te(java.\-lang\-.ref.\-Soft\-Reference, org.vmmagic.unboxed.Object\-Reference) which is a caller of addCandidate.

We can follow the stack back up to the top where we will read a FP of 0 (look in rvm/src/org/jikesrvm/ia32/StackframeLayoutConstants.java for details)

\end{section}


\setNextFileName{ExperimentalGuidelines.html}
\begin{chapter}{Experimental Guidelines}
\label{cha:experimentalguidelines}

This section provides some tips on collecting performance numbers with Jikes RVM. 

\begin{section}{Which boot image should I use?}

To make a long story short the best performing configuration of Jikes RVM will almost always be \spverb+production+. Unless you really know what you are doing, don't use any other configuration to do a performance evaluation of Jikes RVM.

Any boot image you use for performance evaluation must have the following characteristics for the results to be meaningful:
\begin{itemize}
    \item \spverb+config.assertions=none+. Unless this is set, the runtime system and optimizing compiler will perform fairly extensive assertion checking. This introduces significant runtime overhead. By convention, a configuration with the \spverb+Fast+ prefix disables assertion checking.
    \item \spverb+config.bootimage.compiler=opt+. Unless this is set, the boot image will be compiled with the baseline compiler and virtual machine performance will be abysmal. Jikes RVM has been designed under the assumption that aggressive inlining and optimization will be applied to the VM source code.
\end{itemize}

\end{section}

\begin{section}{Compiler Replay}

The compiler-replay methodology is deterministic and eliminates memory allocation and mutator variations due to non-deterministic application of the adaptive compiler. We need this latter methodology because the non-determinism of the adaptive compilation system makes it a difficult platform for detailed performance studies. For example, we cannot determine if a variation is due to the system change being studied or just a different application of the adaptive compiler. The information we record and use are hot methods and blocks information. We also record dynamic call graph with calling frequency on each edge for inlining decisions.

\textit{Note that in December 2011, compiler replay was significantly improved.   The notes below apply to the post December 2011 version of replay.}

Here is how to use it:

\begin{subsection}{Generate Advice}

There are three kinds of advice used by the replay system, each is workload-specific (ie you should generate advice files for each benchmark):
\begin{itemize}
  \item \textbf{Compilation advice (.ca file).} This advice records for every compiled method which compiler (base or opt) and if opt, at which optimization level it should be compiled.  Replay compilation will not work without a compilation advice file.
  \item \textbf{Edge counts (.ec file).} This advice captures edge counts generated by the execution of baseline-compiled code.   Edge counts are used by the compiler to understand which edges in the control flow graph are hot.   At the time of writing, edge counts were measured as contributing about 2% to the bottom line in terms of performance (average of DaCapo, jvm98 and jbb)
  \item \textbf{Dynamic callgraph (.dc file).}  This advice captures the dynamic call graph, which allows the compiler to understand the frequency with which particular call chains occur.  This is particularly useful in guiding inlining decisions.  At the time of writing the call graph contributes about 8% to the bottom line in terms of performance.
\end{itemize}


One way to gather advice is to execute the benchmark multiple times under controlled settings, producing profiles at each execution.   Then establish the fastest execution among the set of runs, and choose the profiles associated with that execution as the advice files.   A common methodology is to invoke each benchmark 20 times (ie take the best invocation from a set of 20 trials), and in each invocation, run 10 iterations of the benchmark (ie the advice will then capture the warmed-up, steady state of the benchmark). For more advanced methodologies, please refer to current research papers on this topic.

When generating the advice, you will need to use the following command line arguments (typically use all six arguments, so that all three advice files are generated at each invocation):

\begin{lstlisting}[title=For adaptive compilation profile]
-X:aos:enable_advice_generation=true -X:aos:cafo=my_compiler_advice_file.ca
\end{lstlisting}

\begin{lstlisting}[title=For edge count profile]
-X:base:profile_edge_counters=true -X:base:profile_edge_counter_file=my_edge_counter_file.ec
\end{lstlisting}

\begin{lstlisting}[title=For dynamic call graph profile]
-X:aos:dcfo=my_dynamic_call_graph_file.dc -X:aos:final_report_level=2 
\end{lstlisting}

\end{subsection}

\begin{subsection}{Executing with advice}

The basic model is simple.  At a nominated time in the execution of a program, all methods specified in the .ca advice file will be (re)compiled with the compiler and optimization level nominated in the advice file.  Broadly, there are two ways of initiating bulk compilation: a) by calling the method \texttt{org.jikes\-rvm.adaptive.re\-com\-pi\-la\-tion.Bulk\-Compile.compile\-All\-Methods()} during execution, and b) by using the \texttt{-X:aos:enable\_precompile=true} flag at the command line to trigger bulk compilation at boot time.  A standard methodology is to use a benchmark harness call back mechanism to call \texttt{compileAllMethods()} at the end of the first iteration of the benchmark.   At the time of writing this gave performance roughly 2% faster than the 10th iteration of regular adaptive compilation.  Because precompilation occurs early, the compiler has less information about the classes, and in consequence the performance of precompilation is about 9% slower than the 10th iteration of adaptive compilation.

For \textbf{'warmup' replay} (where \texttt{org.jikes\-rvm.adaptive.re\-com\-pi\-la\-tion.Bulk\-Compile.compile\-All\-Methods()} is called at the end of the first iteration):

\begin{lstlisting}
-X:aos:initial_compiler=base -X:aos:enable_bulk_compile=true -X:aos:enable_recompilation=false -X:aos:cafi=benchmark.ca -X:vm:edgeCounterFile=benchmark.ec -X:aos:dcfi=benchmark.dc
\end{lstlisting}

For \textbf{precompile replay} (where bulk compilation occurs at boot time):

\begin{lstlisting}
-X:aos:initial_compiler=base -X:aos:enable_precompile=true -X:aos:enable_recompilation=false -X:aos:cafi=benchmark.ca -X:vm:edgeCounterFile=benchmark.ec -X:aos:dcfi=benchmark.dc
\end{lstlisting}

\end{subsection}

\begin{subsection}{Verbosity}

You can alter the verbosity of the replay behavior with the flag \texttt{-X:aos:bulk\_compilation\_verbosity}, which by default (0) is silent, but will produce more information about the recompilation with values of 1 or 2. 

\end{subsection}

\end{section}

\begin{section}{Measuring GC performance}

MMTk includes a statistics subsystem and a harness mechanism for measuring its performance.  If you are using the DaCapo benchmarks, the MMTk harness can be invoked using the '-c MMTkCallback' command line option, but for other benchmarks you will need to invoke the harness by calling the static methods

\begin{lstlisting}[language=Java]
org.mmtk.plan.Plan.harnessBegin()
org.mmtk.plan.Plan.harnessEnd()
\end{lstlisting}

at the appropriate places.  Other command line switches that affect the collection of statistics are

\begin{table}[h]
\centering
\begin{tabular}{p{0.4\linewidth}p{0.55\linewidth}}
Option & Description \\
-X:gc:printPhaseStats=true & Print statistics for each mutator/gc phase during the run \\
-X:gc:xmlStats=true & Print statistics in an XML format (as opposed to human-readable format) \\
-X:gc:verbose & This is incompatible with MMTk's statistics system. \\
-X:gc:variableSizeHeap=false & Disable dynamic resizing of the heap \\
\end{tabular}
\end{table}


Unless you are specifically researching flexible heap sizes, it is best to run benchmarks in a fixed size heap, using a range of heap sizes to produce a curve that reflects the space-time tradeoff.  Using replay compilation and measuring the second iteration of a benchmark is a good way to produce results with low noise.

There is an active debate among memory management and VM researchers about how best to measure performance, and this section is not meant to dictate or advocate any particular position, simply to describe one particular methodology.

\end{section}


\begin{section}{Jikes RVM is really slow! What am I doing wrong?}

Perhaps you are not seeing stellar Jikes\textsuperscript{TM} RVM performance. If Jikes RVM as described above is not competitive product JVMs, we recommend you test your installation with the DaCapo benchmarks. We expect Jikes RVM performance to be very close to Sun's HotSpot 1.5 server running the DaCapo benchmarks. Of course, running DaCapo well does not guarantee that Jikes RVM runs all codes well.

Some kinds of code will not run fast on Jikes RVM. Known issues include:
\begin{enumerate}
  \item Jikes RVM start-up may be slow compared to the some product JVMs.
  \item Remember that the non-adaptive configurations (\texttt{-X:aos:enable\_recompilation=false -X:aos:initial\_compiler=opt}) opt\hyp compile \textit{every} me\-thod the first time it executes. With aggressive optimization levels, opt-compiling will severely slow down the first execution of each method. For many benchmarks, it is possible to test the quality of generated code by either running for several iterations and ignoring the first, or by building a warm-up period into the code. The SPEC benchmarks already use these strategies. The adaptive configuration does not have this problem; however, we cannot stipulate that the adaptive system will compete with the product on short-running codes of a few seconds.
  \item Performance on tight loops may suffer. The Jikes RVM mechanism for safe points (thread preemption for garbage collection, on-stack-replacement, profiling, etc) relies on the insertion of a yield test on every back edge. This will hurt tight loops, including many simple microbenchmarks. We should someday alleviate this problem by strip-mining and hoisting the yield point out of hot loops, or implementing a safe point mechanism that does not require an explicit check.
  \item The load balancing in the system is naive and unfair. This can hurt some styles of codes, including bulk-synchronous parallel programs.
\end{enumerate}

The Jikes RVM developers wish to ensure that Jikes RVM delivers competitive performance. If you can isolate reproducible performance problems, please let us know.

\end{section}

\begin{section}{Stability of Jikes RVM}

Jikes RVM is not as stable as commercial JVMs such as HotSpot or J9. Design your evaluation systems (e.g. scripts) so that they can deal with crashes and deadlocks/livelocks. The latter can be dealt with by running Jikes RVM with a timelimit. For example, if you are using Linux and shell scripts, you can use the \href{http://devel.ringlet.net/sysutils/timelimit/}{timelimit} program to terminate the Jikes RVM after a set time.

\end{section}

\end{chapter}


\setNextFileName{GetTheSource.html}
\begin{chapter}{Get the Source}
\label{cha:getthesource}

The source code for the Jikes RVM is stored in a \href{http://mercurial.selenic.com/}{Mercurial} repository. You can browse the online mercurial repository at \url{http://hg.code.sourceforge.net/p/jikesrvm/code}.

A developer can either work with the version control system or download one of the releases. If you are interested in doing development of Jikes RVM you should probably use Mercurial instead of downloading a release.

\begin{section}{Download a Release}

Major and minor releases of Jikes RVM occur at regular intervals. These releases are archived in the \href{http://sourceforge.net/projects/jikesrvm/files/}{file download area} in either tar-gzip (jikesrvm-<version>.tar.gz) or tar-bzip2 (jikesrvm-<version>.tar.bz2) format. Use your web browser to download the latest version of Jikes RVM then to extract the tar-gzip archive type:

\begin{lstlisting}
$ tar xvzf jikesrvm-<version>.tar.gz
\end{lstlisting}

or for the tar-bzip2 archive type:

\begin{lstlisting}
$ tar xvjf jikesrvm-<version>.tar.bz2
\end{lstlisting}

\end{section}

\begin{section}{Use Mercurial}

The source code for Jikes RVM is stored in a Mercurial repository. Mercurial and other distributed revision control systems (e.g. Git) are quite different from centralized version control systems like CVS and Subversion. If you are not familiar with Mercurial, you can find instructions on Mercurial use at \url{http://mercurial.selenic.com/guide/}. There is also a \href{http://hgbook.red-bean.com/}{Mercurial book}.

After installing Mercurial the current version of source can be downloaded via:

\begin{lstlisting}
$ hg clone http://hg.code.sourceforge.net/p/jikesrvm/code
\end{lstlisting}

This will clone the Jikes RVM repository into the newly created directory jikesrvm.

If you need a specific version, it is recommended to clone the complete repository nonetheless. You can then switch to a specific release, e.g. 2.4.6, by doing the following:

\begin{lstlisting}
$ cd jikesrvm
$ hg checkout 2.4.6
\end{lstlisting}

If you are a not core developer you will not be able to push changes to the main Jikes RVM repository directly. If you want to contribute to the Jikes RVM, please take a look at this \href{http://www.jikesrvm.org/Contributions/}{page}.

\end{section}

\end{chapter}


\setNextFileName{ModifyingJikesRVM.html}
\begin{section}{Modifying Jikes RVM}
\label{sec:modifyingjikesrvm}

The sections \hyperref[sec:codingstyle]{Coding Style} and \hyperref[sec:codingconventions]{Coding Conventions} give a rough overview on existing coding conventions.

Jikes RVM is a bleeding-edge research project. You will find that some of the code does not live up to product quality standards. Don't hesitate to \href{http://www.jikesrvm.org/HowToHelp/}{help rectify this} by contributing clean-ups, refactorings, bug fixes, tests and missing documentation to the project.

\end{section}


\setNextFileName{AddingANewGarbageCollector.html}
\begin{section}{Adding a new garbage collector}
\label{sec:addinganewgarbagecollector}


\begin{subsection}{Overview}

This document describes how to add a new garbage collector to Jikes RVM.  We don't address how to design a new GC algorithm, just how to add a "new" GC to the system and then build it.  We do this by cloning an existing GC.  We leave it to you to design your own GC!

\end{subsection}

\begin{subsection}{Prerequisites}

Ensure that you have got a clean copy of the source (either a recent release or the hg tip) and can correctly and successfully build one of the base garbage collectors.  There's little point in trying to build your own until you can reliably build an existing one.  I suggest you start with MarkSweep, and that you use the \hyperref[sec:usingbuildit]{buildit} script:

\begin{lstlisting}
$ bin/buildit <targetmachine> BaseBase MarkSweep
\end{lstlisting}

Then test your GC:

\begin{lstlisting}
$ bin/buildit <targetmachine> -t gctest BaseBase MarkSweep
\end{lstlisting}

You should have seen some output like this:

\begin{lstlisting}
test:
     [echo] Test Result for [BaseBaseMarkSweep|gctest] InlineAllocation (default) : SUCCESS
     [echo] Test Result for [BaseBaseMarkSweep|gctest] ReferenceTest (default) : SUCCESS
     [echo] Test Result for [BaseBaseMarkSweep|gctest] ReferenceStress (default) : SUCCESS
     [echo] Test Result for [BaseBaseMarkSweep|gctest] FixedLive (default) : SUCCESS
     [echo] Test Result for [BaseBaseMarkSweep|gctest] LargeAlloc (default) : SUCCESS
     [echo] Test Result for [BaseBaseMarkSweep|gctest] Exhaust (default) : SUCCESS  
\end{lstlisting}

If this is not working, you should probably go and (re) read the section in the user guide on \hyperref[part:careandfeeding]{how to build and run} the VM.

\end{subsection}

\begin{subsection}{Cloning the MarkSweep GC}

 The best way to do this is in eclipse or a similar tool (see here for how to work with eclipse):
\begin{enumerate}
    \item Clone the \textit{org.mmtk.plan.marksweep as org.mmtk.plan.\textbf{mygc}}
      \begin{enumerate}
        \item You can do this with Eclipse:
          \begin{enumerate}
            \item Navigate to org.mmtk.plan.marksweep (within MMTk/src)
            \item Right click over org.mmtk.plan.marksweep and select "Copy"
            \item Right click again, and select "Paste", and name the target \newline org.mmtk.plan.mygc (or whatever you like)
            \item This will have cloned the marksweep GC in a new package called org.mmtk.plan.mygc
          \end{enumerate}
        \item or by hand:
          \begin{enumerate}
            \item Copy the directory MMTk/org/mmtk/plan/marksweep to \newline MMTk/org/mmtk/plan/mygc
            \item Edit each file within MMTk/org/mmtk/plan/mygc and change its package declaration to org.mmtk.plan.mygc
          \end{enumerate}
        \item We can leave the GC called "MS" for now (the file names will all be MMTk/org/mmtk/plan/mygc/MS*.java)
      \end{enumerate}
    \item Clone the BaseBaseMarkSweep.properties file as BaseBaseMyGC.properties:
      \begin{enumerate}
        \item Go to build/configs, and right click over BaseBaseMarkSweep.properties, and select "Copy"
        \item Right click and select "Paste", and paste as BaseBaseMyGC.properties
        \item Edit BaseBaseMyGC.properties, changing the text:
\begin{lstlisting}
config.mmtk.plan=org.mmtk.plan.marksweep.MS
\end{lstlisting}
to
\begin{lstlisting}
config.mmtk.plan=org.mmtk.plan.mygc.MS
\end{lstlisting}
      \end{enumerate}
    \item Now test your new GC:
\end{enumerate}
\begin{lstlisting}
$ bin/buildit <targetmachine> -t gctest BaseBase MyGC
\end{lstlisting}


You should have got similar output to your test of MarkSweep above.

That's it.  You're done. \smiley

\end{subsection}

\begin{subsection}{Making it Prettier}

You may have noticed that when you cloned the package \textit{org.mmtk.plan.marksweep}, all the classes retained their old names (although in your new namespace; \textit{org.mmtk.plan.\textbf{mygc}}).  You can trivially change the class names in an IDE like eclipse.  You can do the same with your favorite text editor, but you'll need to be sure that you change the references carefully.  To change the class names in eclipse, just follow the procedure below for each class in \textit{org.mmtk.plan.\textbf{mygc}}:
\begin{enumerate}
  \item Navigate to the class you want changed (eg  org.mmtk.plan.mygc.MS)
  \item Right click on the class (MS) and select \textit{"Refactor\textrightarrow\ Rename..."} and then type in your new name, (eg \textit{MyGC})
  \item Do the same for each of the other classes:
    \begin{itemize}
      \item MS \textrightarrow\  MyGC
      \item MSCollector \textrightarrow\  MyGCCollector
      \item MSConstraints \textrightarrow\  MyGCConstraints
      \item MSMutator \textrightarrow\  MyGCMutator
      \item MSTraceLocal \textrightarrow\  MyGCTraceLocal
    \end{itemize}
  \item Edit your configuration/s to ensure they refer to the renamed classes (since your IDE is unlikely to have done this automatically for you)
    \begin{itemize}
      \item Go to \textit{build/configs}, and edit each file \textit{*MyGC.properties} to refer to your renamed classes
    \end{itemize}
\end{enumerate}

\end{subsection}

\begin{subsection}{Beyond BaseBaseMyGC}

You probably want to build with configurations other than just BaseBase.  If so, clone configurations from MarkSweep, just as you did above (for example, clone \textit{FullAdaptiveMarkSweep} as \textit{FullAdaptive\textbf{MyGC}}). It's best to leave the Fast configurations for last, when you're sure that your GC is working correctly.

\end{subsection}

\begin{subsection}{What Next?}

Once you have this working, you have successfully created and tested your own GC without writing a line of code! You are ready to start the slightly more tricky process of writing your own garbage collector code.

If you are writing a new GC, you should definitely be aware of the \hyperref[cha:themmtktestharness]{MMTk test harness}, which allows you to test and debug MMTk in a very well contained pure Java environment, without the rest of Jikes RVM.  This allows you to write unit tests and corner cases, and moreover, allows you to edit and debug MMTk entirely from within your IDE.

\end{subsection}

\end{section}


\setNextFileName{TestingJikesRVM.html}
\begin{section}{Testing Jikes RVM}
\label{sec:testingjikesrvm}

Jikes RVM includes provisions to run unit tests as well as functional and performance tests. It also includes a number of actual tests, both unit and functional ones.

\begin{subsection}{Unit Tests}

Jikes RVM makes writing simple unit tests easy. Simply give your JUnit 4 tests a name ending in Test and place test sources under \spverb+rvm/test-src+. The tests will be picked up automatically.

The tests are then run on the bootstrap VM, i.e. the JVM used to build Jikes RVM. You can also \hyperref[sec:buildingjikesrvm]{configure the build} to run unit tests on the newly built Jikes RVM. Note that this may significantly increase the build times of slow configurations (e.g. prototype and protype-opt).

If you are developing new unit tests, it may be helpful to run them on an existing Jikes RVM image. This can be done by using the Ant target \spverb+unit-tests-on-existing-image+. The path for the image is determined by the usual properties of the Ant build.
\end{subsection}

\begin{subsection}{Functional and Performance Tests}

See \hyperref[sec:externaltestresources]{External Test Resources} for details or downloading prerequisites for the functional tests. The tests are executed using an Ant build file and produce results that conform to the definition below. The results are aggregated and processed to produce a high level report defining the status of Jikes RVM.

The testing framework was designed to support continuous and periodical execution of tests. A \textit{"test-run"} occurs every time the testing framework is invoked. Every \textit{"test-run"} will execute one or more \textit{"test-configuration"}s. A \textit{"test-configuration"} defines a particular build \textit{"configuration"} (See \hyperref[sec:configuringjikesrvm]{Configuring Jikes RVM} for details) combined with a set of parameters that are passed to the RVM during the execution of the tests. i.e. a particular \textit{"test-configuration"} may pass parameters such as \texttt{-X:aos:enable\_recompilation=false -X:aos:initial\_compiler=opt -X:irc:O1} to test the Level 1 Opt compiler optimizations.

Every \textit{"test-configuration"} will execute one or more \textit{"group"}s of tests. Every \textit{"group"} is defined by a Ant build.xml file in a separate sub-directory of \spverb+$RVM_ROOT/testing/tests+. Each \textit{"test"} has a number of input parameters such as the classname to execute, the parameters to pass to the RVM or to the program. The \textit{"test"} records a number of values such as execution time, exit code, result, standard output etc. and may also record a number of statistics if it is a performance test.

The project includes several different types of test runs and the description of each the test runs and their purpose is given in \hyperref[sec:testrundescriptions]{Test Run Descriptions}.

Note that the \hyperref[sec:usingbuildit]{buildit script} provides a fast and easy way to build and the system.  The script is simply a wrapper around the mechanisms described below.

\begin{subsubsection}{Ant properties}

There is a number of ant properties that control the test process. Besides the properties that are already defined in \hyperref[sec:buildingjikesrvm]{Building Jikes RVM}, special test properties may also be specified.


\begin{table}
\centering
\begin{tabular}{p{0.2\linewidth}p{0.5\linewidth}p{0.3\linewidth}}
Property & Description & Default \\
test-run.name & The name of the test-run. The name should match one of the files located in the build/test-runs/ directory minus the '.properties' extension. & pre-commit \\
results.dir & The directory where Ant stores the results of the test run. & \$\{jikes\-rvm.dir\}/\newline re\-sults \\
results.archive & The directory where Ant gzips and archives a copy of test run results and reports. & \$\{re\-sults.dir\}/\newline archive \\
send.reports & Define this property to send reports via email. & (Undefined) \\
mail.from & The from address used when emailing report. & jikesrvm-core@\newline lists.sourceforge.\newline net \\
mail.to & The to address used when emailing report. & jikesrvm-\newline regression@\newline lists.sourceforge.\newline net \\
mail.host & The host to connect to when sending mail. & localhost \\
mail.port & The port to connect to when sending mail. & 25 \\
\textless configuration\textgreater .\newline built & If set to true, the test process will skip the build step for specified configurations. For the test process to work the build must already be present. & (Undefined) \\
skip.build & If defined the test process will skip the build step for all configurations and the javadoc generation step. For the test process to work the build must already be present. & (Undefined) \\
skip.javadoc & If defined the test process will skip the javadoc generation step. & (Undefined) \\
\end{tabular}
\caption{Test properties}
\end{table}

\end{subsubsection}

\begin{subsubsection}{Defining a test-run}

A \textit{test-run} is defined by a number of properties located in a property file located in the \spverb+build/test-runs/+ directory.

The property test.configs is a whitespace separated list of test-configuration "tags". Every tag uniquely identifies a particular test-configuration. Every test-configuration is defined by a number of properties in the property file that are prefixed with test.config.\textless tag\textgreater . See the test run property table for the possible properties.

\begin{table}
\centering
\begin{tabular}{p{0.2\linewidth}p{0.6\linewidth}p{0.2\linewidth}}
Property & Description & Default \\
tests & The names of the test groups to execute. & None \\
name & The unique identifier for test-configuration. & "" \\
configuration & The name of the RVM build configuration to test. & \textless tag\textgreater  \\
target & The name of the RVM build target. This can be used to trigger compilation of a profiled image & "main" \\
mode & The test mode. May modify the way test groups execute. See individual groups for details. & "" \\
extra.rvm.args & Extra arguments that are passed to the RVM. These may be varied for different runs using the same image. & "" \\
\end{tabular}
\caption{Test run properties}
\end{table}

The order of the test-configurations in \textit{test.configs} is the order that the test-configurations are tested. The order of the groups in \textit{test.config.\textless tag\textgreater .test} is the order that the tests are executed.

The simplest test-run is defined in the following figure. It will use the build configuration \textit{"prototype"} and execute tests in the \textit{"basic"} group.

\begin{lstlisting}[title=build/test-runs/simple.properties]
test.configs=prototype
test.config.prototype.tests=basic
\end{lstlisting}

The test process also expands properties in the property file so it is possible to define a set of tests once but use them in multiple test-configurations as occurs in the following figure. The groups basic, optests and dacapo are executed in both the prototype and prototype-opt test configurations.

\begin{lstlisting}[title=build/test-runs/property-expansion.properties]
test.set=basic optests dacapo
test.configs=prototype prototype-opt
test.config.prototype.tests=${test.set}
test.config.prototype-opt.tests=${test.set}
\end{lstlisting}

Each test can have additional parameters specified that will be used by the test infrastructure when starting the Jikes RVM instance to execute the test. These additional parameters are described in table for test specific parameters.

\begin{table}
\centering
\begin{tabular}{p{0.2\linewidth}p{0.3\linewidth}p{0.2\linewidth}p{0.3\linewidth}}
Parameter & Description & Default Property & Default Value \\
initial.heapsize & The initial size of the heap. & \$\{test.initial.heapsize\} & \$\{config.default-heapsize.initial\} \\
max.heapsize & The initial size of the heap. & \$\{test.max.heapsize\} & \$\{config.default-heapsize.maximum\} \\
max.opt.level & The maximum optimization level for the tests or an empty string to use the Jikes RVM default. & \$\{test.max.opt.level\} & "" \\
processors & The number of processors to use for garbage collection for the test or 'all' to use all available processors. & \$\{test.processors\} & all \\
time.limit & The time limit for the test in seconds. After the time limit expires the Jikes RVM instance will be forcefully terminated. & \$\{test.time.limit\} & 1000 \\
class.path & The class path for the test. & \$\{test.class.path\} & \\
extra.args & Extra arguments that are passed to Jikes RVM. & \$\{test.rvm.extra.args\} & "" \\
exclude & If set to true, the test will be not be executed. & "" \\
\end{tabular}
\caption{Test specific parameters}
\end{table}


To determine the value of a test specific parameters, the following mechanism is used:

\begin{enumerate}
  \item    Search for one of the the following ant properties, in order.
    \begin{enumerate}
        \item test.config.\textless build-configuration\textgreater .\textless group\textgreater .\textless test\textgreater .\textless parameter\textgreater 
        \item test.config.\textless build-configuration\textgreater .\textless group\textgreater .\textless parameter\textgreater 
        \item test.config.\textless build-configuration\textgreater .\textless parameter\textgreater 
        \item test.config.\textless build-configuration\textgreater .\textless group\textgreater .\textless test\textgreater .\textless parameter\textgreater 
        \item test.config.\textless build-configuration\textgreater .\textless group\textgreater .\textless parameter\textgreater 
    \end{enumerate}
  \item If none of the above properties are defined then use the parameter that was passed to the \textless rvm\textgreater\ macro in the ant build file.
  \item If no parameter was passed to the \textless rvm\textgreater\ macro then use the default value which is stored in the "Default Property" as specified in the above table. By default the value of the "Default Property" is specified as the "Default Value" in the above table, however a particular build file may specify a different "Default Value".
\end{enumerate}

\end{subsubsection}

\begin{subsubsection}{Excluding tests}

Sometimes it is desirable to exclude tests. The test exclusion may occur as the test is known to fail on a particular target platform, build configuration or maybe it just takes too long. To exclude a test, you must define the test specific parameter "exclude" to true either in .ant.properties or in the test-run properties file.

For example, at the time of writing the Jikes RVM does not fully support volatile fields and as a result the test named "TestVolatile" in the "basic" group will always fail. Rather than being notified of this failure we can disable the test by adding a property such as "test.config.basic.TestVolatile.exclude=true" into test-run properties file.

\end{subsubsection}

\begin{subsubsection}{Executing a test-run}

The tests are executed by the Ant driver script \textit{test.xml}. The \textit{test-run.name} property defines the particular test-run to execute and if not set defaults to \textit{"sanity"}. The command \spverb+ant -f test.xml -Dtest-run.name=simple+ executes the test-run defined in \textit{build/test-runs/simple.properties}. When this command completes you can point your browser at \newline \spverb+${results.dir}/tests/${test-run.name}/Report.html+ to get an overview on test run or at \newline \spverb+${results.dir}/tests/${test-run.name}/Report.xml+ for an XML document describing test results.

\end{subsubsection}

\end{subsection}

\end{section}


\setNextFileName{ExternalTestResources.html}
\begin{section}{External Test Resources}
\label{sec:externaltestresources}

The tests included in the source tree are designed to test the correctness and performance of the Jikes RVM. This document gives a step by step instructions for setting up the external dependencies for these tests.

The first step is selecting the base directory where all the external code is to be located. The property \spverb+external.lib.dir+ needs to be set to this location. i.e.

\begin{lstlisting}[breaklines=true, breakatwhitespace=false]
> echo "external.lib.dir=/home/peter/Research/External" >> .ant.properties
> mkdir -p /home/peter/Research/External
\end{lstlisting}

Then you need to follow the instructions below for the desired benchmarks. The instructions assume that the environment variable \spverb+BENCHMARK_ROOT+ is set to the same location as the \spverb+external.lib.dir+ property.

\begin{subsection}{Open Source Benchmarks}

In the future other benchmarks such as BigInteger, \href{http://www.sable.mcgill.ca/ashes/}{Ashes} or \href{http://www.volano.com/benchmarks.html}{Volano} may be included.

\begin{subsubsection}{Dacapo}

\href{http://dacapobench.org}{Dacapo} describes itself as "This benchmark suite is intended as a tool for Java benchmarking by the programming language, memory management and computer architecture communities. It consists of a set of open source, real world applications with non-trivial memory loads. The suite is the culmination of over five years work at eight institutions, as part of the DaCapo research project, which was funded by a National Science Foundation ITR Grant, CCR-0085792."

The release needs to be downloaded and placed in the \$BENCHMARK\_ROOT/dacapo/ directory. i.e.

\begin{lstlisting}[breaklines=true, breakatwhitespace=false]
> mkdir -p $BENCHMARK_ROOT/dacapo/
> cd $BENCHMARK_ROOT/dacapo/
> wget http://sourceforge.net/projects/dacapobench/files/archive/2006-10/dacapo-2006-10.jar/download?use_mirror=autoselect"
\end{lstlisting}

\end{subsubsection}

\begin{subsubsection}{jBYTEmark}

jBYTEmark was a benchmark developed by Byte.com a long time ago.

\begin{lstlisting}[breaklines=true, breakatwhitespace=false]
> mkdir -p $BENCHMARK_ROOT/jBYTEmark-0.9
> cd $BENCHMARK_ROOT/jBYTEmark-0.9
> wget http://img.byte.com/byte/bmark/jbyte.zip
> unzip -jo jbyte.zip 'app/class/*'
> unzip -jo jbyte.zip 'app/src/jBYTEmark.java'
> ... Edit jBYTEmark.java to delete "while (true) {}" at the end of main. ...
> javac jBYTEmark.java
> jar cf jBYTEmark-0.9.jar *.class
> rm -f *.class jBYTEmark.java
\end{lstlisting}

\end{subsubsection}

\begin{subsubsection}{CaffeineMark}

\href{http://www.benchmarkhq.ru/cm30/info.html}{CaffeineMark} describes itself as "The CaffeineMark is a series of tests that measure the speed of Java programs running in various hardware and software configurations. CaffeineMark scores roughly correlate with the number of Java instructions executed per second, and do not depend significantly on the the amount of memory in the system or on the speed of a computers disk drives or internet connection."

\begin{lstlisting}[breaklines=true, breakatwhitespace=false]
> mkdir -p $BENCHMARK_ROOT/CaffeineMark-3.0
> cd $BENCHMARK_ROOT/CaffeineMark-3.0
> wget http://www.benchmarkhq.ru/cm30/cmkit.zip
> unzip cmkit.zip
\end{lstlisting}

\end{subsubsection}

\begin{subsubsection}{xerces}

Process some large documents using xerces XML parser.

\begin{lstlisting}[breaklines=true, breakatwhitespace=false]
> cd $BENCHMARK_ROOT
> wget http://archive.apache.org/dist/xml/xerces-j/Xerces-J-bin.2.8.1.tar.gz
> tar xzf Xerces-J-bin.2.8.1.tar.gz
> mkdir -p $BENCHMARK_ROOT/xmlFiles
> cd $BENCHMARK_ROOT/xmlFiles
> wget http://www.ibiblio.org/pub/sun-info/standards/xml/eg/shakespeare.1.10.xml.zip
> unzip shakespeare.1.10.xml.zip
\end{lstlisting}
\end{subsubsection}

\begin{subsubsection}{Soot}

\href{http://sable.github.io/soot/}{Soot} describes itself as "Originally, Soot started off as a Java optimization framework. By now, researchers and practitioners from around the world use Soot to analyze, instrument, optimize and visualize Java and Android applications."

\begin{lstlisting}[breaklines=true, breakatwhitespace=false]
> mkdir -p $BENCHMARK_ROOT/soot-2.2.3
> cd $BENCHMARK_ROOT/soot-2.2.3
> wget http://www.sable.mcgill.ca/software/sootclasses-2.2.3.jar
> wget http://www.sable.mcgill.ca/software/jasminclasses-2.2.3.jar
\end{lstlisting}

\end{subsubsection}

\begin{subsubsection}{Java Grande Forum Sequential Benchmarks}

Java Grande Forum Sequential Benchmarks is a benchmark suite designed for single processor execution.

\begin{lstlisting}[breaklines=true, breakatwhitespace=false]
> mkdir -p $BENCHMARK_ROOT/JavaGrandeForum
> cd $BENCHMARK_ROOT/JavaGrandeForum
> wget http://www2.epcc.ed.ac.uk/javagrande/seq/jgf_v2.tar.gz
> tar xzf jgf_v2.tar.gz
\end{lstlisting}

\end{subsubsection}

\begin{subsubsection}{Java Grande Forum Multi-threaded Benchmarks}

Java Grande Forum Multi-threaded Benchmarks is a benchmark suite designed for parallel execution on shared memory multiprocessors.

\begin{lstlisting}[breaklines=true, breakatwhitespace=false]
> mkdir -p $BENCHMARK_ROOT/JavaGrandeForum
> cd $BENCHMARK_ROOT/JavaGrandeForum
> wget http://www2.epcc.ed.ac.uk/javagrande/threads/jgf_threadv1.0.tar.gz
> tar xzf jgf_threadv1.0.tar.gz
\end{lstlisting}

\end{subsubsection}

\begin{subsubsection}{JLex Benchmark}

\href{http://www.cs.princeton.edu/~appel/modern/java/JLex/}{JLex} is a lexical analyzer generator, written for Java, in Java.

\begin{lstlisting}[breaklines=true, breakatwhitespace=false]
> mkdir -p $BENCHMARK_ROOT/JLex-1.2.6/classes/JLex
> cd $BENCHMARK_ROOT/JLex-1.2.6/classes/JLex
> wget http://www.cs.princeton.edu/~appel/modern/java/JLex/Archive/1.2.6/Main.java
> mkdir -p $BENCHMARK_ROOT/QBJC
> cd $BENCHMARK_ROOT/QBJC
> wget http://www.ocf.berkeley.edu/~horie/qbjlex.txt
> mv qbjlex.txt qb1.lex
\end{lstlisting}

\end{subsubsection}

\end{subsection}

\begin{subsection}{Proprietary Benchmarks}

\begin{subsubsection}{SPECjbb2005}

\href{http://www.spec.org/jbb2005/}{SPECjbb2005} describes itself as "SPECjbb2005 (Java Server Benchmark) is SPEC's benchmark for evaluating the performance of server side Java. Like its predecessor, SPECjbb2000, SPECjbb2005 evaluates the performance of server side Java by emulating a three-tier client/server system (with emphasis on the middle tier). The benchmark exercises the implementations of the JVM (Java Virtual Machine), JIT (Just-In-Time) compiler, garbage collection, threads and some aspects of the operating system. It also measures the performance of CPUs, caches, memory hierarchy and the scalability of shared memory processors (SMPs). SPECjbb2005 provides a new enhanced workload, implemented in a more object-oriented manner to reflect how real-world applications are designed and introduces new features such as XML processing and BigDecimal computations to make the benchmark a more realistic reflection of today's applications." SPECjbb2005 requires a license to download and use.

SPECjbb2005 can be run on command line via:

\begin{lstlisting}[breaklines=true, breakatwhitespace=false]
$RVM_ROOT/rvm -X:processors=1 -Xms400m -Xmx600m -classpath jbb.jar:check.jar spec.jbb.JBBmain -propfile SPECjbb.props
\end{lstlisting}

SPECjbb2005 may also be run as part regression tests.

\begin{lstlisting}[breaklines=true, breakatwhitespace=false]
> mkdir -p $BENCHMARK_ROOT/SPECjbb2005
> cd $BENCHMARK_ROOT/SPECjbb2005
> ...Extract package here???...
\end{lstlisting}

\end{subsubsection}

\begin{subsubsection}{SPECjbb2000}

\href{http://www.spec.org/jbb2000/}{SPECjbb2000} describes itself as "SPECjbb2000 (Java Business Benchmark) is SPEC's first benchmark for evaluating the performance of server-side Java. Joining the client-side SPECjvm98, SPECjbb2000 continues the SPEC tradition of giving Java users the most objective and representative benchmark for measuring a system's ability to run Java applications." SPECjbb2000 requires a license to download and use. Benchmarks should no longer be performed using SPECjbb2000 as the benchmarks have very different characteristics.

\begin{lstlisting}[breaklines=true, breakatwhitespace=false]
> mkdir -p $BENCHMARK_ROOT/SPECjbb2000
> cd $BENCHMARK_ROOT/SPECjbb2000
> ...Extract package here???...
\end{lstlisting}

\end{subsubsection}

\begin{subsubsection}{SPEC JVM98 Benchmarks}

\href{http://www.spec.org/jvm98/}{JVM98} features: "Measures performance of Java Virtual Machines. Applicable to networked and standalone Java client computers, either with disk (e.g., PC, workstation) or without disk (e.g., network computer) executing programs in an ordinary Java platform environment. Requires Java Virtual Machine compatible with JDK 1.1 API, or later." SPEC JVM98 Benchmarks require a license to download and use.

\begin{lstlisting}[breaklines=true, breakatwhitespace=false]
> mkdir -p $BENCHMARK_ROOT/SPECjvm98
> cd $BENCHMARK_ROOT/SPECjvm98
> ...Extract package here???...
\end{lstlisting}

\end{subsubsection}

\end{subsection}

\end{section}

\setNextFileName{TestRunDescriptions.html}
\begin{section}{Test Run Descriptions}
\label{sec:testrundescriptions}

The Jikes RVM project contains several different test runs with different purposes. This document attempts to capture the purpose of each different test run.

\begin{subsection}{pre-commit}

This test run MUST be run prior to committing code. It is relatively short and is designed to capture as many potential bugs in the shortest possible time. It is expected that the pre-commit test run will take 7-15 minutes on modern intel architecture.

\end{subsection}

\begin{subsection}{core}

There is a set of workloads we consider important (i.e. dacapo and SPEC*). There is a set of build configurations we consider important (ie prototype, development, production). We as a group wish to guarantee that all important workloads will will run correctly on all important build configurations, i.e. We should NEVER regress. The core test run is designed to identify as early as possible any failures in this matrix of build configuration x workload. It is run continuously 24 hours a day (or at least every time a change is made). It is expected that the core test run will take 2-6 hours to complete depending on the environment.

The best way to identify the failures is to stress test the system by forcing frequent garbage collections and compilation at specific optimization levels (and perhaps frequent thread switching and frequent OSR events in the future). It is critical that we have a stable research base so intermittent failures are NOT acceptable. If we can not pass a stress test then there is no guarantee that we have a stable research base.

\end{subsection}

\begin{subsection}{sanity}

The sanity test runs cover a larger number of build configurations and workloads. They may not always pass and may test many of the less frequently used configurations (gctrace, gcspy, and individual stress tests) and less important workloads. Performance tests are also included in this test run. Something we use to gauge the health of the project as a whole and to track regressions. These are run once a day on major platforms. These time to complete can vary but expected to take several hours at the least.

\end{subsection}

\begin{subsection}{Other test runs}

A set of test runs that are used for testing specific aspects of the system from performance, gcmap bug finding, io hammering etc. There may also be a set of personal/site-specific test runs included in this set that are not checked into Mercurial repository.

\end{subsection}

\begin{subsection}{Summary}

We must NEVER regress in the core test run. The pre-commit test run attempts to ensure no core regressions this while keeping running time reasonable. The sanity test run gives us an overall picture on the health of the code base. While the other test runs are used at different times for different purposes.

\end{subsection}

\end{section}

\chapter{Architecture}

This section describes the architecture of Jikes RVM. The RVM can be divided into the following components:

% TODO links
\begin{itemize}
  \item Core Runtime Services: (thread scheduler, class loader, library support, verifier, etc.) This element is responsible for managing all the underlying data structures required to execute applications and interfacing with libraries.
  \item Magic: The mechanisms used by Jikes RVM to support low-level systems programming in Java.
  \item Compilers: (baseline, optimizing, JNI) This component is responsible for generating executable code from bytecodes.
  \item Memory managers: This component is responsible for the allocation and collection of objects during the execution of an application.
  \item Adaptive Optimization System: This component is responsible for profiling an executing application and judiciously using the optimizing compiler to improve its performance.
\end{itemize}

\NextFile{Magic.html}
\begin{section}{Magic}

Most Java runtimes rely upon the foreign language APIs of the underlying platform operating system to implement runtime behaviour which involves interaction with the underlying platform. Runtimes also occasionally employ small segments of machine code to provide access to platform hardware state. Note that this is expedient rather than mandatory. With a suitably smart Java bytecode compiler it would be quite possible to implement a full Java-in-Java runtime i.e. one comprising only compiled Java code (the JNode project is an attempt to implement a runtime along these lines; the Xerox, MIT, Lambda and TI Explorer Lisp machine implementations and the Xerox Smalltalk implementation were highly successful attemtps at fully compiled language runtimes).

This section provides information on \textcolor{red}{$\bigstar$} magic \textcolor{red}{$\bigstar$} which is an escape hatch that Jikes™ RVM provides to implement functionality that is not possible using the pure Java™ programming language. For example, the Jikes RVM garbage collectors and runtime system must, on occasion, access memory or perform unsafe casts. The compiler will also translate a call to Magic.threadSwitch() into a sequence of machine code that swaps out old thread registers and swaps in new ones, switching execution to the new thread's stack resumed at its saved PC

There are three mechanisms via which the Jikes RVM \textcolor{red}{$\bigstar$} magic \textcolor{red}{$\bigstar$} is implemented:
\begin{itemize}
  \item Compiler Intrinsics: Most methods are within class librarys but some functions are built in (that is, intrinsic) to the compiler. These are referred to as intrinsic functions or intrinsics.
  \item Compiler Pragmas: Some intrinsics are do not provide any behaviour but instead provide information to the compiler that modifies optimizations, calling conventions and activation frame layout. We rever to these mechanisms as compiler pragmas.
  \item Unboxed Types: Besides the primitive types, all Java values are boxed types. Conceptually, they are represented by a pointer to a heap object. However, an unboxed type is represented by the value itself. All methods on an unboxed type must be Compiler Intrinsics.
\end{itemize}

The mechanisms are used to implement the following functionality:
\begin{itemize}
  \item \hyperref[sec:rawmemoryaccess]{RawMemoryAccess}: Unfetted access to memory.
  \item Uninterruptible Code: Declaring code to be uninterruptible.
  \item Alternative Calling Conventions: Declaring different calling conventions and activation frame layout.
\end{itemize}

\end{section}

\setNextFileName{RawMemoryAccess.html}
\begin{section}{Raw Memory Access}
\label{sec:rawmemoryaccess}

The type \verb+org.vmmagic.Address+ is used to represent a machine-dependent address type. \verb+org.vmmagic.Address+ is an unboxed type. In the past, the base type \verb+int+ was used to represent addresses but this approach had several shortcomings. First, the lack of abstraction makes porting nightmarish. Equally important is that Java type \verb+int+ is signed whereas addresses are more appropriately considered unsigned. The difference is problematic since an unsigned comparison on \verb+int+ is inexpressible in the Java programming language.

To overcome these problems, instances of \verb+org.vmmagic.Address+ are used to represent addresses. The class supports the expected well-typed methods like adding an integer offset to an address to obtain another address, computing the difference of two addresses, and comparing addresses. Other operations that make sense on \verb+int+ but not on addresses are excluded like multiplication of addresses. Two methods deserve special attention: converting an address into an integer and the inverse. These methods should be avoided where possible.

Without special intervention, using a Java object to represent an address would be at best abysmally inefficient. Instead, when the Jikes RVM compiler encounters creation of an address object, it will return the primitive value that represents an address for that platform. Currently, the address type maps to either a 32-bit or 64-bit unsigned integer. Since an address is an unboxed type it must obey the rules outlined in Unboxed Types.

\end{section}

\setNextFileName{OptTestHarness.html}
\begin{section}{OptTestHarness}
\label{sec:opttestharness}

For optimizing compiler development, it is sometimes useful to exercise careful control over which classes are compiled, and with which optimization level. In many cases, a prototype-opt image will suit this process using the command line option \texttt{-X:aos:initial\_compiler=opt} combined with \texttt{-X:aos:enable\_recompilation=false}. This configuration invokes the optimizing compiler on each method run.The \spverb#OptTestHarness# provides even more control over the optimizing compiler. This driver program allows you to invoke the optimizing compiler as an "application" running on top of the VM.

\begin{table}
\begin{tabular}{p{0.47\linewidth}p{0.47\linewidth}}
-useBootOptions & Use the same OptOptions as the bootimage compiler. \\
-longcommandline \textless filename\textgreater & Read commands (one per line) from a file \\
+baseline & Switch default compiler to baseline \\
-baseline & Switch default compiler to optimizing \\
-load \textless class\textgreater & Load a class \\
-class \textless class\textgreater & Load a class and compile all its methods \\
-method \textless class\textgreater \textless method\textgreater  [- or \textless descrip\textgreater] & Compile method with default compiler \\
-methodOpt \textless class\textgreater \textless method\textgreater  [- or \textless descrip\textgreater] & Compile method with opt compiler \\
-methodBase \textless class\textgreater \textless method\textgreater  [- or \textless descrip\textgreater] & Compile method with base compiler \\
-er \textless class\textgreater \textless method\textgreater  [- or \textless descrip\textgreater] \{args\} & Compile with default compiler and execute a method \\
-performance & Show performance results \\
-oc & pass an option to the optimizing compiler \\
\end{tabular}
\caption{OptTestHarness command line options}
\end{table}

\begin{subsection}{Examples}

To use the OptTestHarness program:

\begin{lstlisting}
rvm org.jikesrvm.tools.oth.OptTestHarness -class Foo
\end{lstlisting}

will invoke the optimizing compiler on all methods of class \spverb#Foo#.

\begin{lstlisting}
rvm org.jikesrvm.tools.oth.OptTestHarness -method Foo bar -
\end{lstlisting}

will invoke the optimizing compiler on the first method bar of class \spverb#Foo# it loads.

\begin{lstlisting}
rvm org.jikesrvm.tools.oth.OptTestHarness -method Foo bar '(I)V;'
\end{lstlisting}

will invoke the optimizing compiler on method \spverb#Foo.bar(I)V;#.
You can specify any number of -method and -class options on the command line. Any arguments passed to OptTestHarness via -oc will be passed on directly to the optimizing compiler. So:

\begin{lstlisting}
rvm org.jikesrvm.tools.oth.OptTestHarness -oc:O1 -oc:print_final_hir=true -method Foo bar -
\end{lstlisting}

will compile \spverb#Foo.bar# at optimization level O1 and print the final HIR.

\end{subsection}

\end{section}


\NextFile{CostBenefitModel.html}
\begin{section}{Cost Benefit Model}
The Jikes RVM Adaptive Optimization System attempts to evaluate the break-even point for each action using an online competitive algorithm.  It relies on an analytic model to estimate the costs and benefits of each selective recompilation action, and evaluates the best actions according to the model predictions online.

A key advantage of this approach is that it allows a designer to extend the simple "break-even" cost-benefit model to account for more sophisticated adaptive policies, such as selective compilation with multiple optimization levels, on-stack-replacement, and long-running analyses.

In general, each potential action will incur some cost and may confer some benefit. For example, recompiling a method will certainly consume some CPU cycles, but could speed up the program execution by generating better code. In this discussion we focus on costs and benefits defined in terms of time (CPU cycles). However, in general, the controller could consider other measures of cost and benefit, such as memory footprint, garbage allocated, or locality disrupted.

The controller will take some action when it estimates the benefit to exceed the cost. More precisely, when the controller wakes at time $t$, it considers a set of $n$ available actions, the set $A = \{A_1, A_2, ..., A_n\}$. For any subset $S$ in $P(A)$, the controller can estimate the cost $C(S)$ and benefit $B(S)$ of performing all actions $A_i$ in $S$. The controller will attempt to choose the subset $S$ that maximizes $B(S) - C(S)$. Obviously $S = \{\}$ has $B(S) = C(S) = 0$; the controller takes no action if it cannot find a profitable course.

In practice, the precise cost and benefit of each action cannot be known; so, the controller must rely on estimates to make decisions.

The basic model the controller uses to decide which method to recompile, at which optimization level, and at what time is as follows.

Suppose that when the controller wakes at time $t$, and each method $m$ is currently optimized at optimization level $m_i, 0 \leq i \leq k$. Let $M$ be the set of loaded methods in the program. Let $A_{jm}$ be the action "recompile method m at optimization level $j$, or do nothing if $j = i$."

The controller must choose an action for each $m$ in $M$. The set of available actions is $Actions = \{A_{jm} | 0 \leq j \leq k, m \in M\}$.

Each action has an estimated cost and benefit: $C(A_{jm})$, the cost of taking action $A_{jm}$, for $0 \leq j \leq k$ and $T(A_{jm})$, the expected time the program will spend executing method $m$ in the future, if the controller takes action $A_{jm}$.

For $S$ in $Actions$, define $C(S) = \sum_{s \in S} C(s)$. Given $S$, for each $m$ in $M$, define $A_{min_m}$ to be the action $A_{jm}$ in $S$ that minimizes $T(A_{jm})$.  Then define $T(S) = \sum_{m \in M} T(A_{min_m})$.

Using these estimated values, the controller chooses the set $S$ that minimizes $C(S) + T(S)$. Intuitively, for each method $m$, the controller chooses the recompilation level $j$ that minimizes the expected future compilation time and running time of $m$.

It remains to define the functions $C$ and $T$ for each recompilation action. The basic model models the cost $C$ of compiling a method $m$ at level $j$ as a linear function of the size of $m$. The linear function is determined by an offline experiment to fit constants to the model.

The basic model estimates that the speedup for any optimization level $j$ is constant. The implementation determines the constant speedup factor for each optimization level offline, and uses the speedup to compute $T$ for each method and optimization level.

We assume that if the program has run for time $t$, then the program will run for another $t$ units, and then terminate. We further assume program behavior in the future will resemble program behavior in the past. Therefore, for each method we estimate that if no optimization action is performed $T(A_{jm})$ is equal to the time spent executing method $m$ so far.

Let $M=(m_1, ..., m_k)$ be the $k$ compiled methods. When the controller wakes at time $t$, each compiled method $m$ has been sampled $\sum m$ times. Let $\delta$ be the sampling interval, measured in seconds. The controller estimates that method $m$ has executed $\delta \sum m$ seconds so far, and will execute for another $\delta \sum m$ seconds in the future.

When driving recompilation based on sampling, the controller can limit its attention to the set of methods that were sampled in the previous sampling interval. This optimization does not lose precision; if the number of samples associated with a method has not changed, then the controller's estimate of the method's future execution time will not change. This implies that if the controller were to consider a
method that does not appear in the previous sampling interval, the controller would make exactly the same decision it did the last time it considered the method. This optimization, limiting the number of methods the controller must examine in each sampling interval, greatly reduces the amount of work performed by the controller.

Suppose the controller recompiles method m from optimization level $i$ to optimization level $j$ after having seen $\sum m$ samples. Let $S_i$ and $S_j $be the speedup ratios for optimization levels $i$ and $j$, respectively. After optimizing at level $j$, we adjust the sample data to represent the system state as if it had executed method $m$ at optimization level $j$ since program startup. So, we set the new number of samples for $m$ to be $\sum m \cdot (S_i/S_j)$. Thus to compute the time spent in $m$, we need know only one number, the "effective" number of samples.
\end{section}


// TODO convert to latex with proper inclusion of eps image which we didn't have before
Life Cycle of a Compiled Method
===============================
:author: David Grove
:date: 07-07-2008

In early implementations of Jikes RVM's adaptive system, compilation required holding a global lock that serialized compilation and also prevented classloading from occurring concurrently with compilation.  This bottleneck was removed in version 2.1.0 by switching to a finer-grained locking discipline to coordinate compilation, speculative optimization, and class loading. Since no published description of this locking protocol exists outside of the source code, we briefly summarize the life cycle of a compiled method here.

When Jikes RVM compiles a method, it creates a compiled method object to represent this particular compilation of the source method.  A compiled method has a unique id, and stores the compiled code and associated compiler meta-data. After a brief initialization phase, the compiled method transitions from uncompiled to compiling when compilation begins. During compilation, the optimizing compiler may perform speculative optimizations that can be invalidated by future class loading.  Each time the compiler so speculates, it records a relevant entry in an invalidation database.  Upon finishing compilation, the system checks to ensure that the current compilation has not already been  invalidated by concurrent classloading.  If it has not, then the system installs the compiled code, and subsequent  invocations will branch to the newly created code.

Each time a class is loaded, the system checks the invalidation database to identify the set of compiled methods to mark as obsolete,
because this classloading action invalidates speculative optimizations previously applied to that method.  A method may transition from either compiling or installed to obsolete due to a classloading-induced invalidation.  A method can also transition from installed to obsolete when the adaptive system selects a method for optimizing recompilation and a new compiled method is installed to replace it.

image:images/93224965.eps[life cycle of a compiled method]

Once a method is marked obsolete, it will never be invoked again.  However, before the generated code for the compiled method can be garbage collected, all existing invocations of the compiled method must be complete.  A compiled method transitions from obsolete to  dead when no invocations of it exist on any thread stack.  Jikes RVM detects this as part of the stack scanning phase of garbage collection; as stack frames are scanned, their compiled methods are marked as active.  Any obsolete method that is not marked as active when stack scanning completes is marked as dead and the reference to it is removed from the compiled method table.  It will then be freed during the next garbage collection


\setNextFileName{IR.html}
\begin{section}{IR}
\label{sec:ir}

The optimizing compiler intermediate representation (IR) is held in an object of type \spverb#IR# and includes a list of instructions. Every instruction is classified into one of the pre-defined instruction formats. Each instruction includes an operator and zero or more operands. Instructions are grouped into basic blocks; basic blocks are constrained to having control-flow instructions at their end. Basic blocks fall-through to other basic blocks or contain branch instructions that have a destination basic block label. The graph of basic blocks is held in the \spverb#cfg# (control-flow graph) field of IR.

This section documents basic information about the intermediate represenation. For a tutorial based introduction to the material it is highly recommended that you read the presentation \href{http://www.jikesrvm.org/Resources/Presentations/}{Jikes RVM Optimizing Compiler Intermediate Code Representation}.

\begin{subsection}{IR Operators}

The IR operators are defined by the class \spverb#Operators#, which in turn is automatically generated from a template by a driver. The input to the driver are two files, both called \spverb#OperatorList.dat#. One input file resides in
\spverb#$RVM_ROOT/rvm/src-generated/opt-ir# and defines machine-independent operators. The other resides in
\spverb#$RVM_ROOT/rvm/src-generated/opt-ir/$\{arch\}# and defines machine-dependent operators, where \spverb#$\{arch\}# is the specific instruction architecture of interest.

Each operator in \spverb#OperatorList.dat# is defined by a five-line record, consisting of:

\begin{itemize}
  \item \spverb#SYMBOL#: a static symbol to identify the operator
  \item \spverb#INSTRUCTION_FORMAT#: the instruction format class that accepts this operator.
  \item \spverb#TRAITS#: a set of characteristics of the operator, composed with a bit-wise or (\textbar ) operator. See Operator.java for a list of valid traits.
  \item \spverb#IMPLDEFS#: set of registers implicitly defined by this operator; usually applies only to machine-dependent operators
  \item \spverb#IMPLUSES#: set of registers implicitly used by this operator; usually applies only to machine-dependent operators
\end{itemize}

For example, the entry in \spverb#OperatorList.dat# that defines the integer addition operator is
\begin{lstlisting}
INT_ADD
Binary
none
<blank line>
<blank line>
\end{lstlisting}

The operator for a conditional branch based on values of two references is defined by
\begin{lstlisting}
REF_IFCOMP
IntIfCmp
branch | conditional
<blank line>
<blank line>
\end{lstlisting}
Additionally, the machine-specific \spverb+OperatorList.dat+ file contains another line of information for use by the assembler. See the file for details.

\end{subsection}


\begin{subsection}{Instruction Format}

Every IR instruction fits one of the pre-defined \textit{Instruction Formats}. The Java package \spverb#org.jikesrvm.compilers.opt.ir# defines roughly 75 architecture\hyp independent instruction formats. For each instruction format, the package includes a class that defines a set of static methods by which optimizing compiler code can access an instruction of that format.

For example, \spverb#INT_MOVE# instructions conform to the \spverb#Move# instruction format. The following code fragment shows code that uses the \spverb#Operators# interface and the \spverb#Move# instruction format:

\begin{lstlisting}[language=Java]
import org.jikesrvm.compilers.opt.ir.*;
class X {
  void foo(Instruction s) {
    if (Move.conforms(s)) {     // if this instruction fits the Move format
      RegisterOperand r1 = Move.getResult(s);
      Operand r2 = Move.getVal(s);
      System.out.println("Found a move instruction: " + r1 + " := " + r2);
    } else {
      System.out.println(s + " is not a MOVE");
    }
  }
}
\end{lstlisting}

This example shows just a subset of the access functions defined for the Move format. Other static access functions can set each operand (in this case, \spverb#Result# and \spverb#Val#), query each operand for nullness, clear operands, create Move instructions, mutate other instructions into Move instructions, and check the index of a particular operand field in the instruction. See the Javadoc\textsuperscript{TM} reference for a complete description of the API.

Each fixed-length instruction format is defined in the text file \spverb#$RVM_ROOT/rvm/src-generated/opt-ir/InstructionFormatList.dat#. Each record in this file has four lines:

\begin{itemize}
\item \spverb#NAME#: the name of the instruction format
\item \spverb#SIZES#: the number of operands defined, defined and used, and used
\item \spverb#SIZES#: the number of operands defined, defined and used, and used
      \begin{itemize}
        \item \spverb#D/DU/U#: Is this operand a def, use, or both?
        \item \spverb#NAME#: the unique name to identify the operand
        \item \spverb#TYPE#: the type of the operand (a subclass of Operand)
        \item \spverb#[opt]#: is this operand optional?
      \end{itemize}
\item \spverb#VARSIG#: a description of repeating operands, used for variable-length instructions.
\end{itemize}

So for example, the record that defines the Move instruction format is

\begin{lstlisting}
Move
1 0 1
"D Result RegisterOperand" "U Val Operand"
<blank line>
\end{lstlisting}

This specifies that the \spverb+Move+ format has two operands, one def and one use. The def is called \spverb+Result+ and must be of type \spverb+RegisterOperand+. The use is called \spverb+Val+ and must be of type \spverb+Operand+.

A few instruction formats have variable number of operands. The format for these records is given at the top of \spverb+InstructionFormatList.dat+. For example, the record for the variable-length \spverb+Call+ instruction format is:

\begin{lstlisting}
Call
1 0 3 1 U 4
"D Result RegisterOperand" \
"U Address Operand" "U Method MethodOperand" "U Guard Operand opt"
"Param Operand"
\end{lstlisting}

This record defines the \spverb+Call+ instruction format. The second line indicates that this format always has at least 4 operands (1 def and 3 uses), plus a variable number of uses of one other type. The trailing 4 on line 2 tells the template generator to generate special constructors for cases of having 1, 2, 3, or 4 of the extra operands. Finally, the record names the \spverb+Call+ instruction operands and constrains the types. The final line specifies the name and types of the variable-numbered operands. In this case, a \spverb+Call+ instruction has a variable number of (use) operands called \spverb+Param+. Client code can access the \spverb+ith+ parameter operand of a Call instruction \spverb+s+ by calling \spverb+Call.getParam(s,i)+.

A number of instruction formats share operands of the same semantic meaning and name. For convenience in accessing like instruction formats, the template generator supports four common operand access types:
\begin{itemize}
  \item \spverb+ResultCarrier+: provides access to an operand of type \spverb+RegisterOperand+ named \spverb+Result+.
  \item \spverb+GuardResultCarrier+: provides access to an operand of type \spverb+RegisterOperand+ named \spverb+GuardResult+.
  \item \spverb+LocationCarrier+: provides access to an operand of type \spverb+LocationOperand+ named \spverb+Location+.
  \item \spverb+GuardCarrier+: provides access to an operand of type \spverb+Operand+ named \spverb+Guard+.
\end{itemize}

For example, for any instruction \spverb+s+ that carries a \spverb+Result+ operand (eg. \spverb+Move+, \spverb+Binary+, and \spverb+Unary+ formats), client code can call \spverb+ResultCarrier.conforms(s)+ and \spverb+ResultCarrier.getResult(s)+ to access the \spverb+Result+ operand.

Finally, a note on rationale. Religious object-oriented philosophers will cringe at the \spverb+InstructionFormats+. Instead, all this functionality could be implemented more cleanly with a hierarchy of instruction types exploiting (multiple) inheritance. We rejected the class hierarchy approach due to efficiency concerns of frequent virtual/interface method dispatch and type checks. Recent improvements in our interface invocation sequence and dynamic type checking algorithms may alleviate some of this concern.

\end{subsection}

\end{section}

\chapter{MMTk Tutorial}

% TODO formatting

This tutorial will build up a sophisticated garbage collector from scratch, starting with the empty shell that is the NoGC "collector" in MMTk (collector is a misnomer in this case since NoGC does not collect), and gradually adding functionality.

This tutorial will tell you the mechanics of building a collector in MMTk. It will tell you how but it does not tell you anything about why. The tutorial thus serves two purposes: 1) to give you some insight into the mechanics of MMTk (but not the underlying reasons or design rationale), and 2) show you that the mechanics of building a non-trivial GC in MMTk is not hard, hopefully giving you confidence to start exploring MMTk more deeply.
Icon

% TODO use head
The current version of the tutorial was written with respect to the Jikes RVM just prior to 3.0.2. So please use either the head or 3.0.2 (if it is available).

\end{document}