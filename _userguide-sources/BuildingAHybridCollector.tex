\setNextFileName{BuildingAHybridCollector.html}
\begin{section}{Building a Hybrid Collector}
\label{sec:buildingahybridcollector}

Extend the Tutorial plan to create a "copy-MS" collector, which allocates into a copying nursery and at collection time, copies nursery survivors into a mark-sweep space. This plan does not require a write barrier (it is not strictly generational, as it will collect the whole heap each time the heap is full). Later we will extended it with a write barrier, allowing the nursery to be collected in isolation. Such a collector would be a generational mark-sweep collector, similar to GenMS.

\begin{subsection}{Add a Copying Nursery}

This step will change your simple collector from using a bump pointer to a free list (but still without any garbage collection).

\begin{enumerate}
  \item In \spverb+TutorialConstraints+, make the following changes:
    \begin{enumerate}
      \item Override the \spverb+movesObjects()+ method to return \spverb+true+, reflecting that we are now building a copying collector:
        \begin{lstlisting}[language=Java]
@Override
public boolean movesObjects() { return true; }
        \end{lstlisting}
      \item Remove the restriction on default alloc bytes (since default allocation will now go to a bump-pointed space). To do this, remove the override of \spverb+maxNonLOSDefaultAllocBytes()+.
      \item Add a restriction on the maximum size that may be copied into the (default) non-LOS mature space:
        \begin{lstlisting}[language=Java]
@Override
public int maxNonLOSCopyBytes() { return SegregatedFreeListSpace.MAX_FREELIST_OBJECT_BYTES;}
        \end{lstlisting}
    \end{enumerate}
  \item In \spverb+Tutorial+, add a nursery space:
    \begin{enumerate}
      \item Create a new space, \spverb+nurserySpace+, of type \spverb+CopySpace+. The new space will initially be a from-space, so provide \spverb+false+ as the third argument. Initialize the space with a contiguous virtual memory region consuming 0.15 of the heap by passing "0.15" and "true" as arguments to the constructor of VMRequest(more on this later). Create and initialize a new integer constant to hold the descriptor for this new space:
        \begin{lstlisting}[language=Java]
public static final CopySpace nurserySpace = new CopySpace("nursery", false, VMRequest.highFraction(0.15f));
public static final int NURSERY = nurserySpace.getDescriptor();
        \end{lstlisting}
      \item Add the necessary import statements
      \item Add \spverb+nurserySpace+ to the \spverb+PREPARE+ and \spverb+RELEASE+ phases of \texttt{col\-lec\-tion\-Pha\-se()}, prior to the existing calls to msTrace. Pass true to nurserySpace.prepare() indicating that the nursery is a \textit{from-space} during collection.
      \item Fix accounting so that \spverb+Tutorial+ accounts for space consumed by \spverb+nurserySpace+:
         \begin{enumerate}
           \item Add \spverb+nurserySpace+ to the equation in \spverb+getPagesUsed()+
         \end{enumerate}
      \item Since initial allocation will be into a copying space, we need to account for copy reserve:
         \begin{enumerate}
           \item Add a method to override \texttt{get\-Col\-lec\-tion\-Re\-serve()} which returns \texttt{nur\-se\-ry\-Spa\-ce.re\-ser\-ved\-Pa\-ges() + su\-per.get\-Col\-lec\-tion\-Re\-ser\-ve()}
           \item Add a method to override \spverb+getPagesAvail()+, returning \texttt{get\-To\-tal\-Pa\-ges() - get\-Pa\-ges\-Re\-ser\-ved()) >> 1};
         \end{enumerate}
    \end{enumerate}
\end{enumerate}

\end{subsection}

\begin{subsection}{Add nursery allocation}

In \spverb+TutorialMutator+, replace the free-list allocator (\spverb+MarkSweepLocal+) with a nursery allocator: Use an instance of \spverb+CopyLocal+, calling it \spverb+nursery+. The constructor argument should be \texttt{Tu\-to\-rial.nur\-se\-ry\-Spa\-ce}:
  \begin{enumerate}
    \item change \spverb+alloc()+ to use \spverb+nursery.alloc()+ rather than \spverb+ms.alloc()+.
    \item remove the call to \spverb+msSpace.postAlloc()+ from \spverb+postAlloc()+ since there is no special post-allocation work necessary for the new copy space. The call to \spverb+super.postAlloc()+ should remain conditional on \spverb+allocator != Tutorial.ALLOC_DEFAULT+.
    \item change the check within \spverb+getAllocatorFromSpace()+ to check against \texttt{Tu\-to\-rial.nur\-se\-ry\-Spa\-ce} and to return \spverb+nursery+.
    \item adjust \spverb+collectionPhase+
      \begin{enumerate}
        \item replace call to \verb+ms.prepare()+ with \verb+nursery.reset()+
        \item remove call to \verb+ms.release()+ since there are no actions necessary for the nursery allocator upon release.
      \end{enumerate}
  \end{enumerate}
\end{subsection}

\begin{subsection}{Add copying to the collector}

In \spverb+TutorialCollector+ add the capacity for the collector to allocate (copy), since our new hybrid collector will perform copying.

\begin{enumerate}
  \item Add local allocators for both large object space and the mature space:
    \begin{lstlisting}[language=Java]
private final LargeObjectLocal los = new LargeObjectLocal(Plan.loSpace);
private final MarkSweepLocal mature = new MarkSweepLocal(Tutorial.msSpace);
    \end{lstlisting}
  \item Add an \spverb+allocCopy()+ method that conditionally allocates to the LOS or mature space:
    \begin{lstlisting}[language=Java]
@Override
public final Address allocCopy(ObjectReference original, int bytes,
                               int align, int offset, int allocator) {
  if (allocator == Plan.ALLOC_LOS)
    return los.alloc(bytes, align, offset);
  else
    return mature.alloc(bytes, align, offset);
}
    \end{lstlisting}
  \item Add a \spverb+postCopy()+ method that conditionally calls LOS or mature space post-copy actions:
    \begin{lstlisting}[language=Java]
@Override
public final void postCopy(ObjectReference object, ObjectReference typeRef,
                           int bytes, int allocator) {
if (allocator == Plan.ALLOC_LOS)
  Plan.loSpace.initializeHeader(object, false);
else
  Tutorial.msSpace.postCopy(object, true);
}
    \end{lstlisting}
\end{enumerate}

\end{subsection}

\begin{subsection}{Make necessary changes to TutorialTraceLocal}

\begin{enumerate}
  \item Add nurserySpace clauses to \spverb+isLive()+ and \spverb+traceObject()+:
     \begin{enumerate}
       \item Add the following to \spverb+isLive()+:
          \begin{lstlisting}[language=Java]
if (Space.isInSpace(Tutorial.NURSERY, object))
  return Tutorial.nurserySpace.isLive(object);
          \end{lstlisting}
       \item Add the following to traceObject():
          \begin{lstlisting}[language=Java]
if (Space.isInSpace(Tutorial.NURSERY, object))
  return Tutorial.nurserySpace.traceObject(this, object, Tutorial.ALLOC_DEFAULT);
          \end{lstlisting}
      \end{enumerate}
  \item Add a new \spverb+willNotMoveInCurrentCollection()+ method, which identifies those objects which do not move (necessary for copying collectors):
    \begin{lstlisting}[language=Java]
@Override
public boolean willNotMoveInCurrentCollection(ObjectReference object) {
  return !Space.isInSpace(Tutorial.NURSERY, object);
}
    \end{lstlisting}
  \item Modify \spverb+PlanSpecificConfig+ to add the new \spverb+nursery+ space:
    \begin{lstlisting}[language=Java]
new PlanSpecific("org.mmtk.plan.tutorial.Tutorial").addExpectedSpaces("ms", "nursery"), "Tutorial");
    \end{lstlisting}
\end{enumerate}

With these changes, \spverb+Tutorial+ should now work. You should be able to again build a \spverb+BaseBaseTutorial+ image and test it against any benchmark. Again, if you use \spverb+-X:gc:verbose=3+ you can see the movement of data among the spaces at each garbage collection.

\end{subsection}

\end{section}