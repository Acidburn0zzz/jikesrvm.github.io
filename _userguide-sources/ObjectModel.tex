\setNextFileName{ObjectModel.html}
\begin{section}{Object Model}
\label{sec:objectmodel}

An object model dictates how to represent objects in storage; the best object model will maximize efficiency of frequent language operations while minimizing storage overhead. Jikes RVM's object model is defined by ObjectModel.

\begin{subsection}{Overview}

Values in the Java™ programming language are either \textit{primitive} (e.g. \spverb+int+, \spverb+double+, etc.) or they are \textit{references} (that is, pointers) to objects. Objects are either \textit{arrays} having elements or \textit{scalar} objects having fields. Objects are logically composed of two primary sections: an object header (described in more detail below) and the object's instance fields (or array elements).

The following non-functional requirements govern the Jikes RVM object model:
\begin{itemize}
  \item instance field and array accesses should be as fast as possible,
  \item null-pointer checks should be performed by the hardware if possible,
  \item method dispatch and other frequent runtime services should be fast,
  \item other (less frequent) Java operations should not be prohibitively slow, and
  \item per-object storage overhead (ie object header size) should be as small as possible.
\end{itemize}

Assuming the reference to an object resides in a register, compiled code can access the object's fields at a fixed displacement in a single instruction. To facilitate array access, the reference to an array points to the first (zeroth) element of an array and the remaining elements are laid out in ascending order. The number of elements in an array, its \textit{length}, resides just before its first element. Thus, compiled code can access array elements via base + scaled index addressing.

The Java programming language requires that an attempt to access an object through a null object reference generates a \texttt{NullPointerException}. In Jikes RVM, references are machine addresses, and \spverb+null+ is represented by address $0$. On Linux, accesses to both very low and very high memory can be trapped by the hardware, thus all null checks can be made implicit.

\end{subsection}

\begin{subsection}{Object Header}

Logically, every object header contains the following components:
\begin{itemize}
  \item textbf{TIB Pointer:} The TIB (Type Information Block) holds information that applies to all objects of a type. The structure of the TIB is defined by \spverb+TIBLayoutConstants+. A TIB includes the virtual method table, a pointer to an object representing the type, and pointers to a few data structures to facilitate efficient interface invocation and dynamic type checking.
  \item textbf{Hash Code:} Each Java object has an identity hash code. This can be read by \spverb+Object.hashCode()+ or in the case that this method was overridden, by \spverb+System.identityHashCode+. The default hash code is usually the location in memory of the object, however, with some garbage collectors objects can move. So the hash code remains the same, space in the object header may be used to hold the original hash code value.
  \item textbf{Lock:} Each Java object has an associated lock state. This could be a pointer to a lock object or a direct representation of the lock.
  \item textbf{Array Length:} Every array object provides a length field that contains the length (number of elements) of the array.
  \item textbf{Garbage Collection Information:} Each Java object has associated information used by the memory management system. Usually this consists of one or two mark bits, but this could also include some combination of a reference count, forwarding pointer, etc.
  \item textbf{Misc Fields:} In experimental configurations, the object header can be expanded to add additional fields to every object, typically to support profiling.
\end{itemize}

An implementation of this abstract header is defined by two files:
\begin{itemize}
  \item \spverb+JavaHeader+, which supports TIB access, default hash codes, and locking. It also provides a few bits for use by the memory management subsystem.
  \item \spverb+MiscHeader+, which supports adding additional fields to all objects.
\end{itemize}

Information that is specific to garbage collection uses the available bits from the Java header. Depending on the chosen garbage collector, the available bits can be accessed via an appropriate class, e.g.:
\begin{itemize}
  \item \spverb+HeaderByte+ which provides access to methods for logging and unlogging for various collectors
  \item \spverb+RCHeader+ for reference counting garbage collectors
  \item \spverb+ForwardingWord+ which provides methods for object forwarding which is used by some copying collectors
\end{itemize}

\end{subsection}

\begin{subsection}{Field Layout}

Fields tend to be recorded in the Java class file in the order they are declared in the Java source file. We lay out fields in the order they are declared with some exceptions to improve alignment and pack the fields in the object.

Fields of type \spverb+double+ and \spverb+long+ benefit from being $8$ byte aligned. Every \spverb+RVMClass+ records the preferred alignment of the object as a whole. We lay out \spverb+double+ and \spverb+long+ fields first (and object references if these are $8$ bytes long) so that we can avoid making holes in the field layout for alignment. We don't do this for smaller fields as all objects need to be a multiple of $4$ bytes in size.

When we lay out fields we may create holes to improve alignment. For example, an \spverb+int+ following a \spverb+byte+, we'll create a $3$ byte hole following the \spverb+byte+ to keep the \spverb+int+ $4$ byte aligned. Holes in the field layout can be $1$, $2$ or $4$ bytes in size. As fields are laid out, holes are used to avoid increasing the size of the object. Sub-classes inherit the hole information of their parent, so holes in the parent object can be reused by their children. 

\end{subsection}

\end{section}
