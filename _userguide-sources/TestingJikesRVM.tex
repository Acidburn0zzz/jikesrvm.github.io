\setNextFileName{TestingJikesRVM.html}
\begin{section}{Testing Jikes RVM}
\label{sec:testingjikesrvm}

Jikes RVM includes provisions to run unit tests as well as functional and performance tests. It also includes a number of actual tests, both unit and functional ones.

\begin{subsection}{Unit Tests}

Jikes RVM makes writing simple unit tests easy. Simply give your JUnit 4 tests a name ending in Test and place test sources under \spverb+rvm/test-src+. The tests will be picked up automatically.

The tests are then run on the bootstrap VM, i.e. the JVM used to build Jikes RVM. You can also \hyperref[sec:buildingjikesrvm]{configure the build} to run unit tests on the newly built Jikes RVM. Note that this may significantly increase the build times of slow configurations (e.g. prototype and protype-opt).

If you are developing new unit tests, it may be helpful to run them on an existing Jikes RVM image. This can be done by using the Ant target \spverb+unit-tests-on-existing-image+. The path for the image is determined by the usual properties of the Ant build.
\end{subsection}

\begin{subsection}{Functional and Performance Tests}

See \hyperref[sec:externaltestresources]{External Test Resources} for details or downloading prerequisites for the functional tests. The tests are executed using an Ant build file and produce results that conform to the definition below. The results are aggregated and processed to produce a high level report defining the status of Jikes RVM.

The testing framework was designed to support continuous and periodical execution of tests. A \textit{"test-run"} occurs every time the testing framework is invoked. Every \textit{"test-run"} will execute one or more \textit{"test-configuration"}s. A \textit{"test-configuration"} defines a particular build \textit{"configuration"} (See \hyperref[sec:configuringjikesrvm]{Configuring Jikes RVM} for details) combined with a set of parameters that are passed to the RVM during the execution of the tests. i.e. a particular \textit{"test-configuration"} may pass parameters such as \texttt{-X:aos:enable\_recompilation=false -X:aos:initial\_compiler=opt -X:irc:O1} to test the Level 1 Opt compiler optimizations.

Every \textit{"test-configuration"} will execute one or more \textit{"group"}s of tests. Every \textit{"group"} is defined by a Ant build.xml file in a separate sub-directory of \spverb+$RVM_ROOT/testing/tests+. Each \textit{"test"} has a number of input parameters such as the classname to execute, the parameters to pass to the RVM or to the program. The \textit{"test"} records a number of values such as execution time, exit code, result, standard output etc. and may also record a number of statistics if it is a performance test.

The project includes several different types of test runs and the description of each the test runs and their purpose is given in \hyperref[sec:testrundescriptions]{Test Run Descriptions}.

Note that the \hyperref[sec:usingbuildit]{buildit script} provides a fast and easy way to build and the system.  The script is simply a wrapper around the mechanisms described below.

\begin{subsubsection}{Ant properties}

There is a number of ant properties that control the test process. Besides the properties that are already defined in \hyperref[sec:buildingjikesrvm]{Building Jikes RVM}, special test properties may also be specified.


\begin{table}
\centering
\begin{tabular}{p{0.2\linewidth}p{0.5\linewidth}p{0.3\linewidth}}
Property & Description & Default \\
test-run.name & The name of the test-run. The name should match one of the files located in the build/test-runs/ directory minus the '.properties' extension. & pre-commit \\
results.dir & The directory where Ant stores the results of the test run. & \$\{jikes\-rvm.dir\}/\newline re\-sults \\
results.archive & The directory where Ant gzips and archives a copy of test run results and reports. & \$\{re\-sults.dir\}/\newline archive \\
send.reports & Define this property to send reports via email. & (Undefined) \\
mail.from & The from address used when emailing report. & jikesrvm-core@\newline lists.sourceforge.\newline net \\
mail.to & The to address used when emailing report. & jikesrvm-\newline regression@\newline lists.sourceforge.\newline net \\
mail.host & The host to connect to when sending mail. & localhost \\
mail.port & The port to connect to when sending mail. & 25 \\
\textless configuration\textgreater .\newline built & If set to true, the test process will skip the build step for specified configurations. For the test process to work the build must already be present. & (Undefined) \\
skip.build & If defined the test process will skip the build step for all configurations and the javadoc generation step. For the test process to work the build must already be present. & (Undefined) \\
skip.javadoc & If defined the test process will skip the javadoc generation step. & (Undefined) \\
\end{tabular}
\caption{Test properties}
\end{table}

\end{subsubsection}

\begin{subsubsection}{Defining a test-run}

A \textit{test-run} is defined by a number of properties located in a property file located in the \spverb+build/test-runs/+ directory.

The property test.configs is a whitespace separated list of test-configuration "tags". Every tag uniquely identifies a particular test-configuration. Every test-configuration is defined by a number of properties in the property file that are prefixed with test.config.\textless tag\textgreater . See the test run property table for the possible properties.

\begin{table}
\centering
\begin{tabular}{p{0.2\linewidth}p{0.6\linewidth}p{0.2\linewidth}}
Property & Description & Default \\
tests & The names of the test groups to execute. & None \\
name & The unique identifier for test-configuration. & "" \\
configuration & The name of the RVM build configuration to test. & \textless tag\textgreater  \\
target & The name of the RVM build target. This can be used to trigger compilation of a profiled image & "main" \\
mode & The test mode. May modify the way test groups execute. See individual groups for details. & "" \\
extra.rvm.args & Extra arguments that are passed to the RVM. These may be varied for different runs using the same image. & "" \\
\end{tabular}
\caption{Test run properties}
\end{table}

The order of the test-configurations in \textit{test.configs} is the order that the test-configurations are tested. The order of the groups in \textit{test.config.\textless tag\textgreater .test} is the order that the tests are executed.

The simplest test-run is defined in the following figure. It will use the build configuration \textit{"prototype"} and execute tests in the \textit{"basic"} group.

\begin{lstlisting}[title=build/test-runs/simple.properties]
test.configs=prototype
test.config.prototype.tests=basic
\end{lstlisting}

The test process also expands properties in the property file so it is possible to define a set of tests once but use them in multiple test-configurations as occurs in the following figure. The groups basic, optests and dacapo are executed in both the prototype and prototype-opt test configurations.

\begin{lstlisting}[title=build/test-runs/property-expansion.properties]
test.set=basic optests dacapo
test.configs=prototype prototype-opt
test.config.prototype.tests=${test.set}
test.config.prototype-opt.tests=${test.set}
\end{lstlisting}

Each test can have additional parameters specified that will be used by the test infrastructure when starting the Jikes RVM instance to execute the test. These additional parameters are described in table for test specific parameters.

\begin{table}
\centering
\begin{tabular}{p{0.2\linewidth}p{0.3\linewidth}p{0.2\linewidth}p{0.3\linewidth}}
Parameter & Description & Default Property & Default Value \\
initial.heapsize & The initial size of the heap. & \$\{test.initial.heapsize\} & \$\{config.default-heapsize.initial\} \\
max.heapsize & The initial size of the heap. & \$\{test.max.heapsize\} & \$\{config.default-heapsize.maximum\} \\
max.opt.level & The maximum optimization level for the tests or an empty string to use the Jikes RVM default. & \$\{test.max.opt.level\} & "" \\
processors & The number of processors to use for garbage collection for the test or 'all' to use all available processors. & \$\{test.processors\} & all \\
time.limit & The time limit for the test in seconds. After the time limit expires the Jikes RVM instance will be forcefully terminated. & \$\{test.time.limit\} & 1000 \\
class.path & The class path for the test. & \$\{test.class.path\} & \\
extra.args & Extra arguments that are passed to Jikes RVM. & \$\{test.rvm.extra.args\} & "" \\
exclude & If set to true, the test will be not be executed. & "" \\
\end{tabular}
\caption{Test specific parameters}
\end{table}


To determine the value of a test specific parameters, the following mechanism is used:

\begin{enumerate}
  \item    Search for one of the the following ant properties, in order.
    \begin{enumerate}
        \item test.config.\textless build-configuration\textgreater .\textless group\textgreater .\textless test\textgreater .\textless parameter\textgreater 
        \item test.config.\textless build-configuration\textgreater .\textless group\textgreater .\textless parameter\textgreater 
        \item test.config.\textless build-configuration\textgreater .\textless parameter\textgreater 
        \item test.config.\textless build-configuration\textgreater .\textless group\textgreater .\textless test\textgreater .\textless parameter\textgreater 
        \item test.config.\textless build-configuration\textgreater .\textless group\textgreater .\textless parameter\textgreater 
    \end{enumerate}
  \item If none of the above properties are defined then use the parameter that was passed to the \textless rvm\textgreater\ macro in the ant build file.
  \item If no parameter was passed to the \textless rvm\textgreater\ macro then use the default value which is stored in the "Default Property" as specified in the above table. By default the value of the "Default Property" is specified as the "Default Value" in the above table, however a particular build file may specify a different "Default Value".
\end{enumerate}

\end{subsubsection}

\begin{subsubsection}{Excluding tests}

Sometimes it is desirable to exclude tests. The test exclusion may occur as the test is known to fail on a particular target platform, build configuration or maybe it just takes too long. To exclude a test, you must define the test specific parameter "exclude" to true either in .ant.properties or in the test-run properties file.

For example, at the time of writing the Jikes RVM does not fully support volatile fields and as a result the test named "TestVolatile" in the "basic" group will always fail. Rather than being notified of this failure we can disable the test by adding a property such as "test.config.basic.TestVolatile.exclude=true" into test-run properties file.

\end{subsubsection}

\begin{subsubsection}{Executing a test-run}

The tests are executed by the Ant driver script \textit{test.xml}. The \textit{test-run.name} property defines the particular test-run to execute and if not set defaults to \textit{"sanity"}. The command \spverb+ant -f test.xml -Dtest-run.name=simple+ executes the test-run defined in \textit{build/test-runs/simple.properties}. When this command completes you can point your browser at \newline \spverb+${results.dir}/tests/${test-run.name}/Report.html+ to get an overview on test run or at \newline \spverb+${results.dir}/tests/${test-run.name}/Report.xml+ for an XML document describing test results.

\end{subsubsection}

\end{subsection}

\end{section}