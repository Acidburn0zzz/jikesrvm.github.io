% TODO convert to LaTeX. code blocks can probably be done via (lst)listings, rest shouldn't be a problem
\begin{section}{IR}

The optimizing compiler intermediate representation (IR) is held in an object of type \verb#IR# and includes a list of instructions. Every instruction is classified into one of the pre-defined instruction formats. Each instruction includes an operator and zero or more operands. Instructions are grouped into basic blocks; basic blocks are constrained to having control-flow instructions at their end. Basic blocks fall-through to other basic blocks or contain branch instructions that have a destination basic block label. The graph of basic blocks is held in the \verb#cfg# (control-flow graph) field of IR.

% TODO link to the intermediate code representation in the final version of this page
This section documents basic information about the intermediate instruction. For a tutorial based introduction to the material it is highly recommended that you read "Jikes RVM Optimizing Compiler Intermediate Code Representation".

\begin{subsection}{IR Operators}

The IR operators are defined by the class \verb#Operators#, which in turn is automatically generated from a template by a driver. The input to the driver are two files, both called \verb#OperatorList.dat#. One input file resides in
\verb#$RVM_ROOT/rvm/src-generated/opt-ir# and defines machine-independent operators. The other resides in
\verb#$RVM_ROOT/rvm/src-generated/opt-ir/$\{arch\}# and defines machine-dependent operators, where \verb#$\{arch\}# is the specific instruction architecture of interest.

Each operator in \verb#OperatorList.dat# is defined by a five-line record, consisting of:

\begin{itemize}
  \item \verb#SYMBOL#: a static symbol to identify the operator
  \item \verb#INSTRUCTION_FORMAT#: the instruction format class that accepts this operator.
  \item \verb#TRAITS#: a set of characteristics of the operator, composed with a bit-wise or (\textbar ) operator. See Operator.java for a list of valid traits.
  \item \verb#IMPLDEFS#: set of registers implicitly defined by this operator; usually applies only to machine-dependent operators
  \item \verb#IMPLUSES#: set of registers implicitly used by this operator; usually applies only to machine-dependent operators
\end{itemize}

For example, the entry in \verb#OperatorList.dat# that defines the integer addition operator is
\begin{lstlisting}
INT_ADD
Binary
none
<blank line>
<blank line>
\end{lstlisting}

The operator for a conditional branch based on values of two references is defined by
\begin{lstlisting}
REF_IFCOMP
IntIfCmp
branch | conditional
<blank line>
<blank line>
\end{lstlisting}
Additionally, the machine-specific OperatorList.dat file contains another line of information for use by the assembler. See the file for details.

\end{subsection}


\begin{subsection}{Instruction Format}

Every IR instruction fits one of the pre-defined Instruction Formats. The Java package \verb#org.jikesrvm.compilers.opt.ir# defines roughly 75 architecture-independent instruction formats. For each instruction format, the package includes a class that defines a set of static methods by which optimizing compiler code can access an instruction of that format.

For example, \verb#INT_MOVE# instructions conform to the \verb#Move# instruction format. The following code fragment shows code that uses the \verb#Operators# interface and the \verb#Move# instruction format:

\begin{lstlisting}[language=Java]
import org.jikesrvm.compilers.opt.ir.*;
class X {
  void foo(Instruction s) {
    if (Move.conforms(s)) {     // if this instruction fits the Move format
      RegisterOperand r1 = Move.getResult(s);
      Operand r2 = Move.getVal(s);
      System.out.println("Found a move instruction: " + r1 + " := " + r2);
    } else {
      System.out.println(s + " is not a MOVE");
    }
  }
}
\end{lstlisting}

This example shows just a subset of the access functions defined for the Move format. Other static access functions can set each operand (in this case, \verb#Result# and \verb#Val#), query each operand for nullness, clear operands, create Move instructions, mutate other instructions into Move instructions, and check the index of a particular operand field in the instruction. See the Javadoc™ reference for a complete description of the API.

Each fixed-length instruction format is defined in the text file \verb#$RVM_ROOT/rvm/src-generated/opt-ir/InstructionFormatList.dat#. Each record in this file has four lines:

\begin{itemize}
\item \verb#NAME#: the name of the instruction format
\item \verb#SIZES#: the number of operands defined, defined and used, and used
\item \verb#SIZES#: the number of operands defined, defined and used, and used
      \begin{itemize}
        \item \verb#D/DU/U#: Is this operand a def, use, or both?
        \item \verb#NAME#: the unique name to identify the operand
        \item \verb#TYPE#: the type of the operand (a subclass of Operand)
        \item \verb#[opt]#: is this operand optional?
      \end{itemize}
\item \verb#VARSIG#: a description of repeating operands, used for variable-length instructions.
\end{itemize}


So for example, the record that defines the Move instruction format is

TODO the original page has more text but no new formatting elements.

\end{subsection}

\end{section}